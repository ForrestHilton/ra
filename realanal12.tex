\documentclass[12pt]{book}

%For sending to printer such as lulu/createspace, search for %PRINT

%See text for 
%FIXMEevillayouthack
% for evil layout hacks

%\usepackage{pdf14}
%Paper saving
%\documentclass[12pt,openany]{book}
%\documentclass[10pt,openany]{book}
%\documentclass[8pt,openany]{extbook}
%if using smaller font you should also rerun the figures,
% that is in figures/ run run-all-xp.sh (must have epix installed)

\usepackage[T1]{fontenc}

% Footnotes should use symbols, not numbers.  Numbered footnotes are
% evil
\usepackage[perpage,symbol*]{footmisc}

\usepackage[shortlabels]{enumitem}
\usepackage{ifpdf}
\usepackage{amsmath}
\usepackage{amsfonts}
\usepackage{amssymb}
\usepackage{amsthm}
\usepackage{graphicx}
\usepackage{graphbox}
\usepackage{color}

% Remove page numbers on chapter opens
% must come before fullpage
\usepackage{nopageno}

\usepackage[headings]{fullpage}

% smaller margins top/bottom margins
\addtolength{\textheight}{0.8in}
\addtolength{\topmargin}{-0.3in}

% 6x9 version?
%\addtolength{\textheight}{1.2in}
%\addtolength{\paperheight}{1.2in}

%PRINT
%First offset even/odd pages a bit
% not for the coil bound full letter size one
%\addtolength{\oddsidemargin}{0.2in}
%\addtolength{\evensidemargin}{-0.2in}

%PRINT
%Now cut page size a bit.  I'll run it through ghostcript anyway
%to convert to the right size, but this is good for crown quatro
%conversion, don't use for the full letter size versions
%\addtolength{\paperwidth}{-0.3in}
%\addtolength{\paperheight}{-1in}
%\addtolength{\topmargin}{-0.5in}
%\addtolength{\oddsidemargin}{-0.15in}
%\addtolength{\evensidemargin}{-0.15in}

% discourage pagebreak at end of display, put before \end{equation}
\newcommand{\avoidbreak}{\postdisplaypenalty=100}

\usepackage{varioref}
\usepackage{imakeidx}
\PassOptionsToPackage{hyphens}{url}
\usepackage{hyperref} % do NOT set [ocgcolorlinks] here!

%If you have an older tex installation you might need
%to comment out the next line:
%PRINT for a printer (lulu/createspace) run without
\usepackage[ocgcolorlinks]{ocgx2}

%\usepackage[all]{hypcap} %handled by caption
\usepackage[shortalphabetic]{amsrefs}
\usepackage{nicefrac}
\usepackage{microtype}
\usepackage{import}

%\usepackage{draftwatermark}
%\SetWatermarkText{Draft of v5.0 as of \today. May change substantially!}
%\SetWatermarkAngle{90}
%\SetWatermarkHorCenter{0.5in}
%\SetWatermarkColor[gray]{0.7}
%\SetWatermarkScale{0.18}

\usepackage[margin=10pt,font=small,labelfont=bf,labelsep=colon,singlelinecheck=false]{caption}

\usepackage{tikz}
\usetikzlibrary{cd}
%\usepackage{rotating}

\usepackage{cellspace}
\usepackage[toc,nopostdot,sort=use,nomain,automake]{glossaries}

\usepackage{vogtwidebar}

%%%
%NOTE: If changing fonts, also change fonts in figures/run-xp-topdf.sh
% AND then rerun run-all-xp.sh (must have epix installed)
%%%
% Times
%\usepackage{txfonts}
% Times, but symbol/cm/ams math fonts
\usepackage{mathptmx}
% But we do want helvetica for sans
\usepackage{helvet}

%enumitem global options
\setlist{leftmargin=*,itemsep=0.5\itemsep,parsep=0.5\parsep,topsep=0.5\topsep,partopsep=0.5\partopsep}

\newenvironment{myfigureht}{%
\begin{figure}[h!t]
\noindent\rule{\textwidth}{0.4pt}\vspace{12pt}\par\centering}%
{\par\noindent\rule{\textwidth}{0.4pt}
\end{figure}}

% useful
%mbxSTARTIGNORE
\newcommand{\ignore}[1]{}
%mbxENDIGNORE

% analysis/geometry stuff
\newcommand{\ann}{\operatorname{ann}}
\renewcommand{\Re}{\operatorname{Re}}
\renewcommand{\Im}{\operatorname{Im}}
\newcommand{\Orb}{\operatorname{Orb}}
\newcommand{\hol}{\operatorname{hol}}
\newcommand{\aut}{\operatorname{aut}}
\newcommand{\codim}{\operatorname{codim}}
\newcommand{\sing}{\operatorname{sing}}

% reals
\newcommand{\esssup}{\operatorname{ess~sup}}
\newcommand{\essran}{\operatorname{essran}}
\newcommand{\innprod}[2]{\langle #1 | #2 \rangle}
\newcommand{\linnprod}[2]{\langle #1 , #2 \rangle}
\newcommand{\supp}{\operatorname{supp}}
\newcommand{\Nul}{\operatorname{Nul}}
\newcommand{\Ran}{\operatorname{Ran}}
\newcommand{\sabs}[1]{\lvert {#1} \rvert}
\newcommand{\snorm}[1]{\lVert {#1} \rVert}
\newcommand{\abs}[1]{\left\lvert {#1} \right\rvert}
\newcommand{\norm}[1]{\left\lVert {#1} \right\rVert}
\newcommand{\babs}[1]{\bigl\lvert {#1} \bigr\rvert}
\newcommand{\bnorm}[1]{\bigl\lVert {#1} \bigr\rVert}

% sets (some)
\newcommand{\C}{{\mathbb{C}}}
\newcommand{\R}{{\mathbb{R}}}
\newcommand{\Z}{{\mathbb{Z}}}
\newcommand{\N}{{\mathbb{N}}}
\newcommand{\Q}{{\mathbb{Q}}}
\newcommand{\D}{{\mathbb{D}}}
\newcommand{\F}{{\mathbb{F}}}

% consistent
\newcommand{\bB}{{\mathbb{B}}}
\newcommand{\bC}{{\mathbb{C}}}
\newcommand{\bR}{{\mathbb{R}}}
\newcommand{\bZ}{{\mathbb{Z}}}
\newcommand{\bN}{{\mathbb{N}}}
\newcommand{\bQ}{{\mathbb{Q}}}
\newcommand{\bD}{{\mathbb{D}}}
\newcommand{\bF}{{\mathbb{F}}}
\newcommand{\bH}{{\mathbb{H}}}
\newcommand{\bO}{{\mathbb{O}}}
\newcommand{\bP}{{\mathbb{P}}}
\newcommand{\bK}{{\mathbb{K}}}
\newcommand{\bV}{{\mathbb{V}}}
\newcommand{\CP}{{\mathbb{CP}}}
\newcommand{\RP}{{\mathbb{RP}}}
\newcommand{\HP}{{\mathbb{HP}}}
\newcommand{\OP}{{\mathbb{OP}}}
\newcommand{\sA}{{\mathcal{A}}}
\newcommand{\sB}{{\mathcal{B}}}
\newcommand{\sC}{{\mathcal{C}}}
\newcommand{\sF}{{\mathcal{F}}}
\newcommand{\sG}{{\mathcal{G}}}
\newcommand{\sH}{{\mathcal{H}}}
\newcommand{\sM}{{\mathcal{M}}}
\newcommand{\sO}{{\mathcal{O}}}
\newcommand{\sP}{{\mathcal{P}}}
\newcommand{\sQ}{{\mathcal{Q}}}
\newcommand{\sR}{{\mathcal{R}}}
\newcommand{\sS}{{\mathcal{S}}}
\newcommand{\sI}{{\mathcal{I}}}
\newcommand{\sL}{{\mathcal{L}}}
\newcommand{\sK}{{\mathcal{K}}}
\newcommand{\sU}{{\mathcal{U}}}
\newcommand{\sV}{{\mathcal{V}}}
\newcommand{\sX}{{\mathcal{X}}}
\newcommand{\sY}{{\mathcal{Y}}}
\newcommand{\sZ}{{\mathcal{Z}}}
\newcommand{\fS}{{\mathfrak{S}}}

\newcommand{\interior}{\operatorname{int}}

% Topo stuff
\newcommand{\id}{\textit{id}}
\newcommand{\im}{\operatorname{im}}
\newcommand{\rank}{\operatorname{rank}}
\newcommand{\Tor}{\operatorname{Tor}}
\newcommand{\Torsion}{\operatorname{Torsion}}
\newcommand{\Ext}{\operatorname{Ext}}
\newcommand{\Hom}{\operatorname{Hom}}

%extra thingies
\newcommand{\mapsfrom}{\ensuremath{\text{\reflectbox{$\mapsto$}}}}
\newcommand{\from}{\ensuremath{\leftarrow}}
\newcommand{\dhat}[1]{\hat{\hat{#1}}}
\newcommand{\spn}{\operatorname{span}}

% San Serif fonts
%\renewcommand{\familydefault}{\sfdefault}

% To allow skrinking to 5.5 x 8.5 inches without whitespaces
% Make sure to rerun makeindex and makeglossaries as well
% Useful for printing on lulu.com and saving on paper
%\addtolength{\textheight}{2.13in}
%\addtolength{\paperheight}{2.13in}

\definecolor{mypersianblue}{rgb}{0.11, 0.22, 0.73}

\hypersetup{
    pdfborderstyle={/S/U/W 0.5}, %this just in case ocg isn't there
    %PRINT do this (and comment out the above):
    %pdfborder={0 0 0},
    citecolor=mypersianblue,
    filecolor=mypersianblue,
    linkcolor=mypersianblue,
    urlcolor=mypersianblue,
    pdfkeywords={real analysis, Riemann integral, derivative, limit, sequence},
    pdfsubject={Real Analysis},
    pdftitle={Basic Analysis: Introduction to Real Analysis},
    pdfauthor={Jiri Lebl}
}

% This is really just for marking these
%so that they are easy to find
% The second argument is the thing for the both volumes together thing
%mbxSTARTIGNORE
\newcommand{\volIref}[2]{#2\@}
%mbxENDIGNORE

% Set up our index
\makeindex

%mbxSTARTIGNORE

% Very simple indexing
\newcommand{\myindex}[1]{#1\index{#1}}

% Quotes
\newcommand{\myquote}[1]{``#1''}

% define this to be empty to kill notes
\newcommand{\sectionnotes}[1]{\noindent \emph{Note: #1} \medskip \par}

% Define this to be empty to not skip page before the sections to
% save some paper
\newcommand{\sectionnewpage}{\clearpage}
%\newcommand{\sectionnewpage}{}

%mbxENDIGNORE

\author{Ji\v{r}\'i Lebl}

\title{Basic Analysis: Introduction to Real Analysis}

% Don't include subsections
\setcounter{tocdepth}{1}

\theoremstyle{plain}
\newtheorem{thm}{Theorem}[section]
\newtheorem{lemma}[thm]{Lemma}
\newtheorem{prop}[thm]{Proposition}
\newtheorem{cor}[thm]{Corollary}

\theoremstyle{remark}
\newtheorem{remark}[thm]{Remark}

\theoremstyle{definition}
\newtheorem{defn}[thm]{Definition}

\newtheoremstyle{exercise}% name
  {}% Space above
  {}% Space below
  {\itshape \small}% Body font
  {}% Indent amount 1
  {\bfseries \itshape \small}% Theorem head font
  {:}% Punctuation after theorem head
  {.5em}% Space after theorem head 2
  {}% Theorem head spec (can be left empty, meaning "normal")

\newenvironment{exnote}{\small}{}

\theoremstyle{exercise}
\newtheorem{exercise}{Exercise}[section]

\newtheoremstyle{example}% name
  {}% Space above
  {}% Space below
  {}% Body font
  {}% Indent amount 1
  {\bfseries}% Theorem head font
  {:}% Punctuation after theorem head
  {.5em}% Space after theorem head 2
  {}% Theorem head spec (can be left empty, meaning "normal")

\theoremstyle{example}
\newtheorem{example}[thm]{Example}

% referencing
%mbxSTARTIGNORE
\newcommand{\figureref}[1]{\hyperref[#1]{Figure~\ref*{#1}}}
\newcommand{\tableref}[1]{\hyperref[#1]{Table~\ref*{#1}}}
\newcommand{\chapterref}[1]{\hyperref[#1]{chapter~\ref*{#1}}}
\newcommand{\Chapterref}[1]{\hyperref[#1]{Chapter~\ref*{#1}}}
\newcommand{\sectionref}[1]{\hyperref[#1]{\S\ref*{#1}}}
\newcommand{\exerciseref}[1]{\hyperref[#1]{Exercise~\ref*{#1}}}
\newcommand{\remarkref}[1]{\hyperref[#1]{Remark~\ref*{#1}}}
\newcommand{\exampleref}[1]{\hyperref[#1]{Example~\ref*{#1}}}
\newcommand{\thmref}[1]{\hyperref[#1]{Theorem~\ref*{#1}}}
\newcommand{\propref}[1]{\hyperref[#1]{Proposition~\ref*{#1}}}
\newcommand{\lemmaref}[1]{\hyperref[#1]{Lemma~\ref*{#1}}}
\newcommand{\corref}[1]{\hyperref[#1]{Corollary~\ref*{#1}}}
\newcommand{\defnref}[1]{\hyperref[#1]{Definition~\ref*{#1}}}
%mbxENDIGNORE


% List of Symbols/Notation
\newglossary[nlg]{notation}{not}{ntn}{List of Notation}

\loadglsentries{notations}
\makeglossaries

%mbx <!--BEFORE book-->

%mbxdocinfo    <website>
%mbxdocinfo        <title>www.jirka.org/ra/</title>
%mbxdocinfo        <url>https://www.jirka.org/ra/</url>
%mbxdocinfo    </website>
%mbxdocinfo
%mbxdocinfo    <brandlogo source="logo.png" url="https://www.jirka.org/ra/" />
%mbxdocinfo
%mbxdocinfo    <rename element="inlineexercise">Exercise</rename>
%mbxdocinfo    <rename element="divisionalexercise">Exercise</rename>
%mbxdocinfo
%mbxdocinfo    <latex-preamble>
%mbxdocinfo        <package>cancel</package>
%mbxdocinfo    </latex-preamble>
%mbxdocinfo
%mbxdocinfo    <initialism>RA</initialism>


%mbxmacro \newcommand{\nicefrac}[2]{{{}^{#1}}\!/\!{{}_{#2}}}
%mbxmacro \newcommand{\unitfrac}[3][\!\!]{#1 \,\, {{}^{#2}}\!/\!{{}_{#3}}}
%mbxmacro \newcommand{\unit}[2][\!\!]{#1 \,\, #2}
%mbxmacro \newcommand{\noalign}[1]{}
%mbxmacro \newcommand{\qed}{\qquad \Box}
%mbxmacro \newcommand{\qedhere}{}
%mbxmacro \newcommand{\widebar}[1]{\overline{#1}}

\begin{document}

%mbx <title>Basic Analysis I &amp; II</title>
%mbx <subtitle>Introduction to Real Analysis, Volumes I &amp; II</subtitle>
%mbx
%mbx <frontmatter>
%mbx   <titlepage>
%mbx
%mbx     <author>
%mbx       <personname>Jiří Lebl</personname>
%mbx       <department>Department of Mathematics</department>
%mbx       <institution>Oklahoma State University</institution>
%mbx       <email>jiri.lebl@gmail.com</email>
%mbx     </author>
%mbx
%mbx     <date><today /></date>
%mbx
%mbx   </titlepage>
%mbx
%mbx   <colophon>
%mbx
%mbx     <copyright>
%mbx       <year>2009<ndash />2021</year>
%mbx       <holder>Jiří Lebl</holder>
%mbx       <minilicense>Creative Commons Attribution-Non-commercial-Share Alike 4.0 International License and the Creative Commons Attribution-Share Alike 4.0 International License</minilicense>
%mbx       <shortlicense>
%mbx         This work is dual licensed under the Creative Commons Attribution-Non-commercial-Share Alike 4.0 International License and
%mbx         the Creative Commons Attribution-Share Alike 4.0 International License.  To view a copy of these licenses, visit
%mbx         <url href="https://creativecommons.org/licenses/by-nc-sa/4.0/">https://creativecommons.org/licenses/by-nc-sa/4.0/</url> or
%mbx         <url href="https://creativecommons.org/licenses/by-sa/4.0/">https://creativecommons.org/licenses/by-sa/4.0/</url>
%mbx         or send a letter to Creative Commons PO Box 1866, Mountain View, CA 94042, USA.
%mbx       </shortlicense>
%mbx     </copyright>
%mbx     <p>
%mbx        Version 5.5 (volume I) and 2.5 (volume II)
%mbx     </p>
%mbx     <p>
%mbx       You can use, print, duplicate, share this book as much as you want.  You can
%mbx       base your own notes on it and reuse parts if you keep the license the
%mbx       same.  You can assume the license is either the CC-BY-NC-SA or CC-BY-SA,
%mbx       whichever is compatible with what you wish to do, your derivative works must
%mbx       use at least one of the licenses.
%mbx       Derivative works must be prominently marked as such.
%mbx     </p>
%mbx     <p>
%mbx       During the writing of this book, 
%mbx       the author was in part supported by NSF grant DMS-0900885 and
%mbx       DMS-1362337.
%mbx     </p>
%mbx     <p>
%mbx       The date is the main identifier of version.  The major version / edition
%mbx       number is raised only if there have been substantial changes.
%mbx       Each volume has its own version number.  Edition number started at 4,
%mbx       that is, version 4.0, as it was not kept track of before.
%mbx     </p>
%mbx     <p>
%mbx       See <url href="https://www.jirka.org/ra/">https://www.jirka.org/ra/</url>
%mbx       for more information
%mbx       (including contact information).
%mbx       The LaTeX source for the book is available for possible modification and customization
%mbx       at github: <url href="https://github.com/jirilebl/ra">https://github.com/jirilebl/ra</url>
%mbx     </p>
%mbx
%mbx   </colophon>
%mbx </frontmatter>

%mbxSTARTIGNORE

\ifpdf
  \pdfbookmark{Title Page}{title}
\fi
\newlength{\centeroffset}
\setlength{\centeroffset}{-0.5\oddsidemargin}
\addtolength{\centeroffset}{0.5\evensidemargin}
%\addtolength{\textwidth}{-\centeroffset}
\thispagestyle{empty}
\vspace*{\stretch{1}}
\noindent\hspace*{\centeroffset}\makebox[0pt][l]{\begin{minipage}{\textwidth}
\flushright
{\Huge\bfseries \sffamily Basic Analysis I \& II}
\noindent\rule[-1ex]{\textwidth}{5pt}\\[2.5ex]
\hfill\emph{\Large \sffamily Introduction to Real Analysis, Volumes I \& II}
\end{minipage}}

\vspace{\stretch{1}}
\noindent\hspace*{\centeroffset}\makebox[0pt][l]{\begin{minipage}{\textwidth}
\flushright
{\bfseries 
by Ji{\v r}\'i Lebl\\[3ex]} 
\today
\\
(version 5.5 and 2.5)
% --- 5th edition, 5th update)
% --- 2nd edition, 5th update)
\end{minipage}}

%\addtolength{\textwidth}{\centeroffset}
\vspace{\stretch{2}}


\pagebreak

\vspace*{\fill}

\noindent
Typeset in \LaTeX.


\bigskip

\noindent
Copyright \copyright 2009--2021 Ji{\v r}\'i Lebl

%PRINT
% not for lulu
%\noindent
%ISBN-13: 978-1718862401
%
%\noindent
%ISBN-10: 1718862407

%PRINT
% not for lulu
%\medskip
%\noindent
%Cover image: Hana Highway, Maui, Hawaii, \copyright 2008 Ji{\v r}\'i Lebl, all rights reserved.  Cover
%image cannot be reused in derivative works.

\bigskip

\noindent
\includegraphics[width=1.38in]{figures/license}
\quad
\includegraphics[width=1.38in]{figures/license2}

\bigskip

\noindent
This work
%PRINT
% not for lulu
%(except the cover art)
is dual licensed under
the Creative Commons
Attribution-Non\-commercial-Share Alike 4.0 International License and
the Creative Commons
Attribution-Share Alike 4.0 International License.
To view a
copy of these licenses, visit
\url{https://creativecommons.org/licenses/by-nc-sa/4.0/}
or
\url{https://creativecommons.org/licenses/by-sa/4.0/}
or send a letter to
Creative Commons
PO Box 1866, Mountain View, CA 94042, USA\@.

\bigskip

\noindent
You can use, print, duplicate, share this book as much as you want.  You can
base your own notes on it and reuse parts if you keep the license the
same.  You can assume the license is either the CC-BY-NC-SA or CC-BY-SA\@,
whichever is compatible with what you wish to do, your derivative works must
use at least one of the licenses.
Derivative works must be prominently marked as such.

\bigskip

\noindent
During the writing of this book, 
the author was in part supported by NSF grants DMS-0900885 and
DMS-1362337.

\bigskip

\noindent
The date is the main identifier of version.  The major version / edition
number is raised only if there have been substantial changes.  Each
volume has its own version number.  Edition
number started at 4 for volume I, that is, version 4.0, as it was not kept track of
before.  %The edition given with the ISBN number is the major version.

\bigskip

\noindent
See \url{https://www.jirka.org/ra/} for more information
(including contact information, possible updates and errata).

\bigskip

\noindent
The \LaTeX\ source for the book is available
for possible modification and customization
at github: \url{https://github.com/jirilebl/ra}


% For large print do this
%\large

\microtypesetup{protrusion=false}
\tableofcontents
\microtypesetup{protrusion=true}

\newpage

%mbxENDIGNORE

%mbx <!--HERE IS WHERE WE ADD MBX PREAMBLE AFTER book-->
%mbx <!--HERE IS WHERE WE ADD MBX PREAMBLE 2-->

%%%%%%%%%%%%%%%%%%%%%%%%%%%%%%%%%%%%%%%%%%%%%%%%%%%%%%%%%%%%%%%%%%%%%%%%%%%%%%

%mbxSTARTIGNORE
\newcommand{\VolOneIntroExtrahead}{
\subsection{About Volume I}
}

\newcommand{\VolTwoIntro}{
\subsection{About Volume II}

\input frag-vol2-intro.tex
}
%mbxENDIGNORE


% Introduction to Volume 1
\input ch-vol1-intro.tex

%%%%%%%%%%%%%%%%%%%%%%%%%%%%%%%%%%%%%%%%%%%%%%%%%%%%%%%%%%%%%%%%%%%%%%%%%%%%%%

% Real Numbers chapter
\input ch-real-nums.tex

%%%%%%%%%%%%%%%%%%%%%%%%%%%%%%%%%%%%%%%%%%%%%%%%%%%%%%%%%%%%%%%%%%%%%%%%%%%%%%

% Sequences and Series chapter
\input ch-seq-ser.tex

%%%%%%%%%%%%%%%%%%%%%%%%%%%%%%%%%%%%%%%%%%%%%%%%%%%%%%%%%%%%%%%%%%%%%%%%%%%%%%

% Continuous Functions chapter
\input ch-contfunc.tex

%%%%%%%%%%%%%%%%%%%%%%%%%%%%%%%%%%%%%%%%%%%%%%%%%%%%%%%%%%%%%%%%%%%%%%%%%%%%%%

% The Derivative chapter
\input ch-der.tex

%%%%%%%%%%%%%%%%%%%%%%%%%%%%%%%%%%%%%%%%%%%%%%%%%%%%%%%%%%%%%%%%%%%%%%%%%%%%%%

% The Riemann Integral chapter
\input ch-riemann.tex

%%%%%%%%%%%%%%%%%%%%%%%%%%%%%%%%%%%%%%%%%%%%%%%%%%%%%%%%%%%%%%%%%%%%%%%%%%%%%%

% Sequences of Functions chapter
\input ch-seq-funcs.tex

%%%%%%%%%%%%%%%%%%%%%%%%%%%%%%%%%%%%%%%%%%%%%%%%%%%%%%%%%%%%%%%%%%%%%%%%%%%%%%

% Metric Spaces chapter
\input ch-metric.tex

%%%%%%%%%%%%%%%%%%%%%%%%%%%%%%%%%%%%%%%%%%%%%%%%%%%%%%%%%%%%%%%%%%%%%%%%%%%%%%

% Several Variables and Partial Derivatives chapter
\input ch-several-vars-ders.tex

%%%%%%%%%%%%%%%%%%%%%%%%%%%%%%%%%%%%%%%%%%%%%%%%%%%%%%%%%%%%%%%%%%%%%%%%%%%%%%

% One-dimensional Integrals in Several Variables chapter
\input ch-one-dim-ints-sv.tex

%%%%%%%%%%%%%%%%%%%%%%%%%%%%%%%%%%%%%%%%%%%%%%%%%%%%%%%%%%%%%%%%%%%%%%%%%%%%%%

% Multivariable Integral chapter
\input ch-multivar-int.tex

%%%%%%%%%%%%%%%%%%%%%%%%%%%%%%%%%%%%%%%%%%%%%%%%%%%%%%%%%%%%%%%%%%%%%%%%%%%%%%

% Approximating functions chapter
\input ch-approximate.tex

%%%%%%%%%%%%%%%%%%%%%%%%%%%%%%%%%%%%%%%%%%%%%%%%%%%%%%%%%%%%%%%%%%%%%%%%%%%%%%
%%%%%%%%%%%%%%%%%%%%%%%%%%%%%%%%%%%%%%%%%%%%%%%%%%%%%%%%%%%%%%%%%%%%%%%%%%%%%%
%%%%%%%%%%%%%%%%%%%%%%%%%%%%%%%%%%%%%%%%%%%%%%%%%%%%%%%%%%%%%%%%%%%%%%%%%%%%%%

%must be in separate "paragraph" (empty lines before and after)
% This closes the chapters/appedices above and starts actual backmatter
%mbxCLOSECHAPTER

%mbx <backmatter>

%mbxSTARTIGNORE
\cleardoublepage  
\phantomsection
\addcontentsline{toc}{chapter}{Further Reading}
\markboth{FURTHER READING}{FURTHER READING}
\begin{bibchapter}[Further Reading]
\begin{biblist}[\normalsize]

\bib{BS}{book}{
   author={Bartle, Robert G.},
   author={Sherbert, Donald R.},
   title={Introduction to Real Analysis},
   edition={3},
   publisher={John Wiley \& Sons Inc.},
   place={New York},
   date={2000},
}

\bib{DW}{book}{
   author={D'Angelo, John P.},
   author={West, Douglas B.},
   title={Mathematical Thinking: Problem-Solving and Proofs},
   edition={2},
   publisher={Prentice Hall},
   date={1999},
}

\bib{GIAM}{misc}{
   author={Fields, Joseph E.},
   title={A Gentle Introduction to the Art of Mathematics},
   note={Available at \url{http://giam.southernct.edu/GIAM/}},
}

\bib{Hammack}{misc}{
   author={Hammack, Richard},
   title={Book of Proof},
   note={Available at \url{http://www.people.vcu.edu/~rhammack/BookOfProof/}},
}

\bib{Rosenlicht}{book}{
   author={Rosenlicht, Maxwell},
   title={Introduction to Analysis},
   note={Reprint of the 1968 edition},
   publisher={Dover Publications Inc.},
   place={New York},
   date={1986},
   pages={viii+254},
   isbn={0-486-65038-3},
}

\bib{Rudin:baby}{book}{
   author={Rudin, Walter},
   title={Principles of Mathematical Analysis},
   edition={3},
   note={International Series in Pure and Applied Mathematics},
   publisher={McGraw-Hill Book Co.},
   place={New York},
   date={1976},
   pages={x+342},
}

\bib{Trench}{book}{
   author={Trench, William F.},
   title={Introduction to Real Analysis},
   year={2003},
   publisher={Pearson Education},
   note={\url{http://ramanujan.math.trinity.edu/wtrench/texts/TRENCH_REAL_ANALYSIS.PDF}},
}

\end{biblist}
\end{bibchapter}
%mbxENDIGNORE

%mbx <references xml:id="furtherreading_chapter">
%mbx   <title>Further Reading</title>
%mbx
%mbx   <biblio type="raw" xml:id="biblio-BS" tag="BS">Robert G. Bartle and Donald R. Sherbert,
%mbx     <title>Introduction to Real Analysis</title>,
%mbx     3rd ed., John Wiley &amp; Sons Inc., New York, 2000.</biblio>
%mbx
%mbx   <biblio type="raw" xml:id="biblio-DW" tag="DW">John P. D'Angelo and Douglas B. West,
%mbx     <title>Mathematical Thinking: Problem-Solving and Proofs</title>,
%mbx     2nd ed., Prentice Hall, 1999.</biblio>
%mbx
%mbx   <biblio type="raw" xml:id="biblio-GIAM" tag="F">Joseph E. Fields,
%mbx     <title>A Gentle Introduction to the Art of Mathematics</title>.
%mbx     Available at <url href="http://giam.southernct.edu/GIAM/">http://giam.southernct.edu/GIAM/</url>.</biblio>
%mbx
%mbx   <biblio type="raw" xml:id="biblio-Hammack" tag="H">Richard Hammack,
%mbx     <title>Book of Proof</title>.
%mbx     Available at
%mbx     <url href="http://www.people.vcu.edu/~rhammack/BookOfProof/">http://www.people.vcu.edu/~rhammack/BookOfProof/</url>.</biblio>
%mbx
%mbx   <biblio type="raw" xml:id="biblio-Rosenlicht" tag="R1">Maxwell Rosenlicht,
%mbx     <title>Introduction to Analysis</title>,
%mbx     Dover Publications Inc., New York, 1986, Reprint of the 1968 edition.</biblio>
%mbx
%mbx   <biblio type="raw" xml:id="biblio-Rudin_baby" tag="R2">Walter Rudin,
%mbx     <title>Principles of Mathematical Analysis</title>,
%mbx     3rd ed., McGraw-Hill Book Co., New York, 1976.</biblio>
%mbx
%mbx   <biblio type="raw" xml:id="biblio-Trench" tag="T">William F. Trench,
%mbx     <title>Introduction to Real Analysis</title>,
%mbx     Person Education.
%mbx     Available at
%mbx     <url href="http://ramanujan.math.trinity.edu/wtrench/texts/TRENCH_REAL_ANALYSIS.PDF">http://ramanujan.math.trinity.edu/wtrench/texts/TRENCH_REAL_ANALYSIS.PDF</url>.</biblio>
%mbx
%mbx </references>


%%%%%%%%%%%%%%%%%%%%%%%%%%%%%%%%%%%%%%%%%%%%%%%%%%%%%%%%%%%%%%%%%%%%%%%%%%%%%%
%%%%%%%%%%%%%%%%%%%%%%%%%%%%%%%%%%%%%%%%%%%%%%%%%%%%%%%%%%%%%%%%%%%%%%%%%%%%%%
%%%%%%%%%%%%%%%%%%%%%%%%%%%%%%%%%%%%%%%%%%%%%%%%%%%%%%%%%%%%%%%%%%%%%%%%%%%%%%

%mbxSTARTIGNORE
\cleardoublepage
\phantomsection
\addcontentsline{toc}{chapter}{\indexname}  
\microtypesetup{protrusion=false}
\printindex
\microtypesetup{protrusion=true}
%mbxENDIGNORE

%mbx   <index>
%mbx     <title>Index</title>
%mbx     <index-list />
%mbx   </index>

%%%%%%%%%%%%%%%%%%%%%%%%%%%%%%%%%%%%%%%%%%%%%%%%%%%%%%%%%%%%%%%%%%%%%%%%%%%%%%
%%%%%%%%%%%%%%%%%%%%%%%%%%%%%%%%%%%%%%%%%%%%%%%%%%%%%%%%%%%%%%%%%%%%%%%%%%%%%%
%%%%%%%%%%%%%%%%%%%%%%%%%%%%%%%%%%%%%%%%%%%%%%%%%%%%%%%%%%%%%%%%%%%%%%%%%%%%%%

%
% automake on glossaries doesn't work if the index is before the glossary.
% That's why the List of Notation is last, no other reason.  Problem is
% that printindex does a clearpage which screws up the delayed write18
% that glossaries sets up
%
%FIXME: MBX version of notation list?

%mbxSTARTIGNORE
\begingroup
\renewcommand{\pagelistname}{Page}
\setglossarystyle{long3colheader}
% correctly set up with cellspace
\renewenvironment{theglossary}%
  {\setlength\cellspacetoplimit{4pt}
   \setlength\cellspacebottomlimit{4pt}
   \setlength\LTleft{0pt}
   \setlength\LTright{0pt}
   \markboth{LIST OF NOTATION}{LIST OF NOTATION}
   \begin{longtable}{Sl @{\extracolsep{\fill}} Sl @{\extracolsep{\fill}} Sl}}%
  {\end{longtable}}%
\cleardoublepage
\microtypesetup{protrusion=false}
\printglossary[type=notation] 
\microtypesetup{protrusion=true}
\endgroup
%mbxENDIGNORE

%mbx </backmatter>

\end{document}

