\chapter*{Introduction}
\addcontentsline{toc}{chapter}{Introduction}
\markboth{INTRODUCTION}{INTRODUCTION}

%%%%%%%%%%%%%%%%%%%%%%%%%%%%%%%%%%%%%%%%%%%%%%%%%%%%%%%%%%%%%%%%%%%%%%%%%%%%%%

\section{About this book}

%mbxSTARTIGNORE
\VolOneIntroExtrahead
%mbxENDIGNORE
% pretext (mbx) is always the two volume set
%mbxlatex
%mbxlatex \subsection{About Volume I}
%mbxlatex 

This first volume is a one semester course in basic analysis.
Together with the second volume it is a year-long course.
It started its life
as my lecture notes for teaching Math 444 at the
University of Illinois at Urbana-Champaign (UIUC) in Fall semester 2009.
Later I added the metric space chapter to teach Math 521 at University of
Wisconsin--Madison (UW)\@.
Volume II was added to teach Math 4143/4153 at Oklahoma State University
(OSU)\@.
A prerequisite for these courses is usually a basic proof course,
using 
for example~\cite{Hammack}, \cite{GIAM}, or \cite{DW}.

It should be possible to use the book for
both a basic course for students who do not necessarily wish to
go to graduate school (such as UIUC 444), but also as a more advanced one-semester
course that also covers topics such as metric spaces (such as UW 521).
Here are my suggestions for what to cover in a semester course.  For a
slower course such as UIUC 444:
\begin{center}
\S0.3, \S1.1--\S1.4, \S2.1--\S2.5, \S3.1--\S3.4, \S4.1--\S4.2,
\S5.1--\S5.3, \S6.1--\S6.3
\end{center}
For a more rigorous course covering metric spaces that runs quite a bit faster
(such as UW 521):
\begin{center}
\S0.3, \S1.1--\S1.4, \S2.1--\S2.5, \S3.1--\S3.4, \S4.1--\S4.2,
\S5.1--\S5.3, \S6.1--\S6.2, \S7.1--\S7.6
\end{center}
It should also be possible to run a faster course without metric spaces
covering all sections of chapters 0 through 6.  The approximate number of
lectures given in the section notes through chapter 6 are a very rough
estimate and were designed for the slower course.
The first few chapters of the book can be used in an introductory proofs
course as is done, for example, at Iowa State University Math 201, where 
this book is used in conjunction with Hammack's Book of Proof~\cite{Hammack}.

With volume II one can run a year-long course that also covers multivariable
topics.  It may make sense in this case to cover most of the first volume
in the first semester while leaving metric spaces for the beginning of the
second semester.

The book normally used for the class at UIUC is Bartle and Sherbert,
\emph{Introduction to Real Analysis} third edition
\cite{BS}.
The structure of the beginning of the book somewhat follows the
standard syllabus of UIUC Math 444 and therefore has some similarities with
\cite{BS}.
A major difference is that we define the Riemann integral using
Darboux sums and not tagged partitions.  The Darboux approach is far more
appropriate for a course of this level.

Our approach allows us to fit a course such as UIUC 444 within a semester
and still spend some time on the interchange of limits and end with
Picard's theorem on the existence and uniqueness of solutions of ordinary
differential equations.
This theorem is a wonderful example
that uses many results proved in the book.  For more advanced students,
material may be covered faster so that we arrive at metric spaces and
prove Picard's theorem using the fixed point theorem as is usual.

Other excellent books exist.  My favorite is 
Rudin's excellent
\emph{Principles of Mathematical Analysis}~\cite{Rudin:baby}
or, as it is commonly and lovingly called, \emph{baby Rudin}
(to distinguish it from his other great analysis textbook,
\emph{big Rudin}).  I took a
lot of inspiration and ideas from Rudin.  However, Rudin is a bit more
advanced and ambitious than this present course.
For those that wish to continue
mathematics, Rudin is a fine investment.
An inexpensive and somewhat simpler alternative to Rudin is
Rosenlicht's \emph{Introduction to Analysis}~\cite{Rosenlicht}.
There is also the freely downloadable \emph{Introduction to Real
Analysis} by William Trench~\cite{Trench}.

\medskip

A note about the style of some of the proofs:  Many proofs
traditionally done by contradiction, I prefer to do by
a direct proof or by contrapositive.  While the book does include
proofs by contradiction, I only
do so when the contrapositive statement seemed too awkward, or when 
contradiction follows rather quickly.  In my opinion,
contradiction is more likely to get beginning students into trouble,
as we are talking about objects that do not exist.

I try to avoid unnecessary formalism where it is unhelpful.
Furthermore, the proofs and the language get slightly less formal as we
progress through the book, as more and more details are left out to avoid
clutter.

As a general rule, I use $:=$ instead of $=$ to define an
object rather than to simply show equality.  I use this symbol rather more
liberally than is usual for emphasis.
I use it even when the context is \myquote{local,}
that is, I may simply define a function $f(x) := x^2$
for a single exercise or example.

\medskip

Finally, I would like to acknowledge Jana Ma\v{r}\'ikov\'a,
Glen Pugh, Paul Vojta, Frank Beatrous, S\"{o}nmez \c{S}ahuto\u{g}lu,
Jim Brandt, Kenji Kozai, Arthur Busch,  Anton Petrunin,
Mark Meilstrup, Harold P.\ Boas, Atilla Y{\i}lmaz,
Thomas Mahoney, Scott Armstrong, and Paul Sacks,
Matthias Weber, Manuele Santoprete,
Robert Niemeyer, Amanullah Nabavi,
for teaching with the book and giving me lots of useful feedback.
Frank Beatrous wrote the University of Pittsburgh version extensions, which
served as inspiration for many more recent additions.
I would also like to thank
Dan Stoneham, Jeremy Sutter, Eliya Gwetta, Daniel Pimentel-Alarc\'on,
Steve Hoerning, Yi Zhang, Nicole Caviris,
Kristopher Lee, Baoyue Bi, Hannah Lund,
Trevor Mannella, Mitchel Meyer, Gregory Beauregard,
Chase Meadors, Andreas Giannopoulos, Nick Nelsen,
Ru Wang, Trevor Fancher, Brandon Tague, Wang KP,
an anonymous reader or two, and in general all the students in my classes for suggestions and
finding errors and typos.

% if in the two volume set then this will expand to the intro for vol II
%mbxSTARTIGNORE
\VolTwoIntro
%mbxENDIGNORE
% pretext (mbx) is always the two volume set
%mbxlatex
%mbxlatex \subsection{About Volume II}
%mbxlatex 
%mbxlatex \input frag-vol2-intro.tex
%mbxlatex 

%%%%%%%%%%%%%%%%%%%%%%%%%%%%%%%%%%%%%%%%%%%%%%%%%%%%%%%%%%%%%%%%%%%%%%%%%%%%%%

\sectionnewpage
\section{About analysis} \label{sec:aboutra}

Analysis is the branch of mathematics that deals with inequalities and
limits.  The present course deals with the most basic 
concepts in analysis.  The goal of the course is to acquaint the reader
with rigorous proofs in analysis and also to
set a firm foundation for calculus of one variable (and several variables if
volume II is also considered).

Calculus has prepared you, the student, for using mathematics without telling
you why what you learned is true.  To use, or teach, mathematics
effectively, you cannot simply know \emph{what} is true, you must know
\emph{why} it is true.  This course shows you \emph{why} calculus
is true.  It is here to give you a good understanding of the concept of a
limit, the derivative, and the integral.

Let us use an analogy.
An auto mechanic that has learned to change the oil, fix broken headlights,
and charge the battery, will only be able to do those simple tasks.  He
will be unable to work independently to diagnose and fix problems.
A high school teacher that does not understand the definition of the Riemann
integral or the derivative may not be able to properly answer all the
students' questions.
To this day I remember several nonsensical statements I heard
from my calculus teacher in high school, who simply did not understand
the concept of the limit, though he could \myquote{do} the problems in the
textbook.

\medskip

We start with a discussion of the real number system, most importantly
its completeness property, which is the basis for all that comes after.
We then discuss the simplest form of a limit,
the limit of a sequence.  Afterwards, we study
functions of one variable, continuity, and the derivative.
Next, we define the Riemann integral and prove the fundamental theorem of
calculus.  We discuss sequences of functions and the
interchange of limits.  Finally, we give an introduction to metric
spaces.

\medskip

Let us give the most important difference between analysis and algebra.
In algebra, we prove equalities directly;
we prove that an object, a number perhaps, is equal to another object.
In analysis, we usually prove inequalities,
and we prove those inequalities by estimating.
To illustrate the point, consider the following statement.

\medskip

\emph{Let $x$ be a real number.  If $x < \epsilon$ is true for all
real numbers
$\epsilon > 0$, then $x \leq 0$}.

\medskip

This statement is the general idea of what we do in analysis.
Suppose next we really wish to prove the equality
$x = 0$.  In analysis, we prove two inequalities:
$x \leq 0$ and $x \geq 0$.
To
prove the inequality
$x \leq 0$, we prove 
$x < \epsilon$ for all positive $\epsilon$.
To
prove the inequality
$x \geq 0$, we prove 
$x > -\epsilon$ for all positive $\epsilon$.

\medskip

The term \emph{real analysis} is a little bit of a misnomer.  I prefer to
use simply \emph{analysis}.  The other type of analysis, 
\emph{complex analysis}, really builds up on the present material, rather than
being distinct.  Furthermore, a more advanced course on real
analysis would talk about complex numbers often.
I suspect the nomenclature is
historical baggage.

\medskip

Let us get on with the show\ldots


%%%%%%%%%%%%%%%%%%%%%%%%%%%%%%%%%%%%%%%%%%%%%%%%%%%%%%%%%%%%%%%%%%%%%%%%%%%%%%

\sectionnewpage
\section{Basic set theory} \label{sec:basicset}
\index{set theory}

\sectionnotes{1--3 lectures (some material can be skipped, covered lightly,
or left as reading)}

Before we start talking about analysis, we need to fix some language.
Modern\footnote{The term \myquote{modern} refers to late 19th century up to
the present.}
analysis uses the language of sets, and therefore that is where we start.
We talk about sets in a rather informal way, using the so-called
\myquote{\myindex{na\"ive set theory}.}  Do not worry, that is what majority of
mathematicians use, and it is hard to get into trouble.
The reader has hopefully seen the very basics of set theory
and proof writing before, and this section should be a quick refresher.

\subsection{Sets}

\begin{defn}
A \emph{\myindex{set}} is a collection of objects called
\emph{elements}\index{element} or \emph{members}\index{member}.  A set with
no objects is called the \emph{\myindex{empty set}} and is denoted by
$\emptyset$\glsadd{not:emptyset} (or sometimes by $\{ \}$).
\end{defn}

Think of a set as a club with a certain membership.  For
example, the students who play chess are members of the chess club.
The same student can be a member of many different clubs.
However,
do not take the analogy too far.  A set is only defined by the members
that form the set; two sets that have the same members are the same set.

Most of the time we will consider sets
of numbers.  For example, the set
\glsadd{not:set}
\begin{equation*}
S := \{ 0, 1, 2 \}
\end{equation*}
is the set containing
the three elements 0, 1, and 2.
By \myquote{$:=$},\glsadd{not:def} we mean we are defining what $S$ is, rather than
just showing equality.
We write\glsadd{not:eltof}
\begin{equation*}
1 \in S
\end{equation*}
to denote that the number 1 belongs to the set $S$.  That is, 1 is a member
of $S$.
At times we want to say that two elements are in a set $S$, so we 
write \myquote{$1,2 \in S$} as a shorthand for
\myquote{$1 \in S$ and $2 \in S$.}

Similarly, we write\glsadd{not:noteltof}
\begin{equation*}
7 \notin S
\end{equation*}
to denote that the number 7 is not in $S$.  That is, 7 is not a member of
$S$.

The elements of all sets under consideration come from some set we call the
\emph{\myindex{universe}}.  For simplicity,
we often consider the universe to be the set that contains only the elements
we are interested in.
The universe is generally understood from context
and is not explicitly mentioned.  In this course, our universe will
most often be the set of real numbers.

While the elements of a set are often numbers,
other objects, such as other sets, can be elements of a set.
A set may also contain some of the same elements as another set.  For example,
\begin{equation*}
T := \{ 0, 2 \}
\end{equation*}
contains the numbers 0 and 2.  In this case all elements of $T$ also
belong to $S$.  We write $T \subset S$.\glsadd{not:subset}
See \figureref{fig:sets} for a diagram.
%More formally we make the
%following definition.

\begin{myfigureht}
\subimport*{figures/}{sets.pdf_t}
\caption{A diagram of the example sets $S$ and its subset $T$.\label{fig:sets}}
\end{myfigureht}

\begin{defn}
%FIXMEevillayouthack
\pagebreak[2]
\leavevmode
\begin{enumerate}[(i)]
\item
A set $A$ is a \emph{\myindex{subset}}
of a set $B$ if $x \in A$ implies $x \in B$, and we write $A \subset B$.
That is, all members of $A$ are also members of $B$.  At times we
write $B \supset A$ to mean the same thing.
\item
Two sets $A$ and $B$ are \emph{\myindex{equal}}\glsadd{not:equal} if $A \subset B$ and $B
\subset A$.  We write $A = B$.
That is, $A$ and $B$ contain exactly the same elements.
If it is not true that $A$ and $B$ are equal, then 
we write $A \not= B$.
\item
A set $A$ is a \emph{\myindex{proper subset}} of $B$ if $A \subset B$
and $A \not= B$.  We write $A \subsetneq B$.\glsadd{not:subsetneq}
\end{enumerate}
\end{defn}

For the example $S$ and $T$ defined above, $T \subset S$, but
$T \not= S$.  So $T$ is a proper subset of $S$.
If $A = B$, then $A$ and $B$ are simply two names for the
same exact set.

To define sets, one often uses
the
\emph{\myindex{set building notation}},\glsadd{not:setbuilder}
\begin{equation*}
\bigl\{ x \in A : P(x) \bigr\} .
\end{equation*}
This notation refers to a subset of the set $A$ containing all elements
of $A$ that satisfy the property $P(x)$.
Using $S = \{ 0, 1, 2 \}$ as above, $\{ x \in S : x \not= 2 \}$
is the set $\{ 0, 1 \}$.
The notation is sometimes
abbreviated as $\bigl\{ x : P(x) \bigr\}$, that is, $A$ is not mentioned when understood from context.
Furthermore, $x \in A$ is sometimes replaced with a formula to make the notation
easier to read.  

\begin{example}
The following are sets including the standard notations.
\begin{enumerate}[(i)]
\item The set of \emph{\myindex{natural numbers}},
\glsadd{not:natnums}
$\N := \{ 1, 2, 3, \ldots
\}$.
\item The set of \emph{\myindex{integers}},
\glsadd{not:ints} 
$\Z := \{ 0, -1, 1, -2, 2, \ldots
\}$.
\item The set of \emph{\myindex{rational numbers}},
\glsadd{not:rationals} 
$\Q := \bigl\{ \frac{m}{n} :  m, n \in \Z \text{ and } n \not= 0 \bigr\}$.
\item The set of even natural numbers, $\{  2m : m \in \N \}$.
\item The set of real numbers,
\glsadd{not:reals} 
$\R$.
\end{enumerate}

Note that $\N \subset \Z \subset \Q \subset \R$.
\end{example}

We create new sets out of old ones by applying some natural operations.

\begin{defn}
\pagebreak[2]
\leavevmode
\begin{enumerate}[(i)]
\item
A \emph{\myindex{union}} of two sets $A$ and $B$ is defined as
\glsadd{not:union}
\begin{equation*}
A \cup B := \{ x : x \in A \text{ or } x \in B \} .
\end{equation*}
\item
An \emph{\myindex{intersection}} of two sets $A$ and $B$ is defined as
\glsadd{not:intersection}
\begin{equation*}
A \cap B := \{ x : x \in A \text{ and } x \in B \} .
\end{equation*}
\item
A \emph{complement of $B$ relative to $A$\index{complement relative to}}
(or \emph{\myindex{set-theoretic difference}} of $A$ and $B$) is defined as
\glsadd{not:setminus}
\begin{equation*}
A \setminus B := \{ x : x \in A \text{ and } x \notin B \} .
\end{equation*}
\item
We say
\emph{\myindex{complement}} of $B$ and write $B^c$ \glsadd{not:complement}
instead of $A \setminus B$ if
the set $A$ is either the entire universe or if it is the obvious
set containing $B$, and is understood from context.
\item
We say sets $A$ and $B$ are \emph{\myindex{disjoint}} if $A \cap B =
\emptyset$.
\end{enumerate}
\end{defn}

The notation $B^c$ may be a little vague at this point.  If
the set $B$ is a subset of the real numbers $\R$, then $B^c$ means
$\R \setminus B$.  If $B$ is naturally a subset of the natural numbers,
then $B^c$
is $\N \setminus B$.  If ambiguity can arise, we 
use the set difference notation $A \setminus B$.

\begin{myfigureht}
\subimport*{figures/}{figsetop.pdf_t}
\caption{Venn diagrams of set operations, the result of the operation is
shaded.\label{figsetop}}
\end{myfigureht}
We illustrate the operations on the
\emph{Venn diagrams}\index{Venn diagram} in \figureref{figsetop}.
Let us now establish one of most basic theorems about sets and logic.

\begin{thm}[DeMorgan]\index{DeMorgan's theorem}
Let $A, B, C$ be sets.  Then
\begin{align*}
{(B \cup C)}^c &= B^c \cap C^c , \\
{(B \cap C)}^c &= B^c \cup C^c ,
\end{align*}
or, more generally,
\begin{align*}
A \setminus (B \cup C) &= (A \setminus B) \cap (A \setminus C) , \\
A \setminus (B \cap C) &= (A \setminus B) \cup (A \setminus C) .
\end{align*}
\end{thm}

\begin{proof}
The first statement is proved by the second statement if we
assume the set $A$ is our \myquote{universe.}

Let us prove $A \setminus (B \cup C) = (A \setminus B) \cap (A \setminus C)$.
Remember the definition of equality of sets.  First, we must show that
if $x \in A \setminus (B \cup C)$, then
$x \in (A \setminus B) \cap (A \setminus C)$.  Second, we must also show that
if $x \in (A \setminus B) \cap (A \setminus C)$, then
$x \in A \setminus (B \cup C)$.

So let us assume $x \in A \setminus (B \cup C)$.  Then $x$ is in 
$A$, but not in $B$ nor $C$.  Hence $x$ is in $A$ and not in $B$, that is,
$x \in A \setminus B$.  Similarly $x \in A \setminus C$.  Thus
$x \in (A \setminus B) \cap (A \setminus C)$.

On the other hand suppose 
$x \in (A \setminus B) \cap (A \setminus C)$.  In particular,
$x \in (A \setminus B)$, so 
$x \in A$ and $x \notin B$.  Also as $x \in (A \setminus C)$, then $x \notin C$.
Hence $x \in A \setminus (B \cup C)$.

The proof of the other equality is left as an exercise.
\end{proof}

The result above we called a \emph{Theorem}, while most results we call
a \emph{Proposition}, and a few we call a \emph{Lemma} (a result leading to another result) or
\emph{Corollary} (a quick consequence of the preceding result).
Do not read too much into
the naming.  Some of it is traditional, some of it is stylistic choice.
It is not necessarily true that a \emph{Theorem} is always
\myquote{more important} than a
\emph{Proposition} or a \emph{Lemma}.

We will also need to intersect or union several sets at once.  If there are
only finitely many, then we simply apply the union or intersection operation
several times.  However, suppose we have an infinite collection
of sets (a set of sets)
$\{ A_1, A_2, A_3, \ldots \}$.  We define
\glsadd{not:infunion}
\glsadd{not:infintersection}
\begin{align*}
& \bigcup_{n=1}^\infty A_n := \{ x : x \in A_n \text{ for some } n \in \N
\} , \\
& \bigcap_{n=1}^\infty A_n := \{ x : x \in A_n \text{ for all } n \in \N
\} .
\end{align*}

We can also have sets indexed by two natural numbers.  For example, we can have
the set of sets
$\{ A_{1,1}, A_{1,2}, A_{2,1}, A_{1,3}, A_{2,2}, A_{3,1}, \ldots \}$.  Then
we write 
\begin{equation*}
\bigcup_{n=1}^\infty \bigcup_{m=1}^\infty A_{n,m}
=
\bigcup_{n=1}^\infty \left( \bigcup_{m=1}^\infty A_{n,m} \right) .
\end{equation*}
And similarly with intersections.

It is not hard to see that we can take the unions in any order.  However,
switching the order of unions and intersections is not generally permitted without proof.
For instance,
\begin{equation*}
\bigcup_{n=1}^\infty
\bigcap_{m=1}^\infty
\{ k \in \N : mk < n \}
=
\bigcup_{n=1}^\infty \emptyset = \emptyset .
\end{equation*}
However,
\begin{equation*}
\bigcap_{m=1}^\infty
\bigcup_{n=1}^\infty
\{ k \in \N : mk < n \}
=
\bigcap_{m=1}^\infty
\N
=
\N.
\end{equation*}

Sometimes, the index set is not the natural numbers.  In such a case we
require a more general notation.  Suppose $I$ is some set and for each $\lambda \in
I$, there is a set $A_\lambda$.  Then we define
\glsadd{not:arbitunion}
\glsadd{not:arbitintersection}
\begin{equation*}
\bigcup_{\lambda \in I} A_\lambda := \{ x : x \in A_\lambda \text{ for some }
\lambda \in I
\} ,
\qquad
\bigcap_{\lambda \in I} A_\lambda := \{ x : x \in A_\lambda \text{ for all }
\lambda \in I
\} .
\end{equation*}

\subsection{Induction}

When a statement includes an arbitrary natural number,
a common method of proof is the principle of induction.  
We start with the set of natural numbers $\N = \{ 1,2,3,\ldots \}$, and we
give them their natural ordering, 
that is, $1 < 2 < 3 < 4 < \cdots$.
By $S \subset \N$ having a \emph{\myindex{least element}}, we mean that
there exists an $x \in S$,
such that for every
$y \in S$, we have $x \leq y$.

The natural numbers $\N$ ordered in the natural way
possess the so-called \emph{\myindex{well ordering property}}.
We take this property
as an \myindex{axiom}; we simply assume it is true.

%mbxSTARTIGNORE
\theoremstyle{plain}
\newtheorem*{wellordprop}{Well ordering property of $\N$}
\hypertarget{wop:link}{}%
\begin{wellordprop}
Every nonempty subset of $\N$ has a least (smallest) element.
\end{wellordprop}
%mbxENDIGNORE
%mbx <axiom xml:id="wop_link" number="">
%mbx   <title>Well ordering property of <m>\N</m></title>
%mbx   <statement><p>
%mbx   Every nonempty subset of <m>\N</m> has a least (smallest) element
%mbx   </p></statement>
%mbx </axiom>

The \emph{principle of \myindex{induction}}\index{principle of induction} is
the following theorem, which is in a sense\footnote{To be completely
rigorous, this equivalence is only true if we
also assume as an axiom that $n-1$ exists for all natural numbers bigger than $1$, which we
do.  In this book, we are assuming all the usual arithmetic holds.}
equivalent to the well ordering property of the natural numbers.

\begin{thm}[Principle of induction] \label{induction:thm}
Let $P(n)$ be a statement depending on a natural number~$n$.  Suppose that
\begin{enumerate}[(i)]
\item \emph{(\myindex{basis statement})} $P(1)$ is true.
\item \emph{(\myindex{induction step})} If $P(n)$ is true, then $P(n+1)$ is true.
\end{enumerate}
Then $P(n)$ is true for all $n \in \N$.
\end{thm}

\begin{proof}
Let $S$ be the set of natural numbers $n$ for which $P(n)$ is
not true.  Suppose for contradiction that
$S$ is nonempty.  Then $S$ has a least element by the well ordering
property.  Call $m \in S$ the least element of $S$.  We know $1 \notin
S$ by hypothesis.  So $m > 1$, and $m-1$ is a natural number as well.
Since $m$ is the least element of $S$, we know that $P(m-1)$ is true.
But the induction step says that $P(m-1+1) = P(m)$ is true, 
contradicting the statement that $m \in S$.  Therefore, $S$ is empty and 
$P(n)$ is true for all $n \in \N$.
\end{proof}

Sometimes it is convenient to start at a different number than 1,
all that changes is the labeling.  The assumption that
$P(n)$ is true in \myquote{if $P(n)$ is true,
then $P(n+1)$ is true}
is usually called the \emph{\myindex{induction hypothesis}}.

\begin{example}
Let us prove that for all $n \in \N$,
\begin{equation*}
2^{n-1} \leq n! \qquad \text{(recall } n! = 1 \cdot 2 \cdot 3 \cdots n\text{)}.
\end{equation*}
We let $P(n)$ be the statement that
$2^{n-1} \leq n!$ is true.
Plug in $n=1$ to see that $P(1)$ is true.

Suppose $P(n)$ is true.  That is, suppose 
$2^{n-1} \leq n!$ holds.  Multiply both sides by 2 to obtain
\begin{equation*}
2^n \leq 2(n!) .
\end{equation*}
As $2 \leq (n+1)$ when $n \in \N$, we have
$2(n!) \leq (n+1)(n!) = (n+1)!$.  That is,
\begin{equation*}
2^n \leq 2(n!) \leq  (n+1)!,
\end{equation*}
and hence $P(n+1)$ is true.  By the principle of induction,
$P(n)$
is true for all $n \in \N$.  In other words,
$2^{n-1} \leq n!$ is true for all $n \in \N$.
\end{example}

\begin{example} \label{example:geometricsum}
We claim that for all $c \not= 1$,
\begin{equation*}
1 + c + c^2 + \cdots + c^n = \frac{1-c^{n+1}}{1-c} .
\end{equation*}

Proof: It is easy to check that the equation holds with $n=1$.  Suppose 
it is true for $n$.  Then
\begin{equation*}
\begin{split}
1 + c + c^2 + \cdots + c^n + c^{n+1} & =
( 1 + c + c^2 + \cdots + c^n ) + c^{n+1} \\
& = \frac{1-c^{n+1}}{1-c}  + c^{n+1} \\
& = \frac{1-c^{n+1}  + (1-c)c^{n+1}}{1-c} \\
& = \frac{1-c^{n+2}}{1-c} .
\end{split}
\end{equation*}
\end{example}

Sometimes, it is easier to use in the inductive step
that $P(k)$ is true for all $k = 1,2,\ldots,n$, not just for $k=n$.
This principle is called \emph{\myindex{strong induction}} and is equivalent
to the normal induction above.  The proof of that
equivalence is left as an exercise.

\begin{thm}[Principle of strong induction]
\index{principle of strong induction}\index{strong induction}
Let $P(n)$ be a statement depending on a natural number $n$.  Suppose that
\begin{enumerate}[(i)]
\item \emph{(\myindex{basis statement})} $P(1)$ is true.
\item \emph{(\myindex{induction step})} If $P(k)$ is true for all $k = 1,2,\ldots,n$, then $P(n+1)$ is true.
\end{enumerate}
Then $P(n)$ is true for all $n \in \N$.
\end{thm}

\subsection{Functions}

Informally,
a \emph{\myindex{set-theoretic function}}\index{function}
$f$ taking a set $A$ to a set $B$
is a mapping that to each $x \in A$ assigns a unique $y \in B$.  We write
$f \colon A \to B$.\glsadd{not:func}  An example function 
$f \colon S \to T$ taking $S := \{ 0, 1, 2 \}$ to $T := \{ 0, 2 \}$
can be defined
by assigning $f(0) := 2$, $f(1) := 2$, and $f(2) := 0$.  That is, a function $f
\colon A \to B$ is
a black box, into which we stick an element of $A$ and the function
spits out an element of $B$.
Sometimes $f$ is called a \emph{\myindex{mapping}} or a
\emph{\myindex{map}},
and we say $f$ \emph{maps $A$ to $B$}.


Often, functions are defined by some sort of
formula; however, you should really think of a function as just a very big
table of values.
The subtle issue here is that a single function can have several 
formulas, all giving the same function.  Also, for many functions, there is
no formula that expresses its values.

To define a function rigorously, first let us define the Cartesian product.

\begin{defn}
Let $A$ and $B$ be sets.  The \emph{\myindex{Cartesian product}}
is the set of tuples defined as
\glsadd{not:cartprod}
\begin{equation*}
A \times B :=
\bigl\{ (x,y) : x \in A, y \in B \bigr\} .
\end{equation*}
\end{defn}

For instance, $\{ a,b \} \times \{ c , d\} =
\bigl\{
(a,c), (a,d), (b,c), (b,d)
\bigr\}$.
A more complicated example is the set $[0,1] \times [0,1]$: a subset of the
plane bounded by a square with vertices $(0,0)$, $(0,1)$, $(1,0)$, and $(1,1)$.
When $A$ and $B$ are the same set we sometimes use a superscript 2 to denote
such a product.  For example, $[0,1]^2 = 
[0,1] \times [0,1]$ or $\R^2 = \R \times \R$ (the Cartesian plane).

\begin{defn}
A \emph{function} $f \colon A \to B$ is a subset $f$ of $A \times B$
such that for each $x \in A$, there exists a unique $y \in B$ for which $(x,y) \in f$.
We write $f(x) = y$.  Sometimes
the set $f$ is called the \emph{\myindex{graph}} of the function rather than
the function itself.

The set $A$ is called the \emph{\myindex{domain}} of $f$ (and
sometimes confusingly denoted $D(f)$).  The set
\begin{equation*}
R(f) := \{ y \in B : \text{there exists an } x \in A \text{ such that }
f(x)=y \}
\end{equation*}
is called the \emph{\myindex{range}} of $f$.
The set $B$ is called the \emph{\myindex{codomain}} of $f$.
\end{defn}

It is possible that the range $R(f)$ is a proper subset of the codomain $B$,
while the domain of $f$ is always equal to $A$.  We generally 
assume that the domain of $f$ is nonempty.

\begin{example}
From calculus, you are most familiar with functions taking real numbers to real
numbers.  However, you saw some other types of functions as well.
The derivative is a function mapping the set of
differentiable functions to the set of all functions.
Another example is the Laplace transform, which also
takes functions to functions.  Yet another example is the function that takes
a continuous function $g$ defined on the interval $[0,1]$ and returns the
number $\int_0^1 g(x) \,dx$.
\end{example}

\begin{defn}
Consider a function $f \colon A \to B$.  Define
the \emph{\myindex{image}} (or \emph{\myindex{direct image}}) of a subset
$C \subset A$ as
\glsadd{not:dirimage}
\begin{equation*}
f(C) := \bigl\{ f(x) \in B : x \in C \bigr\} .
\end{equation*}
Define the \emph{\myindex{inverse image}} of a subset $D
\subset B$ as
\glsadd{not:invimage}
\begin{equation*}
f^{-1}(D) := \bigl\{ x \in A : f(x) \in D \bigr\} .
\end{equation*}
\end{defn}

In particular, $R(f) = f(A)$, the range is the direct image of
the domain $A$.

\begin{myfigureht}
\subimport*{figures/}{funcimags_full.pdf_t}
\caption{Example of direct and inverse images for the function
$f \colon \{ 1,2,3,4 \} \to \{ a,b,c,d \}$ defined by
$f(1) := b$,
$f(2) := d$,
$f(3) := c$,
$f(4) := b$.\label{figfuncimags}}
\end{myfigureht}

\begin{example}
Define the function $f \colon \R \to \R$ by
$f(x) := \sin(\pi x)$.  Then $f\bigl([0,\nicefrac{1}{2}]\bigr) = [0,1]$, 
$f^{-1}\bigl(\{0\}\bigr) = \Z$, etc.
\end{example}

\begin{prop} \label{st:propinv}
Consider $f \colon A \to B$.  Let $C, D$ be subsets of $B$.  Then
\begin{align*}
& f^{-1}( C \cup D) = f^{-1} (C) \cup f^{-1} (D) , \\
& f^{-1}( C \cap D) = f^{-1} (C) \cap f^{-1} (D) , \\
& f^{-1}( C^c) = {\bigl( f^{-1} (C) \bigr)}^c .
\end{align*}
\end{prop}

Read the last line of the proposition as
$f^{-1}( B \setminus C) = A \setminus f^{-1} (C)$.

\begin{proof}
We start with the union.  If $x \in 
f^{-1}( C \cup D)$, then 
$x$ is taken to $C$ or $D$, that is, $f(x) \in C$ or $f(x) \in D$.  Thus
$f^{-1}( C \cup D) \subset f^{-1} (C) \cup f^{-1} (D)$.  Conversely
if $x \in f^{-1}(C)$, then $x \in f^{-1}(C \cup D)$.  Similarly for
$x \in f^{-1}(D)$.  Hence
$f^{-1}( C \cup D) \supset f^{-1} (C) \cup f^{-1} (D)$, and we have
equality.

The rest of the proof is left as an exercise.
\end{proof}

The proposition does not hold for direct images.  We do have
the following weaker result.

\begin{prop} \label{st:propfor}
Consider $f \colon A \to B$.  Let $C, D$ be subsets of $A$.  Then
\begin{align*}
& f( C \cup D) = f (C) \cup f (D) , \\
& f( C \cap D) \subset f (C) \cap f (D) .
\end{align*}
\end{prop}

The proof is left as an exercise.

\begin{defn}
Let $f \colon A \to B$ be a function.
The function $f$ is said to be
\emph{\myindex{injective}} or
\emph{\myindex{one-to-one}} if $f(x_1) = f(x_2)$ implies $x_1 = x_2$.  In
other words,
$f$ is injective if
for all $y \in B$, the set
$f^{-1}(\{y\})$ is empty or consists of a single element.
We call such an $f$ an \emph{\myindex{injection}}.

If $f(A) = B$, then we say $f$ is
\emph{\myindex{surjective}} or
\emph{\myindex{onto}}.
In other words, $f$ is surjective if the range and the codomain of $f$ are equal.
We call such an $f$ a \emph{\myindex{surjection}}.

If $f$ is both surjective and injective, then
we say $f$ is \emph{\myindex{bijective}} or that $f$ is a
\emph{\myindex{bijection}}.
\end{defn}

When $f \colon A \to B$ is a bijection, then the inverse image of a single
element, $f^{-1}(\{y\})$, is always
a unique element of $A$.  We then consider $f^{-1}$ as a function
\glsadd{not:invfunction}
$f^{-1} \colon B \to A$ and we write simply $f^{-1}(y)$.
In this case, we call $f^{-1}$ the \emph{\myindex{inverse function}} of $f$.
For instance, for the bijection $f \colon \R \to \R$ defined by $f(x) :=
x^3$, we have
$f^{-1}(x) = \sqrt[3]{x}$.

%A final piece of notation for functions that
%we need is the \emph{\myindex{composition of functions}}.

\begin{defn}
Consider $f \colon A \to B$ and $g \colon B \to C$.  The
\emph{composition}\index{composition of functions}
of the functions $f$ and $g$ is the function
$g \circ f \colon A \to C$ defined as
\glsadd{not:composition}
\begin{equation*}
(g \circ f)(x) := g\bigl(f(x)\bigr) .
\end{equation*}
\end{defn}

For example, if $f \colon \R \to \R$ is $f(x):=x^3$ and $g \colon \R \to \R$
is $g(y) = \sin(y)$, then $(g \circ f)(x) = \sin(x^3)$.

\subsection{Relations and equivalence classes}

We often compare two objects in some way.  We say $1 < 2$
for natural numbers, or $\nicefrac{1}{2} = \nicefrac{2}{4}$ for rational
numbers, or $\{ a,c \} \subset \{ a,b,c \}$ for sets.  The `$<$', `$=$', and
`$\subset$' are examples of
relations.

\begin{defn}
Given a set $A$, a \emph{\myindex{binary relation}}\index{relation} on $A$
is a subset $\sR \subset A \times A$,
which are those pairs where the relation is said to hold.
Instead of $(a,b) \in \sR$, we write
$a \, \sR \, b$.
\end{defn}

\begin{example}
Take $A := \{ 1,2,3 \}$.

Consider the relation `$<$'.  The corresponding set 
of pairs is $\bigl\{ (1,2), (1,3), (2,3) \bigr\}$.  So $1 < 2$ holds as $(1,2)$
is in the corresponding set of pairs, but $3 < 1$ does not hold as $(3,1)$ is
not in the set.

Similarly, the relation `$=$'
is defined by the set of pairs $\bigl\{ (1,1), (2,2), (3,3) \big\}$.

Any subset of $A \times A$ is a relation.  Let us define the relation
$\dagger$ via $\bigl\{ (1,2), (2,1), (2,3), (3,1) \bigr\}$, then $1 \dagger 2$ and
$3 \dagger 1$ are
true, but $1 \dagger 3$ is not.
\end{example}

\begin{defn}
%FIXMEevillayouthack
\pagebreak[2]
Let $\sR$ be a relation on a set $A$.  Then $\sR$ is said to be
\begin{enumerate}[(i)]
\item
\emph{Reflexive}\index{reflexive relation} if $a\, \sR \, a$ for
all $a \in A$.
\item
\emph{Symmetric}\index{symmetric relation} if $a\, \sR \, b$ implies
$b \, \sR \, a$.
\item
\emph{Transitive}\index{transitive relation} if $a\, \sR \, b$ and
$b \, \sR \, c$ implies $a\, \sR \, c$.
\end{enumerate}
If $\sR$ is reflexive, symmetric, and transitive, then it is said to be
an \emph{\myindex{equivalence relation}}.
\end{defn}

\begin{example}
Let $A := \{ 1,2,3 \}$.
The relation `$<$' is transitive, but neither reflexive nor symmetric.  The
relation `$\leq$' defined by
$\bigl\{ (1,1), (1,2), (1,3), (2,2), (2,3), (3,3) \bigr\}$
is reflexive and transitive, but not symmetric.
Finally, a relation `$\star$' defined by
$\bigl\{ (1,1), (1,2), \allowbreak (2,1), \allowbreak (2,2), \allowbreak (3,3) \bigr\}$ is
an equivalence relation.
\end{example}

Equivalence relations are useful in that they divide a set into sets of
\myquote{equivalent} elements.

\begin{defn}
Let $A$ be a set and $\sR$ an equivalence relation.
An \emph{\myindex{equivalence class}} of $a \in A$, often denoted
by $[a]$,\glsadd{not:equivclass} is the set $\{ x \in A : a\, \sR \, x \}$.
\end{defn}

For example, given the relation `$\star$' above, there are two equivalence classes,
$[1] = [2] = \{ 1,2 \}$ and $[3] = \{ 3 \}$.

Reflexivity guarantees that $a \in [a]$.  Symmetry guarantees
that if $b \in [a]$, then $a \in [b]$.  Finally, transitivity guarantees that
if $b \in [a]$ and $c \in [b]$, then $c \in [a]$.
In particular, we have the following proposition, whose proof is an
exercise.

\begin{prop} \label{prop:equivclasses}
If $\sR$ is an equivalence relation on a set $A$,
then every $a \in A$ is in exactly one
equivalence class.  Moreover, $a\, \sR \, b$ if and only if $[a] = [b]$.
\end{prop}

\begin{example} \label{example:ratnums}
The set of rational numbers can be defined as equivalence classes of
a pair of an integer and a natural number,
that is elements of $\Z \times \N$.  The relation
is defined by $(a,b) \sim (c,d)$ whenever $ad = bc$.
It is left as an exercise to prove that `$\sim$' is an equivalence relation.
Usually the equivalence class $\bigl[(a,b)\bigr]$ is written as $\nicefrac{a}{b}$.
\end{example}

\subsection{Cardinality}

A subtle issue in set theory and one generating a considerable amount of
confusion among students is that of cardinality, or \myquote{size} of sets.  The
concept of cardinality is important in modern mathematics in general and
in analysis in particular.  In this section, we will see the first really
unexpected theorem.

\begin{defn}
Let $A$ and $B$ be sets.  We say $A$ and $B$ have the same
\emph{\myindex{cardinality}}
when there exists a bijection $f \colon A \to B$.  We denote
by $\abs{A}$ \glsadd{not:cardinality} the equivalence class of all sets with the same cardinality as
$A$ and we simply call $\abs{A}$ the cardinality of $A$.
\end{defn}

For example, $\{ 1,2,3 \}$ has the same cardinality as $\{ a,b,c \}$ by
defining a bijection $f(1) := a$, $f(2) := b$, $f(3) := c$.  Clearly the
bijection is not unique.

The existence of a bijection really is an equivalence relation.
The identity, $f(x) := x$, is a bijection showing reflexivity.
If $f$ is a bijection, then so is $f^{-1}$ showing symmetricity.
If $f \colon A \to B$ and $g \colon B \to C$ are bijections, then
$g \circ f$ is a bijection of $A$ and $C$ showing transitivity.
A set $A$ has the same cardinality as the empty set if and only
if $A$ itself is the empty set:  If $B$ is nonempty, then
no function $f \colon B \to \emptyset$ can exist.
In particular, there is no bijection of $B$ and $\emptyset$.

\begin{defn}
Suppose $A$ has the same cardinality as $\{ 1,2,3,\ldots,n \}$
for some $n \in \N$.
We then write $\abs{A} := n$.  If $A$ is empty, we write $\abs{A} := 0$.
In either case, we say that $A$ is
\emph{\myindex{finite}}.

We say $A$ is \emph{\myindex{infinite}} or \myquote{of infinite cardinality}
if $A$ is not finite.
\end{defn}

That the notation $\abs{A} = n$ is justified we leave as an exercise.  That
is, for each nonempty finite set $A$, there exists a unique natural number
$n$ such that there exists a bijection from $A$ to $\{ 1,2,3,\ldots,n \}$.

We can order sets by size.

\begin{defn} \label{def:comparecards}
We write
\begin{equation*}
\abs{A} \leq \abs{B}
\end{equation*}
if there exists an injection from $A$ to $B$.  We write $\abs{A} = \abs{B}$
if $A$ and $B$ have the same cardinality.  We write $\abs{A} < \abs{B}$
if $\abs{A} \leq \abs{B}$, but $A$ and $B$ do not have the same cardinality.
\end{defn}

We state without proof that
$A$ and $B$ have the same cardinality if and only if
$\abs{A} \leq \abs{B}$ and
$\abs{B} \leq \abs{A}$.  This is the so-called
\myindex{Cantor--Bernstein--Schr\"oder} theorem.
Furthermore, if $A$ and $B$ are any two sets,
we can always write $\abs{A} \leq \abs{B}$ or
$\abs{B} \leq \abs{A}$.  The issues surrounding this
last statement are very subtle.  As we do not require either
of these two statements, we omit proofs.

The truly interesting cases of cardinality are infinite sets.
We will distinguish two types of infinite cardinality.

\begin{defn}
If $\abs{A} = \abs{\N}$, then $A$ is said to be
\emph{\myindex{countably infinite}}.
If $A$ is finite or countably infinite, then we say $A$
is \emph{\myindex{countable}}.
If $A$ is not countable, then
$A$ is said to be \emph{\myindex{uncountable}}.
\end{defn}

The cardinality of $\N$ is usually denoted as
$\aleph_0$ (read as aleph-naught)\footnote{For the fans of the TV show
\emph{Futurama}, there is a movie theater in one episode
called an $\aleph_0$-plex.}.

\begin{example}
The set of even natural numbers has the same cardinality as $\N$.  Proof:
Let $E \subset \N$ be the set of even natural numbers.
Given $k \in E$, write $k=2n$ for some $n \in \N$.
Then $f(n) := 2n$ defines a bijection $f \colon \N \to E$.
\end{example}

In fact, let us mention without proof the following characterization
of infinite sets: \emph{A set is infinite if and only if it is in one-to-one
correspondence with a proper subset of itself}.

\begin{example}
$\N \times \N$ is a countably infinite set.  Proof: Arrange the
elements of $\N \times \N$ as follows
$(1,1)$, $(1,2)$, $(2,1)$, $(1,3)$, $(2,2)$, $(3,1)$, \ldots.  That is,
always write down first all the elements whose two entries sum to $k$,
then write down all the elements whose entries sum to $k+1$ and so on.
Define a bijection with $\N$ by letting 1 go to $(1,1)$,
2 go to $(1,2)$, and so on.  See \figureref{fig:NcrossNcard}.
\begin{myfigureht}
\subimport*{figures/}{NcrossNcard.pdf_t}
\caption{Showing $\N \times \N$ is countable.\label{fig:NcrossNcard}}
\end{myfigureht}
\end{example}

\begin{example}
The set of rational numbers is countable.  Proof: (informal)
For positive rational numbers follow the same procedure
as in the previous example, writing
$\nicefrac{1}{1}$, $\nicefrac{1}{2}$, $\nicefrac{2}{1}$, etc.  However,
leave out fractions (such as $\nicefrac{2}{2}$)
that have already appeared.  The list would continue:
$\nicefrac{1}{3}$, $\nicefrac{3}{1}$, $\nicefrac{1}{4}$,
$\nicefrac{2}{3}$, etc.
For all rational numbers, include $0$ and the negative numbers:
$0$,
$\nicefrac{1}{1}$,
$\nicefrac{-1}{1}$,
$\nicefrac{1}{2}$,
$\nicefrac{-1}{2}$, etc.
\end{example}

For completeness, we mention the following statements
from the exercises.
\emph{If $A \subset
B$ and $B$ is countable, then $A$ is countable.}  The contrapositive of the
statement is that if $A$ is
uncountable, then $B$ is uncountable.
As a consequence, if $\abs{A} < \abs{\N}$, then $A$ is
finite.
Similarly, if $B$ is finite and $A \subset B$, then $A$ is finite.

%FIXMEevillayouthack
\pagebreak[2]
We give the first truly striking result about cardinality.
To do so we need a notation for
the set of all subsets of a set.

\begin{defn}
The \emph{\myindex{power set}} of a set $A$, denoted by $\sP(A)$,
\glsadd{not:powerset} is the set of all subsets of $A$.
\end{defn}

For example, if $A := \{ 1,2\}$,
then $\sP(A) = \bigl\{ \emptyset, \{ 1 \}, \{ 2 \}, \{ 1, 2 \} \bigr\}$.
In particular, $\abs{A} = 2$ and $\abs{\sP(A)} = 4 = 2^2$.
In general,
for a finite set $A$ of cardinality $n$, the
cardinality of $\sP(A)$ is $2^n$.
This fact is left as an exercise.  
Hence, for a finite set $A$,
the cardinality of $\sP(A)$ is strictly
larger than the
cardinality of $A$.  What is an unexpected and
striking fact is that this statement is also true for infinite sets.

\begin{thm}[Cantor%
\footnote{Named after the German mathematician
\href{https://en.wikipedia.org/wiki/Georg_Cantor}{Georg Ferdinand Ludwig
Philipp Cantor} (1845--1918).}]
\index{Cantor's theorem}
\label{cantorspowersetthm}
Let $A$ be a set.  Then
$\abs{A} < \abs{\sP(A)}$.
In particular, there exists no surjection from
$A$ onto $\sP(A)$.
\end{thm}

\begin{proof}
An injection $f \colon A \to \sP(A)$ exists:
For $x \in A$, let $f(x) := \{ x \}$.  Thus,
$\abs{A} \leq \abs{\sP(A)}$.

To finish the proof, we must show that
no function $g \colon A \to \sP(A)$ is a surjection.
Suppose 
$g \colon A \to \sP(A)$ is a function.  So for $x \in A$,
$g(x)$ is a subset of $A$.  Define the set
\begin{equation*}
B := \bigl\{ x \in A : x \notin g(x) \bigr\} .
\end{equation*}
We claim that $B$ is not in the range of $g$ and hence $g$ is not a
surjection.  Suppose for contradiction that there exists an $x_0$ such that $g(x_0) = B$.
Either $x_0 \in B$ or $x_0 \notin B$.  If $x_0 \in B$, then $x_0 \notin
g(x_0) = B$, which is a contradiction.  If $x_0 \notin B$, then $x_0 \in
g(x_0) = B$, which is again a contradiction.  Thus such an $x_0$ does not
exist.  Therefore, $B$ is not in the range of $g$, and $g$ is not a
surjection.  As $g$ was an arbitrary function, no surjection exists.
\end{proof}

One particular consequence of this 
theorem is that there do exist uncountable sets,
as $\sP(\N)$ must be uncountable.
A related fact is that
the set of real numbers (which we study in the next chapter) is uncountable.
The existence of uncountable sets may seem unintuitive, and the theorem
caused quite a controversy at the time
it was announced.  The theorem not only says that uncountable sets exist,
but that there in fact exist progressively larger
and larger infinite sets $\N$, $\sP(\N)$,
$\sP(\sP(\N))$, $\sP(\sP(\sP(\N)))$, etc.

\subsection{Exercises}

\begin{exercise}
Show
$A \setminus (B \cap C) = (A \setminus B) \cup (A \setminus C)$.
\end{exercise}

\begin{exercise}
Prove that the principle of strong induction is equivalent to the standard
induction.
\end{exercise}

\begin{exercise}
Finish the proof of \propref{st:propinv}.
\end{exercise}

\begin{exercise}
\leavevmode
\begin{enumerate}[a)]
\item
Prove \propref{st:propfor}.
\item
Find an example for which equality of sets
in 
$f( C \cap D) \subset f (C) \cap f (D)$
fails.  That is, find an $f$, $A$, $B$, $C$, and $D$ such that
$f( C \cap D)$ is a proper subset of $f(C) \cap f(D)$.
\end{enumerate}
\end{exercise}

\begin{exercise}[Tricky]
Prove that if $A$ is nonempty and finite, then there exists a unique
$n \in \N$ such
that there exists a bijection between $A$ and $\{ 1, 2, 3, \ldots, n \}$.
In other words, the notation $\abs{A} := n$ is justified.
Hint: Show that if $n > m$, then there is no injection from
$\{ 1, 2, 3, \ldots, n \}$ to
$\{ 1, 2, 3, \ldots, m \}$.
\end{exercise}


\begin{exercise}
Prove:
\begin{enumerate}[a)]
\item $A \cap (B \cup C) = (A \cap B) \cup (A \cap C)$.
\item $A \cup (B \cap C) = (A \cup B) \cap (A \cup C)$.
\end{enumerate}
\end{exercise}

\begin{samepage}
\begin{exercise}
Let $A \Delta B$ denote the
\emph{\myindex{symmetric difference}}, that is, the set of all elements that
belong to either $A$ or $B$, but not to both $A$ and $B$.
\begin{enumerate}[a)]
\item
Draw a Venn diagram for
$A \Delta B$.
\item
Show $A \Delta B = (A \setminus B) \cup (B \setminus A)$.
\item
Show $A \Delta B = (A \cup B) \setminus ( A \cap B)$.
\end{enumerate}
\end{exercise}
\end{samepage}

\begin{exercise}
For each $n \in \N$, let $A_n := \{ (n+1)k : k \in \N \}$.
\begin{enumerate}[a)]
\item Find $A_1 \cap A_2$.
\item Find $\bigcup_{n=1}^\infty A_n$.
\item Find $\bigcap_{n=1}^\infty A_n$.
\end{enumerate}
\end{exercise}

\begin{samepage}
\begin{exercise}
Determine $\sP(S)$ (the power set) for each of the following:
\begin{enumerate}[a)]
\item $S = \emptyset$,
\item $S = \{1\}$,
\item $S = \{1,2\}$,
\item $S = \{1,2,3,4\}$.
\end{enumerate}
\end{exercise}
\end{samepage}


\begin{exercise}
Let $f \colon A \to B$ and $g \colon B \to C$ be functions.
\begin{enumerate}[a)]
\item
Prove that if $g \circ f$ is injective, then $f$ is injective.
\item
Prove that if $g \circ f$ is surjective, then $g$ is surjective.
\item
Find an explicit example where $g \circ f$ is bijective, but neither $f$
nor $g$ is bijective.
\end{enumerate}
\end{exercise}

\begin{exercise}
\pagebreak[2]
Prove by induction that $n < 2^n$ for all $n \in \N$.
\end{exercise}

\begin{exercise}
Show that for a finite set $A$ of cardinality $n$, the cardinality
of $\sP(A)$ is $2^n$.
\end{exercise}

\begin{exercise}
Prove $\frac{1}{1\cdot 2} + 
\frac{1}{2\cdot 3} + \cdots + \frac{1}{n(n+1)} = \frac{n}{n+1}$
for all $n \in \N$.
\end{exercise}

\begin{exercise}
Prove $1^3 + 2^3 + \cdots + n^3 = {\left( \frac{n(n+1)}{2} \right)}^2$
for all $n \in \N$.
\end{exercise}

\begin{exercise}
Prove that $n^3 + 5n$ is divisible by $6$ for all $n \in \N$.
\end{exercise}

\begin{exercise}
Find the smallest $n \in \N$ such that $2{(n+5)}^2 < n^3$ and call it $n_0$.
Show that $2{(n+5)}^2 < n^3$ for all $n \geq n_0$.
\end{exercise}

\begin{exercise}
Find all $n \in \N$ such that $n^2 < 2^n$.
\end{exercise}

\begin{exercise}
Prove the 
\hyperlink{wop:link}{well ordering property} of $\N$ using the
\hyperref[induction:thm]{principle of induction}.
\end{exercise}


\begin{exercise}
Give an example of a countably infinite collection of finite sets $A_1, A_2, \ldots$,
whose union is not a finite set.
\end{exercise}

\begin{exercise}
Give an example of a countably infinite collection of infinite sets $A_1, A_2, \ldots$,
with $A_j \cap A_k$ being infinite for all $j$ and $k$, such that
$\bigcap_{j=1}^\infty A_j$
is nonempty and finite.
\end{exercise}

\begin{exercise}
Suppose $A \subset B$ and $B$ is finite. Prove that $A$ is finite.
That is, if $A$ is nonempty, construct a bijection of $A$ to $\{ 1,2,\ldots,n \}$.
\end{exercise}

\begin{exercise}
Prove \propref{prop:equivclasses}.  That is, prove that if $\sR$ is an
equivalence relation on a set $A$, then every $a \in A$ is in exactly one
equivalence class.  Then prove that $a \, \sR \, b$ if and only if $[a] =
[b]$.
\end{exercise}

\begin{exercise}
Prove that the relation `$\sim$'
in \exampleref{example:ratnums} is an equivalence
relation.
\end{exercise}

\begin{exercise}
\leavevmode
\begin{enumerate}[a)]
\item
Suppose $A \subset B$ and $B$ is countably infinite.  By constructing a
bijection, show that $A$ is
countable (that is, $A$ is empty, finite, or countably infinite).
\item 
Use part a) to show that if $\abs{A} < \abs{\N}$, then $A$ is finite.
\end{enumerate}
\end{exercise}

\begin{exercise}[Challenging] \label{exercise:countsubsetbij}
Suppose $\abs{\N} \leq \abs{S}$, or in other words, $S$
contains a countably infinite subset.
Show that there exists a countably infinite subset $A \subset S$ and
a bijection between $S \setminus A$ and $S$.
\end{exercise}
