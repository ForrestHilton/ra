\documentclass[10pt,aspectratio=149]{beamer}

% All the boilerplate is in raslides.sty
% Note that this also pulls in a custom vogtwidebar.sty
\usepackage{raslides}

\author{Ji\v{r}\'i Lebl}

\institute[OSU]{%
Departemento pri Matematiko de Oklahoma {\^S}tata Universitato}

\title{BA: 0.3 part 2}

\date{}

\begin{document}

\theoremstyle{plain}
\newtheorem*{wellordprop}{Well ordering property of $\N$}

\begin{frame}
\titlepage
\end{frame}

\begin{frame}
A common method of proof for statements over $\N$ is \emph{induction}.

\medskip
\pause

Order $\N = \{ 1,2,3,\ldots \}$ naturally:

\[
1 < 2 < 3 < 4 < \cdots 
\]

\medskip
\pause

$S \subset \N$ has a \emph{least element} when $\exists$ $x \in S$ such that
$\forall$
$y \in S$, we have $x \leq y$.

\pause

\begin{wellordprop}
Every nonempty subset of $\N$ has a least element.
\end{wellordprop}

\pause

We take this property as an axiom.

\end{frame}

\begin{frame}
\begin{theorem}[Principle of induction]
Let $P(n)$ be a statement depending on a natural number $n$.
\pause
Suppose that
\begin{enumerate}[(i)]
\item \emph{(basis statement)} $P(1)$ is true.
\pause
\item \emph{(induction step or induction hypothesis)} If $P(n)$ is true, then $P(n+1)$ is true.
\end{enumerate}
\pause
Then $P(n)$ is true for all $n \in \N$.
\end{theorem}
\pause

\textbf{Proof:}
Let $S \coloneqq \{ n \in \N : P(n) \text{ is not true } \}$.

\medskip
\pause

Suppose for contradiction $S \not= \emptyset$.

\medskip
\pause

$S$ has a least element $m \in S$ by the well ordering property.

\medskip
\pause

$1 \notin S$ by hypothesis \quad $\Rightarrow$ \quad
$m > 1$, and $m-1 \in \N$.

\medskip
\pause

$m$ is the least element of $S$ \quad $\Rightarrow$ \quad $P(m-1)$ is true.

\medskip
\pause

The induction step says $P(m-1+1) = P(m)$ is true, so $m \notin S$.

\medskip
\pause

So $m \in S$ \emph{and} $m \notin S$, a contradiction.

\medskip
\pause

So $S = \emptyset$ and $P(n)$ must be true for all $n \in \N$.
\qed.

\medskip
\pause

\textbf{Remark:} It may be convenient to start at a number different than 1.
\end{frame}

\begin{frame}
\textbf{Example:}
Claim: For all $n \in \N$, ~$2^{n-1} \leq n!$  \quad (Recall $n! = 1 \cdot 2 \cdot 3 \cdots n$)

\medskip
\pause

\textbf{Proof:}
Let $P(n)$ be the statement ``$2^{n-1} \leq n!$ is true.''

\pause

$P(1)$ is true by plugging in $n=1$.

\pause

Suppose $P(n)$ is true: $2^{n-1} \leq n!$ holds.

\pause
Multiply both sides by 2 to obtain:
$2^n \leq 2(n!)$.

\pause
As $2 \leq (n+1)$ when $n \in \N$ \quad $\Rightarrow$ \quad $2(n!) \leq (n+1)(n!) = (n+1)!$

\pause
That is, $2^n \leq 2(n!) \leq  (n+1)!$ and hence $P(n+1)$ is true.

\pause
By the principle of induction, $P(n)$
is true for all $n \in \N$.

\medskip
\pause

\textbf{Example:}
Claim: For all $c \not= 1$, we have $1 + c + c^2 + \cdots + c^n = \frac{1-c^{n+1}}{1-c}$

\medskip
\pause

\textbf{Proof:}
Easy to check for $n=1$.

\pause
Suppose it is true for $n$.  Then

\pause
\medskip
$
1 + c + c^2 + \cdots + c^n + c^{n+1}  =
( 1 + c + c^2 + \cdots + c^n ) + c^{n+1}
$

\pause
\medskip
$\qquad = \frac{1-c^{n+1}}{1-c}  + c^{n+1}
\pause
= \frac{1-c^{n+1}  + (1-c)c^{n+1}}{1-c}
\pause
= \frac{1-c^{n+2}}{1-c} .$
\end{frame}

%\begin{frame}
%
%\begin{theorem}[Principle of strong induction]
%Let $P(n)$ be a statement depending on a natural number $n$.  Suppose that
%\begin{enumerate}[(i)]
%\item \emph{(basis statement)} $P(1)$ is true.
%\item \emph{(induction step)} If $P(k)$ is true for all $k = 1,2,\ldots,n$, then $P(n+1)$ is true.
%\end{enumerate}
%Then $P(n)$ is true for all $n \in \N$.
%\end{theorem}
%
%\pause
%
%This is equivalent to regular induction.  Proof is an exercise.
%
%\end{frame}

\begin{frame}
Informally, for sets $A$ and $B$,
a \emph{set-theoretic function} or \emph{mapping} or \emph{map}
\[
f \colon A \to B
\]
is an object that to each $x \in A$ assigns a unique $y \in B$.

\medskip
\pause
\textbf{Example:}
Define $f \colon S \to T$ taking $S \coloneqq \{ 0, 1, 2 \}$ to $T \coloneqq \{ 0, 2 \}$,
by

$f(0) \coloneqq 2$, $f(1) \coloneqq 2$, and $f(2) \coloneqq 0$.

\medskip
\pause

A function $f \colon A \to B$ is a black box,
taking elements of $A$ to elements of $B$.

\medskip
\pause

$f$ may be given by a formula, but $f$ is \textbf{not}
a formula, it is a table of values.

\pause
\medskip

Some functions have several different formulas giving the same function.

\pause
\medskip

Many functions have no formula at all.

\end{frame}

\begin{frame}

%Let's define things rigorously.

\begin{definition}
Let $A$ and $B$ be sets.  The \emph{Cartesian product}
is the set of tuples defined as
\begin{equation*}
A \times B \coloneqq
\bigl\{ (x,y) : x \in A, y \in B \bigr\} .
\end{equation*}
\end{definition}

\pause

\textbf{Example:} $\{ a,b \} \times \{ c , d\} = \bigl\{ (a,c), (a,d), (b,c), (b,d) \bigr\}$.

\medskip
\pause

\textbf{Example:} $[0,1] \times [0,1]$ is the subset of
the plane bounded by a square with vertices $(0,0)$, $(0,1)$, $(1,0)$, and $(1,1)$.

\medskip
\pause

\textbf{Notation:} $A^2 \coloneqq A \times A$.  E.g., $\R^2 = \R \times \R$ is the
plane.

\medskip
\pause

%\end{frame}

%\begin{frame}

\begin{definition}
A \emph{function} $f \colon A \to B$ is a subset $f$ of $A \times B$
such that for each $x \in A$, there exists a unique $y \in B$ for which $(x,y) \in f$.
\pause
We write $f(x) = y$.
\pause
Sometimes
the set $f$ is called the \emph{graph} of the function rather than
the function itself.

\medskip
\pause
$A$ is called the \emph{domain} of $f$ and
$B$ is called the \emph{codomain} of $f$.
\end{definition}

\end{frame}

\begin{frame}

\textbf{Examples:} Functions from calculus taking real numbers
to real numbers.

\pause
\medskip
``derivative'' or ``Laplace transform'' is a function that takes
functions to functions.

\pause
\medskip
Determinant from linear algebra is a function that takes matrices to
numbers.

\pause
\medskip

The ``number of classrooms'' is a function that takes the set of buildings
on campus to the integers.
\end{frame}

\begin{frame}
\begin{definition}
Consider $f \colon A \to B$ and subsets $C \subset A$ and $D \subset B$.

\pause
The \emph{image} (or \emph{direct image}) is
\quad $f(C) \coloneqq \bigl\{ f(x) \in B : x \in C \bigr\}$.

\pause
The set $f(A)$ is called the \emph{range} of $f$.

\pause
The \emph{inverse image} is \quad
$f^{-1}(D) \coloneqq \bigl\{ x \in A : f(x) \in D \bigr\}$.
\end{definition}

\pause

\textbf{Example:}
Define
$f \colon \{ 1,2,3,4 \} \to \{ a,b,c,d \}$ by

$f(1) \coloneqq b$,
$f(2) \coloneqq d$,
$f(3) \coloneqq c$,
$f(4) \coloneqq b$

\subimport*{../figures/}{funcimags_full.pdf_t}

\end{frame}

\begin{frame}

%\begin{example}
%Define the function $f \colon \R \to \R$ by
%$f(x) \coloneqq \sin(\pi x)$.  Then $f\bigl([0,\nicefrac{1}{2}]\bigr) = [0,1]$, 
%$f^{-1}\bigl(\{0\}\bigr) = \Z$, etc.
%\end{example}

\begin{proposition}
Consider $f \colon A \to B$.  Let $C, D \subset B$.  \pause Then

\medskip
$f^{-1}( C \cup D) = f^{-1} (C) \cup f^{-1} (D) , \quad
f^{-1}( C \cap D) = f^{-1} (C) \cap f^{-1} (D)$,

\medskip
\pause
$f^{-1}( C^c) = {\bigl( f^{-1} (C) \bigr)}^c$
\quad
(that is $f^{-1}( B \setminus C) = A \setminus f^{-1} (C)$).
\end{proposition}

\pause
\textbf{Proof:}
Start with the union.

\pause
If $x \in f^{-1}( C \cup D)$, then $f(x) \in C$ or $f(x) \in D$.

\pause
\thus\quad  $f^{-1}( C \cup D) \subset f^{-1} (C) \cup f^{-1} (D)$.

\pause
Conversely if $x \in f^{-1}(C)$, then $x \in f^{-1}(C \cup D)$.
\pause
\quad Similarly for $x \in f^{-1}(D)$.

\pause

\thus \quad $f^{-1}( C \cup D) \supset f^{-1} (C) \cup f^{-1} (D)$, and we have
equality.

\medskip
\pause

The rest of the proof is left as an exercise.
\qed

\pause

\begin{proposition}
Consider $f \colon A \to B$.  Let $C, D \subset A$. \pause  Then

$f( C \cup D) = f (C) \cup f (D) , \qquad
f( C \cap D) \subset f (C) \cap f (D)$.
\end{proposition}

\pause
Proof is an exercise
\end{frame}

\begin{frame}
\begin{definition}
$f \colon A \to B$ is
\emph{injective}, \emph{one-to-one}, or an \emph{injection}
if $f(x_1) = f(x_2)$ $\Rightarrow$ $x_1 = x_2$.

%\pause
%I.e.,
%$f$ is injective if $\forall y \in B$, the set
%$f^{-1}(\{y\})$ is empty or a single element.

\pause
\medskip

If $f(A) = B$, then $f$ is
\emph{surjective},
\emph{onto}, or a \emph{surjection}.

%\pause
%I.e., $f$ is surjective if range = codomain.

\pause
\medskip
If $f$ is both surjective and injective, then
$f$ is \emph{bijective} or a \emph{bijection}.

\pause
\medskip

If $f$ is a bijection, then $\forall y \in B$, $f^{-1}(\{y\})$ is a unique element of $A$,
so by slight abuse of notation
consider
the \emph{inverse function}
$f^{-1} \colon B \to A$
and write $f^{-1}(y)$.
\end{definition}

\pause
\textbf{Example:}
$f \colon \R \to \R$ given by $f(x) \coloneqq
x^3$ is a bijection.

\pause
The inverse function is defined by $f^{-1}(x) = \sqrt[3]{x}$.

\medskip
\pause
\textbf{Example:}
$f \colon \R \to \R$ given by $f(x) \coloneqq
x^2$ is neither an injection as $f(-1) = 1$ \text{and} $f(1)=1$,
nor a surjection as there is no $x$ such that $f(x) = -1$.

\medskip
\pause
\textbf{Example:}
$f \colon \R \to [0,\infty)$ (the non-negative
real numbers) given by $f(x) \coloneqq x^2$ is not an injection (same reason as
above), but 
it is a surjection.

\end{frame}

\begin{frame}
\begin{definition}
The \emph{composition} of $f \colon A \to B$ and $g \colon B \to C$ is
the function
$g \circ f \colon A \to C$
\[
(g \circ f)(x) \coloneqq g\bigl(f(x)\bigr) .
\]
\end{definition}

\pause

E.g., if $f \colon \R \to \R$ is $f(x)\coloneqq x^3$ and $g \colon \R \to \R$
is $g(y) = \sin(y)$,

\pause
then $(g \circ f)(x) = \sin(x^3)$.

\medskip
\pause

Easy exercise:

\pause
a composition of one-to-one maps is one-to-one,

\pause
a composition of onto maps is onto.

\pause
So a composition of bijections is a bijection.
\end{frame}

\end{document}
