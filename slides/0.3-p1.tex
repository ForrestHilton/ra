\documentclass[10pt,aspectratio=169]{beamer}

% All the boilerplate is in raslides.sty
% Note that this also pulls in a custom vogtwidebar.sty
\usepackage{raslides}

\author{Ji\v{r}\'i Lebl}

\institute[OSU]{%
Departemento pri Matematiko de Oklahoma {\^S}tata Universitato}

\title{BA: 0.3 part 1}

\date{}

\begin{document}

\begin{frame}
\titlepage
\end{frame}

\begin{frame}
We start with a set theory refresher.

\pause

\begin{definition}
A \emph{set} is a collection of objects called
\emph{elements} or \emph{members}.  A set with
no objects is called the \emph{empty set} and is denoted by
$\emptyset$ (or sometimes by $\{ \}$).
\end{definition}

\pause

Set is a ``club.''  An element can be a member of
several clubs.

Moreover, sets (clubs) can be members as well.

\medskip
\pause

$S \coloneqq \{ 0, 1, 2 \}$
\quad is the set containing
the three elements 0, 1, and 2.

\medskip
\pause

By ``$\coloneqq$'' we mean definition of $S$.

\medskip
\pause

``$1 \in S$''
means ``1 belongs to $S$.''

\medskip
\pause

``$1,2 \in S$'' is shorthand for
``$1 \in S$ and $2 \in S$.''

\medskip
\pause

``$7 \notin S$'' means ``7 does not belong to $S$.''

\medskip
\pause

Elements come from a \emph{universe}.

\medskip
\pause

Universe is understood from context.  E.g., real numbers.

\end{frame}

\begin{frame}
Every element of $T \coloneqq \{ 0 , 2 \}$ is also an element of $S$.
\quad($S=\{0,1,2\}$)

\medskip
\pause

Then write $T \subset S$.

\hspace*{1.6in}\subimport*{../figures/}{sets.pdf_t}

\pause

\begin{definition}
\leavevmode
\begin{enumerate}[(i)]
\item
A set $A$ is a \emph{subset}
of a set $B$ if $x \in A$ implies $x \in B$, and we write $A \subset B$

(or
$B \supset A$).
%That is, all members of $A$ are also members of $B$.  At times we
%write $B \supset A$ to mean the same thing.
\item \pause
Two sets $A$ and $B$ are \emph{equal} if $A \subset B$ and $B
\subset A$.  We write $A = B$.
%That is, $A$ and $B$ contain exactly the same elements.

If $A$ and $B$ are not equal we write $A \not= B$.
\item \pause
A set $A$ is a \emph{proper subset} of $B$ if $A \subset B$
and $A \not= B$.  We write $A \subsetneq B$.
\end{enumerate}
\end{definition}

\pause
E.g., $T \subset S$, but $T \not= S$ so $T \subsetneq S$.
\end{frame}

\begin{frame}
\emph{Set building notation}:
\quad $\bigl\{ x \in A : P(x) \bigr\}$.

\pause
\medskip

Subset of $A$ containing elements that satisfy property $P(x)$.

\medskip
\pause

E.g., $\{ x \in S : x \not= 2 \} = \{ 0, 1 \}$.

\medskip
\pause

May abbreviate as $\bigl\{ x : P(x) \bigr\}$ if $A$ is understood
from context.

\medskip
\pause

``$x \in A$'' can also be replaced by a formula.

\medskip
\pause

\textbf{Example sets:}

\medskip
\pause

\emph{Natural numbers}, $\N \coloneqq \{ 1, 2, 3, \ldots \}$.

\medskip
\pause

\emph{Integers}, $\Z \coloneqq \{ 0, -1, 1, -2, 2, \ldots \}$.

\medskip
\pause

\emph{Rational numbers}, $\Q \coloneqq \bigl\{ \frac{m}{n} :  m, n \in \Z \text{ and } n \not= 0 \bigr\}$.

\medskip
\pause

Even natural numbers, $\{  2m : m \in \N \}$.

\medskip
\pause

The set of all real numbers is denoted by $\R$.

\medskip
\pause

Note: $\N \subset \Z \subset \Q \subset \R$.

\end{frame}

\begin{frame}
\begin{definition}
\begin{enumerate}[(i)]
\item
A \emph{union} of two sets $A$ and $B$:

$A \cup B \coloneqq \{ x : x \in A \text{ or } x \in B \}$.
\item
\pause
An \emph{intersection} of two sets $A$ and $B$:

$A \cap B \coloneqq \{ x : x \in A \text{ and } x \in B \}$.
\item
\pause
A \emph{complement of $B$ relative to $A$}
(or \emph{set-theoretic difference} of $A$ and $B$):

$A \setminus B \coloneqq \{ x : x \in A \text{ and } x \notin B \}$.
\item
\pause
We say
\emph{complement} of $B$ and write $B^c$
instead of $A \setminus B$ if
$A$ is either the universe or the obvious
set containing $B$ and understood from context.
\item
\pause
$A$ and $B$ are \emph{disjoint} if $A \cap B = \emptyset$.
\end{enumerate}
\end{definition}
\end{frame}

\begin{frame}
\begin{center}
\scalebox{0.9}{
\subimport*{../figures/}{figsetop.pdf_t}}
\end{center}
\end{frame}

\begin{frame}
\begin{theorem}[DeMorgan]
Let $A, B, C$ be sets.  Then
\quad
${(B \cup C)}^c = B^c \cap C^c , \quad
{(B \cap C)}^c = B^c \cup C^c$,

\medskip
\pause

or,
\quad
$A \setminus (B \cup C) = (A \setminus B) \cap (A \setminus C) , \quad
A \setminus (B \cap C) = (A \setminus B) \cup (A \setminus C)$.
\end{theorem}

\pause
\textbf{Proof:}
The second statement proves the first if $A$ is the ``universe.''

\medskip
\pause

Let us prove $A \setminus (B \cup C) = (A \setminus B) \cap (A \setminus C)$
\quad (remember the definition of $=$).

\medskip
\pause

%We must show
%
%First, \quad  $x \in A \setminus (B \cup C)$ \quad $\Rightarrow$ \quad $x \in (A \setminus B) \cap (A \setminus C)$.
%
%Second, \quad $x \in (A \setminus B) \cap (A \setminus C)$ \quad $\Rightarrow$ \quad $x \in A \setminus (B \cup C)$.
%
%\medskip
%\pause

First, suppose
$x \in A \setminus (B \cup C)$.

\pause

Then $x$ is in $A$, but not in $B$ nor $C$.

\pause

Hence $x$ is in $A$ and not in $B$, that is, $x \in A \setminus B$.

\pause

Similarly $x \in A \setminus C$.

\pause

Thus $x \in (A \setminus B) \cap (A \setminus C)$.

\medskip
\pause

On the other hand, suppose 
$x \in (A \setminus B) \cap (A \setminus C)$.

\pause

In particular, $x \in (A \setminus B)$, so $x \in A$ and $x \notin B$.

\pause

As also $x \in (A \setminus C)$, then $x \notin C$.

\pause

Hence $x \in A \setminus (B \cup C)$.

\medskip
\pause

The proof of the other equality is left as an exercise.
\qed
\end{frame}

\begin{frame}
Results called \emph{Theorem}, \emph{Proposition}, \emph{Lemma},
\emph{Corollary}\ldots

\medskip
\pause

Often:

Theorem is a bigger result than Proposition,

\pause
Lemma is a tool or an intermediate step,

\pause
and Corollary is a quick consequence of something.

\medskip
\pause

Often it is just traditional or stylistic.  Do not read into it
\emph{too} much.

\medskip
\pause

Important to distinguish between results (as above)
and an \emph{Axiom}
or a \emph{Postulate}, which are statements assumed to be true
without proof.

\medskip
\pause

We also encountered \emph{Definition}s which simply give language or
notation, but do not actually say anything is true or false.
\pause
(Confusing sometimes when one says ``by definition we have this or that,''
but that just means we write out what the definition means using other
statements, not that the definition itself implies something.)

\end{frame}

\begin{frame}
Unions/intersections can be taken over more sets.
\pause
If
$\{ A_1, A_2, A_3, \ldots \}$,
\pause
then
\begin{align*}
& \bigcup_{n=1}^\infty A_n \coloneqq \{ x : x \in A_n \text{ for some } n \in \N
\} , \\
& \bigcap_{n=1}^\infty A_n \coloneqq \{ x : x \in A_n \text{ for all } n \in \N
\} .
\end{align*}
\pause
We could index by other sets such as two natural numbers:
$\{ A_{1,1}, A_{1,2}, A_{2,1}, A_{1,3}, A_{2,2}, A_{3,1}, \ldots \}$.
\pause
\begin{equation*}
\bigcup_{n=1}^\infty \bigcup_{m=1}^\infty A_{n,m}
=
\bigcup_{n=1}^\infty \left( \bigcup_{m=1}^\infty A_{n,m} \right) .
\end{equation*}
\pause
Same for intersections.
\pause

\medskip
It is not hard to see that the order of the unions can be swapped.

Same for intersections.

\end{frame}

\begin{frame}
But switching mixed unions/intersections is not always possible.

\medskip
\pause

\textbf{Example:}
\begin{equation*}
\bigcup_{n=1}^\infty
\bigcap_{m=1}^\infty
\{ k \in \N : mk < n \}
=
\bigcup_{n=1}^\infty \emptyset = \emptyset .
\end{equation*}
\pause
However,
\begin{equation*}
\bigcap_{m=1}^\infty
\bigcup_{n=1}^\infty
\{ k \in \N : mk < n \}
=
\bigcap_{m=1}^\infty
\N
=
\N.
\end{equation*}

\pause
The index set could be another set than natural numbers in which case we use
the following notation:
\begin{equation*}
\bigcup_{\lambda \in I} A_\lambda \coloneqq \{ x : x \in A_\lambda \text{ for some }
\lambda \in I
\} ,
\qquad
\bigcap_{\lambda \in I} A_\lambda \coloneqq \{ x : x \in A_\lambda \text{ for all }
\lambda \in I
\} .
\end{equation*}

\end{frame}

\end{document}
