\documentclass[10pt,aspectratio=169]{beamer}

% All the boilerplate is in raslides.sty
% Note that this also pulls in a custom vogtwidebar.sty
\usepackage{raslides}

\author{Ji\v{r}\'i Lebl}

\institute[OSU]{%
Departemento pri Matematiko de Oklahoma {\^S}tata Universitato}

\title{BA: 1.1}

\date{}

\begin{document}

\begin{frame}
\titlepage
\end{frame}

\begin{frame}
\begin{definition}
An \emph{ordered set} is a set $S$ together with
a relation $<$ such that
\pause
\begin{enumerate}[(i)]
\item \emph{(trichotomy)} For all $x, y \in S$, exactly one of
$x < y$, $x=y$, or $y < x$ holds.
\pause
\item \emph{(transitivity)} If $x,y,z \in S$ are such that $x < y$ and $y
< z$, then $x < z$.
\end{enumerate}
\pause
We write $x \leq y$ if $x < y$ or $x=y$.  We define
$>$ and $\geq$ in the obvious way.
\end{definition}
\pause

\textbf{Examples:}

\medskip

$\Z$ is an ordered set by letting $x < y$ if and only if $y-x = p$ where $p \in \N$.

\medskip
\pause

$\Q$ is an ordered set by letting $x < y$ if and only if $y-x = \nicefrac{p}{q}$ where $p,q \in \N$.

\medskip
\pause

The set of words is an ordered set by using lexicographic ordering.

\end{frame}

\begin{frame}

\begin{definition}
Let $E \subset S$, where $S$ is an ordered set.
\pause
\begin{enumerate}[(i)]
\item If $\exists b \in S$ such that $x \leq b$ for all $x \in E$,

then $E$ is \emph{bounded above} and $b$
is an \emph{upper bound} of $E$.
\pause
\item If $\exists b \in S$ such that $x \geq b$ for all $x \in E$,

then $E$ is \emph{bounded below} and $b$
is a \emph{lower bound} of $E$.
\pause
\item 
If $\exists$ an upper bound $b_0$ of $E$ such that
$b_0 \leq b$ for all upper bounds $b$ of $E$, then
$b_0$ is called the \emph{least upper bound} or
the \emph{supremum} of $E$.
Write
\[
\sup\, E \coloneqq b_0  .
\]
\item
\pause
If $\exists$ a lower bound $b_0$ of $E$ such that
$b_0 \geq b$ for all lower bounds $b$ of $E$, then
$b_0$ is called the \emph{greatest lower bound} or
the \emph{infimum} of $E$.
Write
\[
\inf\, E \coloneqq b_0  .
\]
\end{enumerate}
\pause
If $E$ is bounded above and bounded below, we say that
$E$ is \emph{bounded}.
\end{definition}
\end{frame}

\begin{frame}
\qquad\scalebox{0.85}{
\subimport*{../figures/}{lub.pdf_t}
}

\pause
\medskip

Notation $\sup E$ and $\inf E$ is justified as
the
supremum (or infimum) is unique (if it exists):

\pause
\medskip

If $b$ and
$b'$ are suprema of $E$, then $b \leq b'$ and $b' \leq b$, because both
$b$ and $b'$ are the least upper bounds, so $b=b'$.

\pause
\medskip

\textbf{Example:}

$S \coloneqq \{ a, b, c, d, e \}$ ordered as $a < b < c < d < e$.
\qquad
$E \coloneqq \{ a, c \}$.

\pause

$c$, $d$, and $e$ are upper bounds of $E$,

\pause

$c$ is the least upper bound or supremum of $E$.

\medskip
\pause

\textbf{Example:}
$E \coloneqq \{ x \in \Q : x < 1 \} \subset \Q$ has a least upper bound 1, but
$1 \notin E$.

\medskip
\pause

\textbf{Example:}
$P \coloneqq \{ x \in \Q : x \geq 0 \} \subset \Q$ has no upper bound.

\end{frame}

\begin{frame}

\begin{definition}
An ordered set $S$ has the \emph{least-upper-bound property} if
every nonempty
$E \subset S$ that is bounded above has a least upper bound
($\sup\, E$ exists in $S$).
\end{definition}

\pause

Also called the \emph{completeness property} or the
\emph{Dedekind completeness property}.

\medskip
\pause

\textbf{Example:}
$\Q$ does not have the least-upper-bound property.

\pause
\medskip

$\{ x \in \Q : x^2 < 2 \}$ does not have a supremum in $\Q$.

\medskip
\pause

We will show later that the supremum would be $\sqrt{2}$, but $\sqrt{2}
\notin \Q$:

\medskip
\pause

Suppose $x \in \Q$ such that $x^2 = 2$.
\pause

$x=\nicefrac{m}{n}$ in lowest terms.

\pause

So ${(\nicefrac{m}{n})}^2 = 2$ or $m^2 = 2n^2$.

\pause

Hence, $m^2$ is divisible by 2, and so $m$ is divisible by 2.

\pause

Write $m = 2k$ and so ${(2k)}^2 = 2n^2$.

\pause
Thus $2k^2 = n^2$, and hence $n$ is divisible by 2.

\pause

Contradiction as  as $\nicefrac{m}{n}$ is in lowest terms.

\medskip
\pause

This is the main reason why analysis needs $\R$ and not just $\Q$.

\end{frame}

\begin{frame}

\begin{definition}
A set $F$ is called a \emph{field} if it has two operations
defined on it, addition $x+y$ and multiplication $xy$, and if it satisfies
the following axioms:
\pause
\begin{enumerate}[({A}1)]
\item
$x \in F$ and $y \in F$ $\Rightarrow$ $x+y \in F$.
\pause
\item
\emph{(commutativity of addition)}
$x+y = y+x$ for all $x,y \in F$.
\pause
\item
\emph{(associativity of addition)}
$(x+y)+z = x+(y+z)$ for all $x,y,z \in F$.
\pause
\item
$\exists 0 \in F$ such that
$0+x = x$ for all $x \in F$.
\pause
\item
For every $x\in F$, there exists $-x \in F$
such that $x + (-x) = 0$.
\end{enumerate}
\begin{enumerate}[({M}1)]
\pause
\item
$x \in F$ and $y \in F$ $\Rightarrow$ $xy \in F$.
\pause
\item
\emph{(commutativity of multiplication)}
$xy = yx$ for all $x,y \in F$.
\pause
\item
\emph{(associativity of multiplication)}
$(xy)z = x(yz)$ for all $x,y,z \in F$.
\pause
\item
There exists $1 \in F$ (and $1 \not= 0$) such that
$1x = x$ for all $x \in F$.
\pause
\item
For every $x\in F$ such that $x \not= 0$ there exists
$\nicefrac{1}{x} \in F$
such that $x(\nicefrac{1}{x}) = 1$.
\pause
\item[(D)]
\emph{(distributive law)} $x(y+z) = xy+xz$
for all $x,y,z \in F$.
\end{enumerate}
\end{definition}

\end{frame}

\begin{frame}

\textbf{Example:} $\Q$ is a field.

\medskip
\pause

\textbf{Example:} $\Z$ is not a field: No $x \in \Z$ such that $2x=1$,
so (M5) not satisfied.

\medskip
\pause

This is not an algebra class, so we'll just assume basic properties
that follow directly from the axioms without proofs.

\medskip
\pause

\begin{definition}
A field $F$ is said to be an \emph{ordered field} if
$F$ is also an ordered set such that
\begin{enumerate}[(i)]
\pause
\item \label{defn:ordfield:i} For $x,y,z \in F$,  $x < y$ implies $x+z <
y+z$.
\pause
\item \label{defn:ordfield:ii} For $x,y \in F$, $x > 0$ and $y > 0$
implies $xy > 0$.
\end{enumerate}
\pause
If $x > 0$, we say $x$ is \emph{positive}.

\pause
If $x < 0$, we say $x$ is \emph{negative}.

\pause
We say $x$ is \emph{nonnegative} if $x \geq 0$,

\pause
and $x$ is \emph{nonpositive} if $x \leq 0$.
\end{definition}

\pause

\textbf{Example:}
Not hard to check that $\Q$ is an ordered field.

\end{frame}

\begin{frame}

\begin{proposition}
Let $F$ be an ordered field and $x,y,z,w \in F$.  Then
\begin{enumerate}[(i)]
\pause
\item \label{prop:bordfield:i} If $x > 0$, then $-x < 0$ (and vice versa).
\pause
\item \label{prop:bordfield:ii} If $x > 0$ and $y < z$, then $xy < xz$.
\pause
\item \label{prop:bordfield:iii} If $x < 0$ and $y < z$, then $xy > xz$.
\pause
\item \label{prop:bordfield:iv} If $x \not= 0$, then $x^2 > 0$.
\pause
\quad\text{In particular, $1 > 0$.}
\pause
\item \label{prop:bordfield:v} If $0 < x < y$, then $0 < \nicefrac{1}{y} < \nicefrac{1}{x}$.
\pause
\item \label{prop:bordfield:vi} If $0 < x < y$, then $x^2 < y^2$.
\pause
\item \label{prop:bordfield:vii} If $x \leq y$ and $z \leq w$, then $x + z \leq y + w$.
\end{enumerate}
\end{proposition}

\pause

\textbf{Proof:}
\eqref{prop:bordfield:i}:  $x > 0$ implies $x + (-x) > 0 + (-x)$
(item (i) of definition).
\pause

Then $0 > -x$. 
\pause
\quad ``vice versa'' follows by similar calculation.

\medskip
\pause

\eqref{prop:bordfield:ii}: $y < z$ implies $0 < z - y$ (item (i) of def).  
\pause

So by item (ii) of the definition: $0 < x(z-y)$.

\pause
Hence $0 < xz - xy$ and by item (i) again, $xy < xz$.

\end{frame}

\begin{frame}

``(iii) If $x < 0$ and $y < z$, then $xy > xz$.'' left as exercise.

\medskip
\pause

``(iv) If $x \not= 0$, then $x^2 > 0$.'':

\pause

Suppose $x > 0$.  By item (ii) of definition, $x^2 > 0$.

\pause

If $x < 0$, use (iii) (with $y=x$, $z=0$) to get $x^2 > 0$.

\medskip
\pause

``(v) If $0 < x < y$, then $0 < \nicefrac{1}{y} < \nicefrac{1}{x}$.'':

\pause

Note $\nicefrac{1}{x} \not=0$.

\pause

If $\nicefrac{1}{x} < 0$, then $\nicefrac{-1}{x} > 0$ by \eqref{prop:bordfield:i}.

\pause

$x > 0$ and $\nicefrac{-1}{x} > 0$ ~~\thus~~
$x(\nicefrac{-1}{x}) > 0$ ~~\thus~~ $-1 > 0$, contradicting $1 > 0$
(use \eqref{prop:bordfield:i})

\pause
So $\nicefrac{1}{x} > 0$.
\pause
\qquad Similarly, $\nicefrac{1}{y} > 0$.

\pause
Thus $(\nicefrac{1}{x})(\nicefrac{1}{y}) > 0$.

\pause
By \eqref{prop:bordfield:ii}
$(\nicefrac{1}{x})(\nicefrac{1}{y})x < (\nicefrac{1}{x})(\nicefrac{1}{y})y$.

\pause
So $\nicefrac{1}{y} < \nicefrac{1}{x}$.

\medskip
\pause

``(vi) If $0 < x < y$, then $x^2 < y^2$.''

``(vii) If $x \leq y$ and $z \leq w$, then $x + z \leq y + w$.''

left as exercises.
\qed

\medskip
\pause

\textbf{Example:}
The complex numbers $\C$ (numbers $x+iy$ where $x,y \in \R$ and $i^2=-1$)
is not an ordered field:  In every ordered field $-1 < 0$.


\end{frame}

\begin{frame}

\begin{proposition}
Let $x,y \in F$, where $F$ is an ordered field.  If
$xy > 0$, then either both $x$ and $y$ are positive, or both are negative.
\end{proposition}

\pause
\textbf{Proof:}
We show the contrapositive: If $x=0$ or $y=0$, or
if $x$ and $y$ have opposite signs, then $xy$ is not positive.

\medskip
\pause

If $x=0$ or $y=0$ is zero, then $xy=0$ and hence not positive.

\medskip
\pause

Suppose $x$ and $y$ are nonzero and have opposite signs.

\pause

WLOG $x > 0$ and $y < 0$.

\pause

Multiply $y < 0$ by $x$ to get $xy < 0x = 0$.
\qed

\end{frame}

\begin{frame}

\begin{proposition}
Let $F$ be an ordered field with the least-upper-bound property.
Let $A \subset F$ be a nonempty set that is bounded below.
Then $\inf\, A$ exists.
\end{proposition}

\pause

\textbf{Proof:}
Let $B \coloneqq \{ -x : x \in A \}$.

\pause
Let $b \in F$ be a lower bound for $A$:
If $x \in A$, then $x \geq b$. In other words, $-x \leq -b$.

\pause
So $-b$ is an upper bound for $B$.

\pause
$F$ has the least-upper-bound property $\Rightarrow$ $c\coloneqq\sup\, B$ exists, and $c \leq -b$.

\pause
As $y \leq c$ for all $y \in B$, then $-c \leq x$ for all $x \in A$.

\pause
So $-c$ is a lower bound for $A$.

\pause
As $-c \geq b$,~~
$-c$ is the greatest lower bound of $A$.
\qed

\end{frame}

\end{document}
