\documentclass[10pt,aspectratio=169]{beamer}

% All the boilerplate is in raslides.sty
% Note that this also pulls in a custom vogtwidebar.sty
\usepackage{raslides}

\author{Ji\v{r}\'i Lebl}

\institute[OSU]{%
Departemento pri Matematiko de Oklahoma {\^S}tata Universitato}

\title{BA: 2.4}

\date{}

\begin{document}

\begin{frame}
\titlepage
\end{frame}

\begin{frame}
It would be nice to check convergence without knowing the limit.

\pause

\begin{definition}
A sequence $\{ x_n \}_{n=1}^\infty$ is a \emph{Cauchy sequence}
if for every $\epsilon > 0$ there exists an $M \in \N$ such that
for all $n \geq M$ and all $k \geq M$, we have
\quad
$\abs{x_n - x_k} < \epsilon$.
\end{definition}

\pause

\textbf{Example:}
$\{ \nicefrac{1}{n} \}_{n=1}^{\infty}$ is a Cauchy sequence.

\medskip
\pause

\textbf{Proof:}  Given $\epsilon > 0$, $\exists$ $M$ such that
$M > \nicefrac{2}{\epsilon}$.

\pause
\thus~ for $n,k \geq M$, \quad
$\nicefrac{1}{n} < \nicefrac{\epsilon}{2}$
and
$\nicefrac{1}{k} < \nicefrac{\epsilon}{2}$.

\pause
\medskip

\thus ~ for $n, k \geq M$, \quad
$\displaystyle \abs{\frac{1}{n} - \frac{1}{k}}
\pause
\leq
\abs{\frac{1}{n}} + \abs{\frac{1}{k}}
\pause
< \frac{\epsilon}{2} + \frac{\epsilon}{2} = \epsilon$. \qed

\pause
\medskip

\textbf{Example:}
$\bigl\{ {(-1)}^n \bigr\}_{n=1}^\infty$ is not a Cauchy sequence.

\medskip
\pause

\textbf{Proof:}
Given any $M \in \N$, take $n \geq M$ to even, and $k \coloneqq n+1$.

\pause
Then
\[
\abs{{(-1)}^n - {(-1)}^k}
\pause
=
\abs{{(-1)}^n - {(-1)}^{n+1}}
\pause
=
\abs{1 - (-1)}
\pause
= 2 .
\]
\pause
\thus
\quad for any $\epsilon \leq 2$ the definition is not satisfied.
%
%\medskip
%
\pause
\quad
\thus \quad The sequence is not Cauchy. \qed

\end{frame}

\begin{frame}

\begin{proposition}
A Cauchy sequence is bounded.
\end{proposition}

\pause
\textbf{Proof:}
Suppose $\{ x_n \}_{n=1}^\infty$ is Cauchy.

\pause
\medskip

Pick an $M$ such that for all
$n,k \geq M$, we have $\abs{x_n-x_k} < 1$.

\pause
\medskip

In particular,
for all $n \geq M$, \quad
$\abs{x_n - x_M} < 1$.

\pause
By the reverse triangle inequality, \quad
$\abs{x_n} - \abs{x_M} \leq \abs{x_n - x_M} < 1$.

\pause
\medskip

Hence for $n \geq M$, \quad
$\abs{x_n} < 1 + \abs{x_M}$.

\pause
\medskip

Let
$B \coloneqq \max \bigl\{ \abs{x_1}, \abs{x_2}, \ldots, \abs{x_{M-1}}, 1+ \abs{x_M}
\bigr\}$.

\pause
\medskip

Then $\abs{x_n} \leq B$ for all $n \in \N$.
\qed
\end{frame}

\begin{frame}

\begin{theorem}
A sequence of real numbers is Cauchy if and only if it converges.
\end{theorem}

\pause
\textbf{Proof:}
Let $\epsilon > 0$ be given and
suppose $\{ x_n \}_{n=1}^\infty$ converges to $x$.

\pause
$\exists$ $M$ such that for $n \geq M$, \quad
$\abs{x_n - x} < \frac{\epsilon}{2}$.

\pause
\medskip

\thus ~ for $n,k \geq M$,
\quad
$\abs{x_n - x_k}
\pause
=
\abs{x_n - x + x - x_k}
\pause
\leq \abs{x_n-x} + \abs{x-x_k}
\pause
< \frac{\epsilon}{2} + \frac{\epsilon}{2} = \epsilon$.

\end{frame}

\begin{frame}

Now suppose $\{ x_n \}_{n=1}^\infty$ is Cauchy.  \pause Thus
$\{x_n\}_{n=1}^\infty$ is bounded.

\pause
\medskip

Define $a \coloneqq \limsup\limits_{n\to \infty} x_n$ and
$b \coloneqq \liminf\limits_{n\to \infty} x_n$.

\pause
By a theorem from 2.3, $\exists$ subsequences
$\{ x_{n_i} \}_{i=1}^\infty$ and
$\{ x_{m_i} \}_{i=1}^\infty$, such that

$\displaystyle
\lim_{i\to\infty} x_{n_i} = a
\quad \text{and} \quad
\lim_{i\to\infty} x_{m_i} = b$.

\pause
\medskip

Given an $\epsilon > 0$,

\pause
$\exists$
$M_1$ such that
$\abs{x_{n_i} - a} < \nicefrac{\epsilon}{3}$ for all $i \geq M_1$,

\pause
$\exists$
$M_2$ such that
$\abs{x_{m_i} - b} < \nicefrac{\epsilon}{3}$ for all $i \geq M_2$,

\pause
$\exists$
$M_3$ such that
$\abs{x_n-x_k} < \nicefrac{\epsilon}{3}$ for all $n,k \geq M_3$.

\pause
Let $M \coloneqq \max \{ M_1, M_2, M_3 \}$.

\pause
\medskip

Consider $i \geq M$. \pause  Then $n_i \geq M$ and $m_i \geq M$.

\pause
\medskip

%\thus\quad
$\displaystyle
\abs{a-b}
\pause
=
\abs{a-x_{n_i}+x_{n_i}
-x_{m_i}+x_{m_i}
-b}
%$
%
\pause
%\medskip
%
%\qquad\qquad
%$\displaystyle
\leq
\abs{a-x_{n_i}}
+ \abs{x_{n_i} -x_{m_i}}
+ \abs{x_{m_i} -b}
\pause
 <
\frac{\epsilon}{3}
+
\frac{\epsilon}{3}
+
\frac{\epsilon}{3}
= \epsilon$.

\pause
\medskip

$\abs{a-b} < \epsilon$ for all $\epsilon > 0$ \wthus $a=b$.

\pause
\medskip

So ~$\displaystyle \limsup_{n\to\infty} x_n = \liminf_{n\to\infty} x_n$~ and
$\{ x_n \}_{n=1}^\infty$ converges.
\qed

\end{frame}

\begin{frame}

\textbf{Remark:}
A set $X$ is
\emph{Cauchy-complete}
if every Cauchy sequence converges in $X$.

\pause
Can be used to define completeness of $\R$ instead of
the least-upper-bound property.

\pause
We proved that for $\R$,
least-upper-bound property implies
Cauchy-complete.

\pause
A way to construct $\R$ is to
``complete'' $\Q$ by ``throwing in'' enough points to make all
Cauchy sequences converge.

\pause
The resulting field has the
least-upper-bound property.

\pause
This thinking generalizes to more abstract settings.

\pause
\medskip

\textbf{Remark:}
Cauchy is stronger than
$\abs{x_{n+1}-x_n} \to 0$
as $n \to \infty$
%
\pause
(or $\abs{x_{n+j}-x_n} \to 0$ for a fixed $j$).

\pause
E.g., the partial sums of the harmonic series satisfy
$x_{n+1}-x_n = \nicefrac{1}{n}$,

but the sequence is not Cauchy.

\pause
In fact,
$\lim\limits_{n\to\infty} \abs{x_{n+j}-x_n} = 0$ for
every $j \in \N$.

\pause
The key point in the definition of Cauchy is that $n$ and $k$
vary independently and can be arbitrarily far apart.

\end{frame}

\end{document}
