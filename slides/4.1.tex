\documentclass[10pt,aspectratio=169]{beamer}

% All the boilerplate is in raslides.sty
% Note that this also pulls in a custom vogtwidebar.sty
\usepackage{raslides}

\author{Ji\v{r}\'i Lebl}

\institute[OSU]{%
Departemento pri Matematiko de Oklahoma {\^S}tata Universitato}

\title{BA: 4.1}

\date{}

\begin{document}

\begin{frame}
\titlepage
\end{frame}

\begin{frame}
\begin{definition}
Let $I$ be an interval, let
$f \colon I \to \R$ be a function, and let $c \in I$.
\pause
If the limit
\begin{equation*}
L \coloneqq \lim_{x \to c} \frac{f(x)-f(c)}{x-c} \qquad \text{exists,}
\end{equation*}
\pause
then $f$ is
\emph{differentiable} at
$c$, $L$ is the \emph{derivative} of $f$ at $c$,
and write $f'(c) \coloneqq L$.

\pause
\medskip

If $f$ is differentiable at all $c \in I$, say
$f$ is \emph{differentiable},
and obtain a function $f' \colon I \to \R$.

\pause
\medskip

Other notation: $\dfrac{df}{dx}$ or $\dfrac{d}{dx}\Bigl( f(x) \Bigr)$.

\pause
\medskip

The expression $\dfrac{f(x)-f(c)}{x-c}$ is called the
\emph{difference quotient}.
\end{definition}

\pause
\textbf{Remark:}
We allow $I$ to be a closed interval,
and we allow $c$ to be an endpoint of $I$.

\end{frame}

\begin{frame}
The graphical interpretation of the derivative:

\medskip

\begin{center}
\scalebox{0.9}{
\subimport*{../figures/}{deriv_derivd.pdf_t}
}
\end{center}

\pause
\medskip

$\dfrac{f(x)-f(c)}{x-c}$ is the slope of the \emph{secant line}
through $\bigl(c,f(c)\bigr)$ and $\bigl(x,f(x)\bigr)$.

\pause
\medskip

$f'(c)$ is the limit of these slopes as $x \to c$.

\end{frame}

\begin{frame}
\textbf{Example:}
Let $f \colon \R \to \R$ be $f(x) \coloneqq x^2$ and $c \in \R$.
\pause
If $x \not= c$, then
\begin{equation*}
\frac{x^2-c^2}{x-c}
\pause
=
\frac{(x+c)(x-c)}{x-c}
\pause
=
(x+c) .
\end{equation*}
\pause
Therefore,
\begin{equation*}
f'(c)
= 
\lim_{x\to c} \frac{x^2-c^2}{x-c}
\pause
=
\lim_{x\to c} (x+c)
\pause
= 2c.
\end{equation*}

\pause
\medskip

\textbf{Example:}
Let $f(x) \coloneqq ax + b$ for $a, b \in \R$.
\pause
Let $c \in \R$.
\pause
For $x \not=c$,
\begin{equation*}
\frac{f(x)-f(c)}{x-c}
\pause
=
\frac{a(x-c)}{x-c}
\pause
= a .
\end{equation*}
\pause
Therefore,
\begin{equation*}
f'(c)
=
\lim_{x\to c} 
\frac{f(x)-f(c)}{x-c}
\pause
=
\lim_{x\to c} 
a
= a.
\end{equation*}

\pause

\textbf{Remark:}
Every differentiable $f$ ``infinitesimally'' behaves like
the affine function $ax + b$.

\end{frame}

\begin{frame}

\textbf{Example:}
$f(x) \coloneqq \sqrt{x}$ is differentiable for $x > 0$.
\pause
\textbf{Proof:}
Fix $c > 0$ and take $x \not= c$, $x > 0$.
\pause
\begin{equation*}
\frac{\sqrt{x}-\sqrt{c}}{x-c}
\pause
=
\frac{\sqrt{x}-\sqrt{c}}{(\sqrt{x}-\sqrt{c})(\sqrt{x}+\sqrt{c})}
\pause
=
\frac{1}{\sqrt{x}+\sqrt{c}} .
\end{equation*}
\pause
Therefore,
\begin{equation*}
f'(c)
=
\lim_{x\to c}
\frac{\sqrt{x}-\sqrt{c}}{x-c}
\pause
=
\lim_{x\to c}
\frac{1}{\sqrt{x}+\sqrt{c}}
\pause
=
\frac{1}{2\sqrt{c}} .
\pushQED{\qed}
\qedhere
\popQED
\end{equation*}

\pause
\medskip

\textbf{Example:}
$f(x) \coloneqq \abs{x}$ is not differentiable at $0$.

\pause
\textbf{Proof:}
When $x > 0$,
\begin{equation*}
\frac{\abs{x}-\abs{0}}{x-0} =
\frac{x-0}{x-0} = 1 .
\end{equation*}
\pause
When $x < 0$,
\begin{equation*}
\frac{\abs{x}-\abs{0}}{x-0} =
\frac{-x-0}{x-0} = -1 .
\end{equation*}
\pause
So the limit $x \to 0$ does not exist. \qed

\end{frame}

\begin{frame}
A continuous function exists which is not differentiable at any point
(Weierstrass).
\pause
However,

\begin{proposition}
Let $f \colon I \to \R$ be differentiable at $c \in I$,
then it is continuous at $c$.
\end{proposition}

\pause
\textbf{Proof:}
\begin{equation*}
\lim_{x\to c}\frac{f(x)-f(c)}{x-c} = f'(c)
\qquad
\pause
\text{and}
\qquad
\lim_{x\to c}(x-c) = 0 .
\end{equation*}

\pause
\begin{equation*}
f(x)-f(c) = 
\left( \frac{f(x)-f(c)}{x-c} \right) (x-c) .
\end{equation*}

\pause
\thus \quad limit of $f(x)-f(c)$ as $x \to c$ exists
\pause
and
\begin{equation*}
\lim_{x\to c} \bigl( f(x)-f(c) \bigr)
\pause
=
\left(\lim_{x\to c} \frac{f(x)-f(c)}{x-c} \right)
\left(\lim_{x\to c} (x-c) \right)
\pause
=
f'(c) \cdot 0  = 0.
\end{equation*}
\pause
\thus \quad $\lim\limits_{x\to c} f(x) = f(c)$
\pause
\wthus $f$ is continuous at $c$.
\qed

\end{frame}

\begin{frame}

\begin{proposition}[Linearity]
Let $I$ be an interval, let
$f \colon I \to \R$ and $g \colon I \to \R$ be differentiable at $c \in I$,
and $\alpha \in \R$.%
\begin{enumerate}[(i)]
\item\pause
Define $h \colon I \to \R$ by $h(x) \coloneqq \alpha f(x)$.  Then
$h$ is differentiable at $c$ and
$h'(c) = \alpha f'(c)$.
\item\pause
Define $h \colon I \to \R$ by $h(x) \coloneqq  f(x) + g(x)$.  Then
$h$ is differentiable at $c$ and
$h'(c) =  f'(c) + g'(c)$.
\end{enumerate}
\end{proposition}

\pause
\textbf{Proof:}
(i) For $x \in I$, $x \not= c$,
\pause
\quad
$\displaystyle
\frac{h(x)-h(c)}{x-c}
\pause
=
\frac{\alpha f(x) - \alpha f(c)}{x-c}
\pause
=
\alpha \frac{f(x) - f(c)}{x-c}$.

\pause
\medskip

The limit as $x \to c$ of the RHS exists
\pause
and
\quad
$\displaystyle
\lim_{x\to c}\frac{h(x)-h(c)}{x-c}
\pause
=
\alpha \lim_{x\to c} \frac{f(x) - f(c)}{x-c}$.

\pause
\medskip

(ii)
For $x \in I$, $x \not= c$,

\pause
\medskip

\quad $\displaystyle
\frac{h(x)-h(c)}{x-c}
\pause
=
\frac{\bigl(f(x) + g(x)\bigr) - \bigl(f(c) + g(c)\bigr)}{x-c}
\pause
=
\frac{f(x) - f(c)}{x-c}
+
\frac{g(x) - g(c)}{x-c}$.

\pause
\medskip

The limit as $x \to c$ of RHS exists
\pause
and

\medskip
\quad $\displaystyle
\lim_{x\to c}\frac{h(x)-h(c)}{x-c}
\pause
=
\lim_{x\to c} \frac{f(x) - f(c)}{x-c}
+
\lim_{x\to c}\frac{g(x) - g(c)}{x-c}$.
\qed

\end{frame}

\begin{frame}

\begin{proposition}[Product rule / Leibniz rule]
Let $I$ be an interval, let
$f \colon I \to \R$ and $g \colon I \to \R$ be 
functions differentiable at $c$.  If $h \colon I \to \R$
is defined by
$h(x) \coloneqq f(x) g(x)$,
\pause
then $h$ is differentiable at $c$ and
\begin{equation*}
h'(c) = f(c) g'(c) + f'(c) g(c) .
\end{equation*}
\end{proposition}

\pause

\textbf{Proof:} Exercise.
\pause
Hint: 
$f(x) g(x) - f(c) g(c) =
f(x)\bigl( g(x) - g(c) \bigr)
+ \bigl( f(x) - f(c) \bigr) g(c)$

\pause

or \qquad
$\Delta (f \cdot g) = f \cdot \Delta g + \Delta f \cdot g$.

\pause
\begin{center}
\scalebox{0.9}{
\subimport*{../figures/}{figprodrule.pdf_t}
}
\end{center}

\end{frame}

\begin{frame}

\begin{proposition}[Quotient rule]
Let $I$ be an interval, let
$f \colon I \to \R$ and $g \colon I \to \R$ be differentiable at $c$
and $g(x) \not= 0$ for all $x \in I$.
If $h \colon I \to \R$
is defined by
$h(x) \coloneqq \frac{f(x)}{g(x)}$,
\pause
then $h$ is differentiable at $c$ and
\begin{equation*}
h'(c) = \frac{f'(c) g(c) - f(c) g'(c)}{{\bigl(g(c)\bigr)}^2} .
\end{equation*}
\end{proposition}

\pause

\textbf{Proof:}  Exercise.

\end{frame}

\begin{frame}

\begin{proposition}[Chain rule]
Let $I_1, I_2$ be intervals, let
$g \colon I_1 \to I_2$ be differentiable at $c \in I_1$,
and
$f \colon I_2 \to \R$ be differentiable at $g(c)$.
\pause
If $h \colon I_1 \to \R$
is defined by
$h(x) \coloneqq (f \circ g) (x) = f\bigl(g(x)\bigr)$,
\pause
then $h$ is differentiable at $c$ and
\begin{equation*}
h'(c) = f'\bigl(g(c)\bigr)g'(c) .
\end{equation*}
\end{proposition}

\pause
\textbf{Proof:}
Let $d \coloneqq g(c)$.

\pause
Define
$u \colon I_2 \to \R$ and $v \colon I_1 \to \R$ by
\begin{equation*}
u(y) \coloneqq
\begin{cases}
 \frac{f(y) - f(d)}{y-d}  & \text{if } y \not=d, \\
 f'(d)                    & \text{if } y = d,
\end{cases}
\pause
\qquad
v(x) \coloneqq
\begin{cases}
 \frac{g(x) - g(c)}{x-c} & \text{if } x \not=c, \\
 g'(c)                   & \text{if } x = c.
\end{cases}
\end{equation*}

\pause
$f$ differentiable at $d = g(c)$ \wthus $u$ is continuous at $d$.

\pause
$g$ differentiable at $c$ \wthus $v$ is continuous at $c$.

\pause
\medskip

$f(y)-f(d) = u(y) (y-d)$
\pause
\quad and \quad
$g(x)-g(c) = v(x) (x-c)
\qquad \forall x,y$.

\pause
\medskip

\thus \quad
$
h(x)-h(c)
\pause
=
f\bigl(g(x)\bigr)-f\bigl(g(c)\bigr)
\pause
=
u\bigl( g(x) \bigr) \bigl(g(x)-g(c)\bigr)
\pause
=
u\bigl( g(x) \bigr) \bigl(v(x) (x-c)\bigr)$.

\end{frame}

\begin{frame}
If $x \not= c$, then
\qquad
$\displaystyle \frac{h(x)-h(c)}{x-c}
\pause
=
u\bigl( g(x) \bigr) v(x)$.

\pause
\medskip

$u$ is continuous at $d$
\wthus
$\displaystyle \lim_{y \to d} u(y) = f'(d) = f'\bigl(g(c)\bigr)$.

\pause
\medskip

$v$ is continuous at $c$
\wthus
$\displaystyle \lim_{x \to c} v(x) = g'(c)$.

\pause
\medskip

$g$ is continuous at $c$
\wthus
$\displaystyle \lim_{x \to c} g(x) = g(c)$.

\pause
\medskip

\thus \quad
$\displaystyle\lim_{x\to c} u\bigl( g(x) \bigr) v(x)$ ~~ exists
\pause
and equals ~~
$f'\bigl(g(c)\bigr) g'(c)$.

\pause
\medskip

\thus \quad $h$ is differentiable at $c$
\pause
and $h'(c) = f'\bigl(g(c)\bigr)g'(c)$.
\qed

\end{frame}

\begin{frame}

\textbf{Exercise:}
Prove the following simple version of L'H\^opital's rule:

\pause
Suppose 
$f \colon (a,b) \to \R$ and $g \colon (a,b) \to \R$ are differentiable
functions
whose derivatives $f'$ and $g'$ are continuous functions.
\pause
Suppose that at $c \in (a,b)$, $f(c) = 0$, $g(c)=0$,
$g'(x) \not= 0$ for all $x \in (a,b)$, and
$g(x) \not= 0$ whenever $x \not= c$.

\pause
\medskip

Note
that the limit of $\nicefrac{f'(x)}{g'(x)}$ as $x$ goes to $c$ exists.

\pause
\medskip

Show that
\begin{equation*}
\lim_{x \to c} \frac{f(x)}{g(x)} = 
\lim_{x \to c} \frac{f'(x)}{g'(x)} .
\end{equation*}
\end{frame}

\end{document}
