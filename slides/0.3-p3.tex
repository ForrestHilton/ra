\documentclass[10pt,aspectratio=149]{beamer}

% All the boilerplate is in ac1slides.sty
% Note that this also pulls in a custom vogtwidebar.sty
\usepackage{ac1slides}

\author{Ji\v{r}\'i Lebl}

\institute[OSU]{%
Departemento pri Matematiko de Oklahoma {\^S}tata Universitato}

\title{BA: 0.3 part 3}

\date{}

\begin{document}

\begin{frame}
\titlepage
\end{frame}

\begin{frame}
Elements of sets can have relations.

E.g., $1 < 2$
for natural numbers, or $\nicefrac{1}{2} = \nicefrac{2}{4}$ for rational
numbers, or $\{ a,c \} \subset \{ a,b,c \}$ for sets.  The `$<$', `$=$', and
`$\subset$' are examples of \emph{relations}.

\pause

\begin{definition}
Given a set $A$, a \emph{binary relation} on $A$
is a subset $\sR \subset A \times A$,
which are those pairs where the relation is said to hold.
Instead of $(a,b) \in \sR$, we write
$a \, \sR \, b$.
\end{definition}

\pause
\textbf{Example:}
Take $A \coloneqq \{ 1,2,3 \}$.
Consider `$<$'.

\pause
\medskip

The corresponding set 
of pairs is $\bigl\{ (1,2), (1,3), (2,3) \bigr\}$.

\pause
\medskip

$1 < 2$ holds as $(1,2)$ is in the corresponding set of pairs,
\pause

$3 < 1$ does not hold as $(3,1)$ is not in the set.

\medskip
\pause

The relation `$=$'
is defined by $\bigl\{ (1,1), (2,2), (3,3) \big\}$.

\medskip
\pause

Any subset of $A \times A$ is a relation.
\pause

E.g., define the relation
$\dagger$ via $\bigl\{ (1,2), (2,1), (2,3), (3,1) \bigr\}$,

\pause
then $1 \dagger 2$ and
$3 \dagger 1$ are
true, but $1 \dagger 3$ is not.
\end{frame}

\begin{frame}

\begin{definition}
Let $\sR$ be a relation on a set $A$.
\pause
Then $\sR$ is
\begin{enumerate}[(i)]
\item
\emph{Reflexive} if $a\, \sR \, a$ for
all $a \in A$.
\pause
\item
\emph{Symmetric} if $a\, \sR \, b$ implies
$b \, \sR \, a$.
\pause
\item
\emph{Transitive} if $a\, \sR \, b$ and
$b \, \sR \, c$ implies $a\, \sR \, c$.
\end{enumerate}
\pause
If $\sR$ is reflexive, symmetric, and transitive, then it is
an \emph{equivalence relation}.
\end{definition}
\pause

\textbf{Example:} Consider $A \coloneqq \{ 1,2,3 \}$.

\medskip
\pause

Relation `$<$' given by
$\bigl\{ (1,2), (1,3), (2,3) \bigr\}$
is transitive, but neither reflexive nor symmetric.

\medskip
\pause

Relation `$\leq$' given by
$\bigl\{ (1,1), (1,2), (1,3), (2,2), (2,3), (3,3) \bigr\}$
is reflexive and transitive, but not symmetric.

\medskip
\pause

Relation `$=$' given by
$\bigl\{ (1,1), (2,2), (3,3) \bigr\}$ is
an equivalence relation.

\medskip
\pause

Relation `$\star$' given by
$\bigl\{ (1,1), (1,2), (2,1), (2,2), (3,3) \bigr\}$ is
an equivalence relation.

\end{frame}

\begin{frame}

\begin{definition}
Let $A$ be a set and $\sR$ an equivalence relation.
An \emph{equivalence class} of $a \in A$, denoted
by $[a]$, is the set $\{ x \in A : a\, \sR \, x \}$.
\end{definition}

\pause
\textbf{Example:}
For $A \coloneqq \{ 1,2,3 \}$ and
`$\star$' defined by
$\bigl\{ (1,1), (1,2), (2,1), (2,2), (3,3) \bigr\}$,
there are two equivalence classes,
$[1] = [2] = \{ 1,2 \}$ and $[3] = \{ 3 \}$.

\medskip
\pause

Reflexivity $\Rightarrow$ $a \in [a]$.

\pause
Symmetry $\Rightarrow$ if $b \in [a]$, then $a \in [b]$.

\pause
Transitivity $\Rightarrow$
if $a \in [b]$ and $b \in [c]$, then $a \in [c]$.

\pause

\begin{proposition}
If $\sR$ is an equivalence relation on a set $A$,
then every $a \in A$ is in exactly one
equivalence class.  Moreover, $a\, \sR \, b$ ~\iffif~ $[a] = [b]$.
\end{proposition}

\pause
Proof is an exercise.

\pause
\medskip

\textbf{Example:}
$\Q$ can be defined as the set of equivalence classes on
$\Z \times \N$ under the relation
$(a,b) \sim (c,d)$ when $ad = bc$ (exercise).
\pause
We write
$\bigl[(a,b)\bigr]$ as $\nicefrac{a}{b}$.

\end{frame}

\begin{frame}
What is the ``size'' of sets?

\pause

\begin{definition}
Let $A$ and $B$ be sets.  $A$ and $B$ have the same
\emph{cardinality}
if $\exists$ a bijection $f \colon A \to B$.
\pause
Denote by $\abs{A}$ the equivalence class of all sets with the same cardinality as
$A$ and we call $\abs{A}$ the cardinality of $A$.
\end{definition}

\pause
Existence of a bijection really is an equivalence relation:

\pause
$f \colon A \to A$, ~ $f(x) \coloneqq x$, ~ is a bijection \wthus reflexivity.

\pause
If $f \colon A \to B$ is a bijection, then so is $f^{-1} \colon B \to A$ \wthus symmetry.

\pause
If $f \colon A \to B$ and $g \colon B \to C$ are bijections, then
$g \circ f \colon A \to C$ is a bijection

\wthus transitivity.

\medskip
\pause

\textbf{Example:}
$\{ 1,2,3 \}$ has the same cardinality as $\{ a,b,c \}$.
\pause
Example bijection: $f(1) \coloneqq a$, $f(2) \coloneqq b$, $f(3) \coloneqq c$.
\pause  The bijection is \emph{not} unique.

\medskip
\pause

\textbf{Example:}
The set $E$ of even natural numbers has the same cardinality as $\N$.

\pause
Proof:
Given $k \in E$, write $k=2n$ for some $n \in \N$.
\pause
So
$f(n) \coloneqq 2n$ defines a bijection $f \colon \N \to E$.

\end{frame}

\begin{frame}

\begin{definition}
Write
$\abs{A} \leq \abs{B}$
if there exists an injection from $A$ to $B$.

\pause
Write $\abs{A} < \abs{B}$
if $\abs{A} \leq \abs{B}$, but
$\abs{A} \not= \abs{B}$.
\end{definition}

\pause
\textbf{Remark:}
It is not a trivial theorem (Cantor--Bernstein--Schr\"oder theorem) that
$\abs{A} = \abs{B}$ if and only if
$\abs{A} \leq \abs{B}$ and
$\abs{B} \leq \abs{A}$.

\medskip
\pause

\textbf{Remark:}
If $A$ and $B$ are any two sets,
we can always write $\abs{A} \leq \abs{B}$ or
$\abs{B} \leq \abs{A}$.  But this is \emph{very} subtle.

\pause
\medskip

$\abs{A} = \abs{\emptyset}$ $\Leftrightarrow$ $A = \emptyset$
\quad
If $B \not= \emptyset$,
no $f \colon B \to \emptyset$ can exist.

\pause
Write $\abs{\emptyset} = 0$.

\medskip
\pause

If $\abs{A} = \abs{\{ 1,2,3,\ldots,n \}}$ for $n \in \N$, write
$\abs{A} = n$ (exercise: such $n$ is unique).

\pause

\begin{definition}
$A$ is \emph{finite} if $\abs{A} \in \N$ or $\abs{A}=0$.

\pause
Otherwise $A$ is \emph{infinite}.

\pause
$A$ is \emph{countably infinite} if $\abs{A} = \abs{\N}$.
\qquad (often denoted $\abs{\N} = \aleph_0$)

\pause
$A$ is \emph{countable} if $A$ is finite or countably infinite.

\pause
If $A$ is not countable, then it is \emph{uncountable}.
\end{definition}
\end{frame}

\begin{frame}
%\emph{Remark:}
%A set is infinite if and only if it is in one-to-one
%correspondence with a proper subset of itself.
%
%\pause
%We skip the proof as this is again \emph{very} subtle and not necessary for us.
%
%\medskip
%\pause

\textbf{Example:}
$\abs{\N \times \N} = \abs{\N}$.

\medskip
\pause
Sketch of proof: Arrange 
$\N \times \N$ as
%$(1,1)$, $(1,2)$, $(2,1)$, $(1,3)$, $(2,2)$, $(3,1)$, \ldots.

%\medskip

\hspace*{1in}\subimport*{../figures/}{NcrossNcard.pdf_t}

%\medskip
\pause

Define a bijection by $f(1) \coloneqq (1,1)$, $f(2) \coloneqq (1,2)$, etc.

\medskip
\pause


%\end{frame}

%\begin{frame}

\textbf{Example:}
$\abs{\Q}=\abs{\N}$.

\medskip
\pause
Sketch of proof:
For positive $\Q$,
as for $\N \times \N$, write
$\nicefrac{1}{1}$, $\nicefrac{1}{2}$, $\nicefrac{2}{1}$, etc.,
then cross out any pair such as $\nicefrac{2}{2}$ that
has already appeared as a rational number.

\pause
For all $\Q$, also include $0$ and the negatives:
$0$,
$\nicefrac{1}{1}$,
$\nicefrac{-1}{1}$,
$\nicefrac{1}{2}$,
$\nicefrac{-1}{2}$,
$\nicefrac{2}{1}$,
$\nicefrac{-2}{1}$,
etc.

%\medskip
%\pause
%
%From the exercises:
%
%\pause
%
%\emph{If $A \subset
%B$ and $B$ is countable, then $A$ is countable.}
%
%\pause
%
%In other words, if $A \subset B$, and $A$ is
%uncountable, then $B$ is uncountable.
%
%\pause
%
%As a consequence, if $\abs{A} < \abs{\N}$, then $A$ is
%finite.
%
%\pause
%
%Also, \emph{if $B$ is finite and $A \subset B$, then $A$ is finite.}
%
\end{frame}

\begin{frame}

\begin{definition}
The \emph{power set} of a set $A$, denoted by $\sP(A)$,
is the set of all subsets of $A$.
\end{definition}

\pause

E.g., if $A \coloneqq \{ 1,2\}$, then $\sP(A) = \bigl\{ \emptyset, \{ 1 \}, \{ 2 \}, \{ 1, 2 \} \bigr\}$.

\pause
$\abs{A} = 2$ and $\abs{\sP(A)} = 4 = 2^2$.

\medskip
\pause

Exercise: If $\abs{A} = n \in \N$, then $\abs{\sP(A)} = 2^n$.

\medskip
\pause

So if $A$ is finite, $\abs{A} = n < 2^n = \abs{\sP(A)}$.

\medskip
\pause

Interestingly, it works for infinite sets too:

\begin{theorem}[Cantor]
$\abs{A} < \abs{\sP(A)}$ for any set $A$.
\end{theorem}

%\pause
%
%In particular, no surjection from
%$A$ onto $\sP(A)$ exists.

\end{frame}

\begin{frame}

\begin{theorem}[Cantor]
$\abs{A} < \abs{\sP(A)}$ for any set $A$.
\end{theorem}

\pause

\textbf{Proof:}
$f \colon A \to \sP(A)$ defined by
$f(x) \coloneqq \{ x \}$ is injective \pause $\Rightarrow$
$\abs{A} \leq \abs{\sP(A)}$.

\medskip
\pause

We must show no surjection exists.

\pause

Suppose $g \colon A \to \sP(A)$ is a function.
Define
\[
B \coloneqq \bigl\{ x \in A : x \notin g(x) \bigr\} .
\]
\pause
Claim: $B$ is not in $g(A)$.

\pause
Suppose for contradiction $g(x_0) = B$ for some $x_0 \in A$.

\pause
Either $x_0 \in B$ or $x_0 \notin B$.

\pause
$x_0 \in B$ \wthus $x_0 \notin g(x_0) = B$, \quad a contradiction.

\pause
$x_0 \notin B$ \wthus $x_0 \in g(x_0) = B$, \quad a contradiction.

\pause
\thus ~
No such $x_0$ exists
\pause
\wthus $g$ is not a surjection
\pause
\wthus no surjection exists.
\qed

\medskip
\pause

\emph{Consequence:} Uncountable sets exist, e.g., $\sP(\N)$.

\medskip
\pause

In fact,
$\abs{\N} < \abs{\sP(\N)} < \abs{\sP(\sP(\N))} < \abs{\sP(\sP(\sP(\N)))}$, etc.

\end{frame}

\end{document}
