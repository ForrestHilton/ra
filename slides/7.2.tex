\documentclass[10pt,aspectratio=169]{beamer}

% All the boilerplate is in raslides.sty
% Note that this also pulls in a custom vogtwidebar.sty
\usepackage{raslides}

\author{Ji\v{r}\'i Lebl}

\institute[OSU]{%
Departemento pri Matematiko de Oklahoma {\^S}tata Universitato}

\title{BA: 7.2}

\date{}

\begin{document}

\begin{frame}
\titlepage
\end{frame}

\begin{frame}
\begin{definition}
Let $(X,d)$ be a metric space, $x \in X$, and $\delta > 0$.
\pause
Define
the \emph{open ball}, or \emph{ball}, of radius $\delta$
around $x$ as
\begin{equation*}
B(x,\delta) \coloneqq \bigl\{ y \in X : d(x,y) < \delta \bigr\} .
\end{equation*}
\pause
Define the \emph{closed ball} as
\begin{equation*}
C(x,\delta) \coloneqq \bigl\{ y \in X : d(x,y) \leq \delta \bigr\} .
\end{equation*}
\end{definition}
\pause

Write $B_X(x,\delta) \coloneqq B(x,\delta)$ or $C_X(x,\delta) \coloneqq C(x,\delta)$,
to emphasize the space.

\pause
\medskip

\textbf{Example:}
Take $\R$ with the standard metric:
\[
B(x,\delta) = (x-\delta,x+\delta) \qquad \text{and} \qquad
C(x,\delta) = [x-\delta,x+\delta] .
\]

\pause
\medskip

\textbf{Example:}
Consider $[0,1]$ as a subspace of $\R$.
In $[0,1]$,
\[
B(0,\nicefrac{1}{2}) = B_{[0,1]}(0,\nicefrac{1}{2})
\pause
= \bigl\{ y \in [0,1] : \abs{0-y} < \nicefrac{1}{2} \bigr\}
\pause
= [0,\nicefrac{1}{2}) .
\]
\pause
Note the difference with $B_{\R}(0,\nicefrac{1}{2}) =
(\nicefrac{-1}{2},\nicefrac{1}{2})$.

\end{frame}

\begin{frame}

\begin{definition}
Let $(X,d)$ be a metric space.

\pause
\medskip

$V \subset X$
is \emph{open}
if $\forall$ $x \in V$, $\exists$ $\delta > 0$ such that
$B(x,\delta) \subset V$.

\pause
\medskip

$E \subset X$ is 
\emph{closed} if $E^c = X \setminus E$ is open.

\pause
\medskip

When $X$ is not clear from context,
we say \emph{$V$ is open in $X$} and \emph{$E$ is closed in $X$}.

\pause
\medskip

If $x \in V$ and $V$ is open, then
$V$ is an \emph{open neighborhood} of $x$ (or
just \emph{neighborhood}).
\end{definition}

\pause
\medskip

\hspace*{2.3in}\scalebox{0.7}{
\subimport*{../figures/}{msopenset.pdf_t}
}

\pause
\vspace*{-0.9in}

\textbf{Remark:} $\delta$ depends on $x$.

\pause
\medskip

An open set $V$ is a set that does not

include its ``boundary.''

\pause
\medskip

If we are in $V$ we are allowed to

``wiggle'' and we'll stay in $V$.

\pause
\medskip

$E$ is closed if everything not in $E$
is some distance away from $E$.

\end{frame}

\begin{frame}

\textbf{Examples:}

\pause
\medskip

$(0,\infty) \subset \R$ is open:

\pause
Given $x \in (0,\infty)$,
let $\delta \coloneqq x$

\pause
\thus \quad
 $B(x,\delta) = (0,2x) \subset (0,\infty)$.

\pause
\medskip

$[0,\infty) \subset \R$ is closed:

\pause
Given $x \in (-\infty,0) =
[0,\infty)^c$,
let $\delta \coloneqq -x$

\pause
\thus \quad
 $B(x,\delta) = (-2x,0) \subset
(-\infty,0) = [0,\infty)^c$.

\pause
\medskip

$[0,1) \subset \R$ is neither open nor closed:

\pause
$\forall$ $\delta > 0$, $B(0,\delta) = (-\delta,\delta)$, contains numbers
not in $[0,1)$
\pause
\hfill \thus \hfill $[0,1)$ is not open.

\pause
$\forall$ $\delta > 0$,
$B(1,\delta) = (1-\delta,1+\delta)$, contains
numbers in $[0,1)$,

\pause
\thus \quad $[0,1)^c$ is not open \wthus $[0,1)$ is not closed.

\end{frame}

\begin{frame}
\begin{proposition}
Let $(X,d)$ be a metric space.
\begin{enumerate}[(i)]
\item
\pause
\label{topology:openi} $\emptyset$ and $X$ are open.
\item
\pause
\label{topology:openii} If $V_1, V_2, \ldots, V_k$ are open subsets of $X$, then
\begin{equation*}
\bigcap_{j=1}^k V_j
\end{equation*}
is also open.  That is, a finite intersection of open sets is open.
\item
\pause
\label{topology:openiii} If $\{ V_\lambda \}_{\lambda \in I}$ is
an arbitrary collection of open subsets of $X$, then
\begin{equation*}
\bigcup_{\lambda \in I} V_\lambda
\end{equation*}
is also open.  That is, a union of open sets is open.
\end{enumerate}
\end{proposition}

%The index set $I$ in \eqref{topology:openiii} can be arbitrarily large.
%By $\bigcup_{\lambda \in I} V_\lambda$, we simply mean the set of
%all $x$ such that $x \in V_\lambda$ for at least one $\lambda \in I$.

\end{frame}

\begin{frame}

\textbf{Proof:}
\eqref{topology:openi}:
$\emptyset$ and $X$ are obviously open in $X$.

\pause
\medskip

\eqref{topology:openii}:
If $x \in \bigcap_{j=1}^k V_j$, then $x \in V_j$ for all $j$.

\pause
$V_j$ are all open
\pause
\wthus $\forall$ $j$ $\exists$ $\delta_j > 0$ 
such that $B(x,\delta_j) \subset V_j$.

\pause
Take $\delta \coloneqq \min \{ \delta_1,\delta_2,\ldots,\delta_k \}$.

\pause
Note $\delta > 0$.

\pause
\thus \quad $B(x,\delta) \subset B(x,\delta_j) \subset V_j$ for every $j$ and so
$B(x,\delta) \subset \bigcap_{j=1}^k V_j$.

\pause
\thus \quad The intersection is open.

\pause
\medskip

\eqref{topology:openiii}:
If $x \in \bigcup_{\lambda \in I} V_\lambda$, then $x \in V_\lambda$ for some
$\lambda \in I$.


\pause
$V_\lambda$ is open
\pause
\wthus $\exists$ $\delta > 0$
such that $B(x,\delta) \subset V_\lambda$.

\pause
\thus \quad
$B(x,\delta) \subset \bigcup_{\lambda \in I} V_\lambda$

\pause
\thus \quad The union is open.
\qed

\pause
\medskip

\textbf{Example:}
\eqref{topology:openii} is not true for an arbitrary intersection:

\pause
$\bigcap_{n=1}^\infty (\nicefrac{-1}{n},\nicefrac{1}{n}) = \{ 0 \}$,
which is not open.

\end{frame}

\begin{frame}
\begin{proposition}
Let $(X,d)$ be a metric space.
\begin{enumerate}[(i)]
\item
\pause
\label{topology:closedi} $\emptyset$ and $X$ are closed.
\item
\pause
\label{topology:closedii} If $\{ E_\lambda \}_{\lambda \in I}$ is
an arbitrary collection of closed subsets of $X$, then
\begin{equation*}
\bigcap_{\lambda \in I} E_\lambda
\end{equation*}
is also closed.  That is, an intersection of closed sets is closed.
\item
\pause
\label{topology:closediii} If $E_1, E_2, \ldots, E_k$ are closed
subsets of $X$, then
\begin{equation*}
\bigcup_{j=1}^k E_j
\end{equation*}
is also closed.  That is, a finite union of closed sets is closed.
\end{enumerate}
\end{proposition}

\pause
\textbf{Proof:} Exercise.

\end{frame}

\begin{frame}

\begin{proposition}
Let $(X,d)$ be a metric space, $x \in X$, and $\delta > 0$.

\pause
Then
$B(x,\delta)$ is open and 
$C(x,\delta)$ is closed.
\end{proposition}

\pause
\textbf{Proof:}
Let $y \in B(x,\delta)$.
\pause
Let $\alpha \coloneqq \delta-d(x,y) > 0$.
\pause
Consider $z \in B(y,\alpha)$:

\pause
\medskip

$
d(x,z) \leq d(x,y) + d(y,z)
$

\pause
\medskip
\quad
$ < d(x,y) + \alpha$

\pause
\medskip
\quad
$= d(x,y) + \delta-d(x,y) = \delta$.

\pause
\vspace*{-0.7in}
\hspace*{2.9in}\scalebox{0.8}{
\subimport*{../figures/}{ballisopen.pdf_t}
}

\vspace*{-0.3in}

\pause
\thus \quad
$B(y,\alpha) \subset B(x,\delta)$ \wthus
$B(x,\delta)$ is open.


\pause
\medskip

That $C(x,\delta)$ is closed is an exercise.
\qed

\pause
\medskip

\textbf{Careful:}
$[0,\nicefrac{1}{2})$ is
an open ball in $[0,1]$, and so $[0,\nicefrac{1}{2})$ is
an open set in $[0,1]$.

\pause
$[0,\nicefrac{1}{2})$ is neither open nor closed in $\R$.

\end{frame}

\begin{frame}

\begin{proposition}
Let $a, b$ be two real numbers, $a < b$.
\pause
Then $(a,b)$, $(a,\infty)$,
and $(-\infty,b)$ are open in $\R$.
\pause
Also $[a,b]$, $[a,\infty)$,
and $(-\infty,b]$ are closed in $\R$.
\end{proposition}

\pause
\textbf{Proof:} Exercise.

\pause
\begin{proposition}
Suppose $(X,d)$ is a metric space, and $Y \subset X$.
\pause
Then $U \subset Y$
is open in $Y$ (in the subspace topology) \wiffif
$\exists$ open set $V \subset X$ (so open in $X$), such that
$V \cap Y = U$.
\end{proposition}

\pause
\textbf{Example:}
Let $X \coloneqq \R$, $Y \coloneqq [0,1]$, $U \coloneqq [0,\nicefrac{1}{2})$.

\pause
$U$ is an open set in $Y$, may take $V \coloneqq (\nicefrac{-1}{2},\nicefrac{1}{2})$.

\pause
\medskip

\textbf{Proof of proposition:}
Suppose $V \subset X$ is open
\pause
and $V \cap Y = U$.

\pause
Suppose $x \in U = V \cap Y$.
\pause
\quad
$\exists$ $\delta > 0$ such that $B_X(x,\delta) \subset V$.

\pause
\medskip

Then
$
B_Y(x,\delta)
\pause
= B_X(x,\delta) \cap Y
\pause
\subset V \cap Y
\pause
= U$.

\pause
\medskip

The opposite direction is an exercise.
\qed

%A hint for finishing the proof (the exercise) is that
%a useful way to think about an open set is as a union of open balls.  If $U$ is
%open, then for each $x \in U$, there is a $\delta_x > 0$ (depending on $x$) such that
%$B(x,\delta_x) \subset U$.  Then $U = \bigcup_{x\in U} B(x,\delta_x)$.

\end{frame}

\begin{frame}

\begin{proposition}
Suppose $(X,d)$ is a metric space, $V \subset X$ is open,
and $E \subset X$ is closed.
\begin{enumerate}[(i)]
\item
\pause
\label{prop:topology:subspacesame:i}
$U \subset V$ is open in the subspace topology \wiffif $U$ is open
in $X$.
\item
\pause
\label{prop:topology:subspacesame:ii}
$F \subset E$ is closed in the subspace topology \wiffif $F$ is
closed in $X$.
\end{enumerate}
\end{proposition}

\pause

\textbf{Proof:}
\eqref{prop:topology:subspacesame:i}:
Suppose $U \subset V$ is open in the subspace topology.

\pause
\medskip

By the previous proposition, $\exists$ $W \subset X$,
open in $X$, such that $U = W \cap V$

\pause
\medskip

Intersection of open sets is open \wthus $U$ is open in $X$.

\pause
\medskip

Now suppose $U$ is open in $X$.

\pause
\medskip

\thus \quad $U = U \cap V$ \wthus
$U$ is open in $V$ (the proposition again)

\pause
\medskip

\eqref{prop:topology:subspacesame:ii}: Exercise.
\qed

\end{frame}

\begin{frame}

\begin{definition}
A nonempty
metric space $(X,d)$ is \emph{connected} if the
only subsets of $X$ that are both open and closed (so-called
\emph{clopen} subsets) are $\emptyset$ and $X$ itself.

\pause
\medskip

If a nonempty $(X,d)$ is not connected we say it is
\emph{disconnected}.

\pause
\medskip

A nonempty subset $A \subset X$ is \emph{connected} if
$A$ is connected with the subspace topology.
\end{definition}

\pause
A nonempty $X$ is connected if
$X = X_1 \cup X_2$ where $X_1 \cap X_2 = \emptyset$ and $X_1$ and $X_2$ are
open, then either $X_1 = \emptyset$ or $X_2 = \emptyset$.

\pause
\medskip

To show $X$ is disconnected, find nonempty
disjoint open sets $X_1$ and
$X_2$ whose union is $X$.

\pause
\medskip

\textbf{Example:}
$B(x,\delta)$ is not always connected.

\pause
Consider $\{ a, b\}$ with the discrete metric.

\pause
\thus \quad $B(a,2) = \{ a , b \}$, which is not
connected:

\pause
$B(a,1) = \{ a \}$ and 
$B(b,1) = \{ b \}$ are open and disjoint.

\end{frame}

\begin{frame}

\begin{proposition}
Let $(X,d)$ be a metric space.
\pause
A nonempty $S \subset X$ is disconnected \wiffif
$\exists$ open sets $U_1$ and
$U_2$ in $X$, such that $U_1 \cap U_2 \cap S = \emptyset$,
$U_1 \cap S \not= \emptyset$,
$U_2 \cap S \not= \emptyset$, and
\vspace*{-6pt}
\begin{equation*}
S = 
\bigl( U_1 \cap S \bigr)
\cup
\bigl( U_2 \cap S \bigr) .
\end{equation*}
\end{proposition}

\pause
\begin{center}
\scalebox{0.7}{
\subimport*{../figures/}{disconsubset.pdf_t}
}
\end{center}

\pause
\textbf{Proof:}
\thus)
Suppose $S$ is disconnected:
\pause
$\exists$
nonempty disjoint $S_1$ and $S_2$ that are
open in $S$ and $S = S_1 \cup S_2$.

\pause
\thus~ $\exists$ $U_1$ and $U_2$
that are open in $X$ such that $U_1 \cap S = S_1$ and
$U_2 \cap S = S_2$.

\pause
$\Leftarrow$)
Suppose $U_1$ and $U_2$ exist.
\pause
~\thus~ $U_1 \cap S$ and $U_2 \cap S$ are open in $S$.

\pause
\thus~ $S$ is disconnected.
\qed

\pause
\medskip

\textbf{Example:}
$S \subset \R$ and $\exists$ $x,y,z$, s.t.  $x < z < y$ with $x,y \in S$
and $z \notin S$.

\pause
Claim: \emph{$S$ is disconnected}.

\pause
Proof:
~$
\bigl( (-\infty,z) \cap S \bigr)
\cup
\bigl( (z,\infty) \cap S \bigr)
= S$

\end{frame}

\begin{frame}

\begin{proposition}
A nonempty set $S \subset \R$ is connected \wiffif $S$ is
an interval or a single point.
\end{proposition}

\pause
\textbf{Proof:}
Suppose $S$ is connected.

\pause
\medskip

If $S$ is a single point, then we are done.

\pause
\medskip

Suppose $x < y$ and $x,y \in S$.

\pause
Suppose $z \in \R$ is such that $x < z < y$.

\pause
$(-\infty,z) \cap S \not= \emptyset$ and $(z,\infty) \cap S \not=
\emptyset$.

\pause
The sets are disjoint.

\pause
As $S$ is connected, their union can't be $S$.

\pause
\thus \quad $z \in S$.

\pause
\thus \quad $S$ is an interval.

\end{frame}

\begin{frame}

If $S$ is a single point, it is connected,
\pause
\quad
so suppose $S$ is an interval.

\pause
\medskip

Consider $U_1$ and $U_2$ open in $\R$, such that

\pause
$U_1 \cap S \not= \emptyset$ and $U_2 \cap S \not= \emptyset$ and $S =
\bigl( U_1 \cap S \bigr) \cup \bigl( U_2 \cap S \bigr)$.

\pause
%We will show that $U_1 \cap S$
%and $U_2 \cap S$ contain a common point, so they are not disjoint,
%proving that $S$ is connected.
Suppose $x \in U_1 \cap S$ and $y \in U_2 \cap S$, WLOG $x < y$.

\pause
$S$ is an interval \wthus $[x,y] \subset S$.

\pause
$U_2 \cap [x,y] \not= \emptyset$
\pause
\quad
let $z \coloneqq \inf (U_2 \cap [x,y])$.
\pause
\quad
WTS $z \in U_1$.

\pause
If $z = x$, then $z \in U_1$.

\pause
If $z > x$, then $\forall$ $\epsilon > 0$,
$B(z,\epsilon)$ has pts of $[x,y] \setminus U_2$,
as $z = \inf (U_2 \cap [x,y])$.

\pause
So $z \notin U_2$ as $U_2$ is open.

\pause
\thus \quad $z \in U_1$  ~~($U_1$ and $U_2$ cover $[x,y]$)

\pause
$U_1$ is open \wthus $B(z,\delta) \subset U_1$ for some $\delta > 0$.

\pause
$z = \inf (U_2 \cap [x,y])$
\pause
\wthus $\exists$
$w \in U_2 \cap [x,y]$ s.t. $w \in [z,z+\delta) \subset B(z,\delta) \subset U_1$.

\pause
\begin{center}
\scalebox{0.8}{
\subimport*{../figures/}{intervalcon.pdf_t}
}
\end{center}

\pause
\thus\quad
$w \in U_1 \cap U_2$
\pause
\wthus
$(U_1 \cap S) \cap (U_2 \cap S) \not= \emptyset$
\pause
\wthus
$S$ is connected.
\qed

\end{frame}

\begin{frame}

\begin{definition}
Let $(X,d)$ be a metric space and $A \subset X$.
\pause
The \emph{closure} of $A$ is the set
\begin{equation*}
\widebar{A} \coloneqq
\bigcap \{ E \subset X : E \text{ is closed and } A \subset E \} .
\end{equation*}
\end{definition}

\pause
\begin{proposition}
Let $(X,d)$ be a metric space and $A \subset X$.
\pause
$\widebar{A}$ is closed,
\pause
and $A \subset \widebar{A}$.

\pause
Furthermore, if $A$ is closed, then $\widebar{A} = A$.
\end{proposition}

\pause
\textbf{Proof:}
$\widebar{A}$ an intersection of closed sets
\pause
\wthus $\widebar{A}$ is closed.

\pause
\medskip

$\exists$ at least one closed set containing $A$, namely $X$ itself,
\pause
\wthus
$A \subset \widebar{A}$.

\pause
\medskip

If $A$ is closed, then $A$ is a closed set that contains $A$.

\pause
\thus \quad $\widebar{A} \subset A$
\pause
\wthus $A = \widebar{A}$.
\qed

\end{frame}

\begin{frame}

\textbf{Example:}
The closure of $(0,1)$ in $\R$ is $[0,1]$.

\pause
\medskip

Proof:  If $E$ is closed and $(0,1) \subset E$, then $0,1 \in E$ (why?).
\pause
\wthus $[0,1] \subset E$.

\pause
$[0,1]$ is closed
\pause
\wthus
$\overline{(0,1)} = [0,1]$.

\pause
\medskip

\textbf{Example:}
Ambient space matters.

\pause
\medskip

If $X = (0,\infty)$, then
the closure of $(0,1)$ in $(0,\infty)$ is $(0,1]$.

\pause
\medskip

$(0,1]$ is closed in $(0,\infty)$ (why?).

\pause
Any closed set $E$ s.t. $(0,1) \subset E$ must contain $1$.

\pause
\thus \quad $(0,1] \subset E$
\pause
\wthus 
$\overline{(0,1)} = (0,1]$ in $(0,\infty)$.

\end{frame}

\begin{frame}

\begin{proposition}
Let $(X,d)$ be a metric space and $A \subset X$.
\pause
\quad
$x \in \widebar{A}$ \wiffif
$\forall$ $\delta > 0$, $B(x,\delta) \cap A \not=\emptyset$.
\end{proposition}

\pause
\textbf{Proof:}
We will show
\pause
\quad
$x \notin \widebar{A}$ \wiffif $\exists$
$\delta > 0$ such that $B(x,\delta) \cap A = \emptyset$.

\pause
\medskip

Suppose $x \notin \widebar{A}$.

\pause
$\widebar{A}$ is closed \wthus $\exists$ $\delta > 0$ such that
$B(x,\delta) \subset \widebar{A}^c$.

\pause
As $A \subset \widebar{A}$,~~
$B(x,\delta) \subset \widebar{A}^c \subset A^c$
\pause
\wthus
$B(x,\delta) \cap A = \emptyset$.

\pause
\medskip

Now suppose $\exists$ $\delta > 0$, such that
$B(x,\delta) \cap A = \emptyset$. 

\pause
\thus\quad
$A \subset {B(x,\delta)}^c$.

\pause
${B(x,\delta)}^c$ is closed,
\pause
\hfill $x \not \in {B(x,\delta)}^c$, 
\pause
\hfill $\widebar{A}$ is the intersection
of closed sets containing $A$

\pause
\thus \quad $x \notin \widebar{A}$.
\qed

\end{frame}

\begin{frame}

\begin{definition}
Let $(X,d)$ be a metric space and $A \subset X$.
The \emph{interior} of $A$ is the set
\begin{equation*}
A^\circ \coloneqq \{ x \in A : \text{there exists a } \delta > 0
\text{ such that } B(x,\delta) \subset A \} .
\end{equation*}
\pause
The \emph{boundary} of $A$ is the set
\begin{equation*}
\partial A \coloneqq \widebar{A}\setminus A^\circ.
\end{equation*}
\end{definition}

\pause
\textbf{Example:}
Suppose $A \coloneqq (0,1]$ and $X \coloneqq \R$.

\pause
\medskip

\thus \quad
$\widebar{A}=[0,1]$, $A^\circ = (0,1)$,
and $\partial A = \{ 0, 1 \}$.

\pause
\medskip

\textbf{Example:}
Consider $X \coloneqq \{ a, b \}$ with the discrete metric,
and $A \coloneqq \{ a \}$.

\pause
\medskip

\thus \quad $\widebar{A} = A^\circ = A$ and $\partial A = \emptyset$.

\end{frame}

\begin{frame}

\begin{proposition}
Let $(X,d)$ be a metric space and $A \subset X$.
\pause
Then $A^\circ$ is open
and $\partial A$ is closed.
\end{proposition}

\pause
\textbf{Proof:}
Given $x \in A^\circ$, $\exists$ $\delta > 0$ such that $B(x,\delta) \subset A$.

\pause
\medskip

If $z \in B(x,\delta)$, $\exists$ $\epsilon > 0$ such that $B(z,\epsilon) \subset B(x,\delta)
\subset A$.

\pause
\thus \quad $z \in A^\circ$
\pause
\wthus $B(x,\delta) \subset A^\circ$
\pause
\wthus $A^\circ$ is open.

\pause
\medskip

$A^\circ$ is open
\wthus
$\partial A = \widebar{A} \setminus A^\circ = \widebar{A} \cap
{(A^\circ)}^c$ is closed.
\qed

\end{frame}

\begin{frame}

\begin{proposition}
Let $(X,d)$ be a metric space and $A \subset X$.

\pause
$x \in \partial A$
\wiffif
$\forall$ $\delta > 0$,
$B(x,\delta) \cap A \not=\emptyset$ and
$B(x,\delta) \cap A^c \not=\emptyset$.
\end{proposition}

\pause
\hspace*{2.3in}\scalebox{0.65}{
\subimport*{../figures/}{msboundary.pdf_t}
}

\vspace*{-1.0in}

\pause
\textbf{Proof:}
Suppose $x \in \partial A =  \widebar{A} \setminus A^\circ$ and

\pause
$\delta > 0$ is arbitrary.

\pause
$x \in \widebar{A}$ \wthus
$B(x,\delta) \cap A \not= \emptyset$.

\pause
If $B(x,\delta) \cap A^c = \emptyset$, then $x \in A^\circ$

\pause
as $x \notin A^\circ$
\wthus $B(x,\delta) \cap A^c \not= \emptyset$.

\pause
\medskip

Now suppose $x \notin \partial A$
\pause
\wthus
$x \notin \widebar{A}$ or $x \in A^\circ$.

\pause
\medskip

If $x \notin \widebar{A}$, then $B(x,\delta) \subset \widebar{A}^c$
for some $\delta > 0$ as $\widebar{A}$ is closed.

\pause
\thus \quad $B(x,\delta) \cap A = \emptyset$ ~~as $\widebar{A}^c \subset A^c$.

\pause
\medskip

If $x \in A^\circ$, then
$B(x,\delta) \subset A$ for some $\delta > 0$,
so $B(x,\delta) \cap A^c = \emptyset$.
\qed

\pause
\begin{corollary}
Let $(X,d)$ be a metric space and $A \subset X$.
Then $\partial A = \widebar{A} \cap \widebar{A^c}$.
\end{corollary}

\end{frame}

\begin{frame}

\textbf{Exercise:}
Prove that in a metric space,
\begin{enumerate}[a)]
\item
\pause
$E$ is closed if and only if $\partial E \subset E$.
\item
\pause
$U$ is open if and only if $\partial U \cap U = \emptyset$.
\end{enumerate}

\pause
\medskip

\textbf{Exercise:}
Prove that in a metric space,
\begin{enumerate}[a)]
\pause
\item
$A$ is open if and only if $A^\circ = A$.
\pause
\item
$U \subset A^\circ$
for every open set $U$ such that $U \subset A$.
\end{enumerate}

\pause
\medskip

\textbf{Exercise:}
Let $(X,d)$ be a metric space and $A \subset X$.  Show that
$A^\circ = \bigcup \{ V : V \text{ is open and } V \subset A \}$.

\end{frame}

\end{document}
