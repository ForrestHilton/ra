\documentclass[10pt,aspectratio=149]{beamer}

% All the boilerplate is in ac1slides.sty
% Note that this also pulls in a custom vogtwidebar.sty
\usepackage{ac1slides}

\author{Ji\v{r}\'i Lebl}

\institute[OSU]{%
Departemento pri Matematiko de Oklahoma {\^S}tata Universitato}

\title{BA: 2.2}

\date{}

\begin{document}

\begin{frame}
\titlepage
\end{frame}

\begin{frame}
\begin{lemma}[Squeeze lemma]
Let $\{ a_n \}_{n=1}^\infty$, 
$\{ b_n \}_{n=1}^\infty$, and 
$\{ x_n \}_{n=1}^\infty$ be sequences such that
\quad
$a_n \leq x_n \leq b_n$ \quad $\forall$ $n \in \N$.
\pause

Suppose $\{ a_n \}_{n=1}^\infty$ and $\{ b_n \}_{n=1}^\infty$ converge and
\quad
$\displaystyle
\lim_{n\to \infty} a_n
=
\lim_{n\to \infty} b_n$.

\pause
Then $\{ x_n \}_{n=1}^\infty$ converges and
\quad
$\displaystyle
\lim_{n\to \infty} x_n
=
\lim_{n\to \infty} a_n
=
\lim_{n\to \infty} b_n$.
\end{lemma}

\pause
\textbf{Proof:}
Let $\displaystyle x \coloneqq \lim_{n\to\infty} a_n = \lim_{n\to\infty} b_n$.
\pause
Let $\epsilon > 0$ be given.

\pause
Find $M_1$ s.t. $\forall$ $n \geq M_1$, ~
$\abs{a_n-x} < \epsilon$,
\pause
and $M_2$  s.t. $\forall$ $n \geq M_2$,
~ $\abs{b_n-x} < \epsilon$.

\pause
Set $M \coloneqq \max \{M_1, M_2 \}$.

\pause
$n \geq M$
\wthus
$x - a_n < \epsilon$ ~or~ $x - \epsilon < a_n$ \quad
\pause
 and \quad  $b_n < x + \epsilon$.

\pause
\thus \quad
$x - \epsilon < a_n \leq x_n \leq b_n < x + \epsilon$.

\pause
\thus \quad $-\epsilon < x_n-x < \epsilon$
\pause
\wthus $\abs{x_n-x} < \epsilon$
\pause
\wthus
$\{x_n\}_{n=1}^\infty$ converges to $x$.
\qed

\pause
\medskip

\begin{center}
\scalebox{0.9}{
\subimport*{../figures/}{figsqueeze.pdf_t}
}
\end{center}

\end{frame}

\begin{frame}

\textbf{Example:}
Consider $\bigl\{ \frac{1}{n\sqrt{n}} \bigr\}_{n=1}^\infty$.

\pause
\medskip

As $\sqrt{n} \geq 1$ for all $n \in \N$,
\begin{equation*}
0 \leq \frac{1}{n\sqrt{n}} \leq \frac{1}{n}
\quad \text{for all } n \in \N.
\end{equation*}

\pause

We already know $\displaystyle \lim_{n \to \infty} \nicefrac{1}{n} = 0$. 

\medskip
\pause

The constant sequence $\{ 0 \}_{n=1}^\infty$ and $\{ \nicefrac{1}{n}
\}_{n=1}^\infty$ in the
squeeze lemma \quad \thus
\begin{equation*}
\lim_{n\to\infty} \frac{1}{n\sqrt{n}} = 0 .
\end{equation*}

\end{frame}

\begin{frame}

\begin{lemma}
Let $\{ x_n \}_{n=1}^\infty$ and $\{ y_n \}_{n=1}^\infty$ be
convergent sequences and
\quad $x_n \leq y_n$
\quad
for all $n \in \N$.

\pause
Then \quad 
$\displaystyle \lim_{n\to\infty} x_n \leq \lim_{n\to\infty} y_n$.
\end{lemma}

\pause
\textbf{Proof:}
Let $\displaystyle x \coloneqq \lim_{n\to\infty} x_n$
and $\displaystyle y \coloneqq \lim_{n\to\infty} y_n$. 

\medskip
\pause

Let 
$\epsilon > 0$ be given.

\medskip
\pause

Find an $M_1$ such that $\abs{x_n-x} < \nicefrac{\epsilon}{2}$ for all $n \geq M_1$,

\pause
and $M_2$ such that $\abs{y_n-y} < \nicefrac{\epsilon}{2}$
for all $n \geq M_2$.

\pause
\medskip

So for some $n \geq \max\{ M_1, M_2 \}$, \quad
$x-x_n < \nicefrac{\epsilon}{2}$ ~and~
$y_n-y < \nicefrac{\epsilon}{2}$.

\pause
\medskip

Add these to get
\quad $y_n-x_n+x-y < \epsilon$
\pause
\quad or \quad $y_n-x_n < y-x+ \epsilon$.

\pause
\medskip

$x_n \leq y_n$ $\forall n$
\pause
\wthus $0 \leq y_n-x_n$ $\forall n$
\pause
\wthus $0 < y-x+ \epsilon$
\pause
\wthus $x-y < \epsilon$.

\pause
\medskip

$\epsilon > 0$ is arbitrary
\pause
\wthus $x-y \leq 0$
\pause
\wthus $x \leq y$.
\qed

\end{frame}

\begin{frame}

\begin{corollary}
\begin{enumerate}[(i)]
\item If $\{ x_n \}_{n=1}^\infty$ is convergent
such that $x_n \geq 0$ $\forall$ $n \in \N$,
then
\quad
$\displaystyle
\lim_{n\to\infty} x_n \geq 0$.
\item\pause
Let $a,b \in \R$ and
$\{ x_n \}_{n=1}^\infty$ be convergent such that
$a \leq x_n \leq b$
$\forall$ $n \in \N$.

Then
$\displaystyle a \leq \lim_{n\to\infty} x_n \leq b$.
\end{enumerate}
\end{corollary}

\pause

Proof is an exercise.

\pause
\medskip

The results are not true with strict inequalities.

\pause
\medskip 

\textbf{Example:}
Let $x_n \coloneqq \nicefrac{-1}{n}$ and $y_n \coloneqq \nicefrac{1}{n}$.

\pause
Then $x_n < y_n$, $x_n < 0$, and $y_n > 0$ for all $n$.

\pause
However, $\displaystyle \lim_{n\to\infty} x_n = \lim_{n\to\infty} y_n = 0$.

\pause
\medskip

Strict inequalities may become non-strict
inequalities when limits are applied.
\end{frame}

\begin{frame}
\begin{proposition}
Let $\{ x_n \}_{n=1}^\infty$ and $\{ y_n \}_{n=1}^\infty$ be convergent sequences.
\pause
\begin{enumerate}[(i)]
\item \label{prop:contalg:i}
$\{ x_n + y_n \}_{n=1}^\infty$ converges and
$\displaystyle
\lim_{n \to \infty} (x_n + y_n) = 
\lim_{n \to \infty} x_n + 
\lim_{n \to \infty} y_n$.
\item \label{prop:contalg:ii}
\pause
$\{ x_n - y_n \}_{n=1}^\infty$ converges and
$\displaystyle
\lim_{n \to \infty} (x_n - y_n) = 
\lim_{n \to \infty} x_n - 
\lim_{n \to \infty} y_n$.
\item \label{prop:contalg:iii}
\pause
$\{ x_n y_n \}_{n=1}^\infty$ converges and
$\displaystyle
\lim_{n \to \infty} (x_n y_n) = 
\left( \lim_{n \to \infty} x_n \right)
\left( \lim_{n \to \infty} y_n \right)$.
\item \label{prop:contalg:iv}
\pause
If ~$\displaystyle \lim_{n\to\infty} y_n \not= 0$~ and ~$y_n \not= 0$ for all $n \in \N$, then
$\left\{ \frac{x_n}{y_n} \right\}_{n=1}^\infty$ converges and

$\displaystyle
\lim_{n \to \infty} \frac{x_n}{y_n} = 
\frac{\lim_{n\to\infty} x_n}{\lim_{n\to\infty} y_n}$.
\end{enumerate}
\end{proposition}

\pause

Note that you can also use constant sequences.

\pause

E.g., if $c \in \R$ and $\{ x_n \}_{n=1}^\infty$ converges, then

$\displaystyle
\lim_{n \to \infty} c x_n = 
c \left( \lim_{n \to \infty} x_n \right), \qquad
\lim_{n \to \infty} (c + x_n) = 
c + \lim_{n \to \infty} x_n$, \qquad etc.

\end{frame}

\begin{frame}

\textbf{Proof:}
\eqref{prop:contalg:i} (addition).

\pause
\medskip

Suppose $\{ x_n \}_{n=1}^\infty$ and $\{ y_n \}_{n=1}^\infty$ are convergent,
~
$\displaystyle x \coloneqq \lim_{n\to\infty} x_n$,
~
$\displaystyle y \coloneqq \lim_{n\to\infty} y_n$.

\pause
\medskip

Let $\epsilon > 0$ be given.  

\pause
Find an $M_1$ such that $\forall$ $n \geq M_1$,
$\abs{x_n - x} < \nicefrac{\epsilon}{2}$.  

\pause
Find an $M_2$ such that $\forall$ $n \geq M_2$,
$\abs{y_n - y} < \nicefrac{\epsilon}{2}$.

\pause
Take $M \coloneqq \max \{ M_1, M_2 \}$.

\pause
\medskip

For all $n \geq M$,

\pause
\medskip

$\displaystyle
\abs{(x_n+y_n) - (x+y)}
\pause
=
\abs{x_n-x + y_n-y}
\pause
\leq
\abs{x_n-x} + \abs{y_n-y}
\pause
<
\frac{\epsilon}{2} +
\frac{\epsilon}{2}
= \epsilon.
$

\medskip
\pause

\eqref{prop:contalg:i} is proved.

\medskip
\pause

Proof of \eqref{prop:contalg:ii} (subtraction) is almost identical
(exercise).

\end{frame}

\begin{frame}

Next \eqref{prop:contalg:iii} (multiplication).

\pause
\medskip

Suppose $\{ x_n \}_{n=1}^\infty$ and $\{ y_n \}_{n=1}^\infty$ are convergent and
$\displaystyle x \coloneqq \lim_{n\to\infty} x_n$,
$\displaystyle y \coloneqq \lim_{n\to\infty} y_n$.

\pause
\medskip

Let $\epsilon > 0$ be given.

\pause
Let $K \coloneqq \max\{ \abs{x}, \abs{y}, \nicefrac{\epsilon}{3} , 1 \}$.

\pause
Find an $M_1$ such that $\forall$ $n \geq M_1$,
$\abs{x_n - x} < \frac{\epsilon}{3K}$.

\pause
Find an $M_2$ such that $\forall$ $n \geq M_2$,
$\abs{y_n - y} < \frac{\epsilon}{3K}$.

\pause
Take $M \coloneqq \max \{ M_1, M_2 \}$.

\pause
\medskip

For all $n \geq M$, 

\pause
\medskip

$\displaystyle
\abs{(x_ny_n) - (xy)}
\pause
=
\abs{(x_n-x+x)(y_n-y+y) - xy}
$

\pause
\medskip

$\displaystyle
\qquad
=
\abs{(x_n-x)y + x(y_n-y) +(x_n-x)(y_n-y)}
$

\pause
\medskip

$\displaystyle
\qquad
\leq
\abs{(x_n-x)y} + \abs{x(y_n - y)} +
\abs{(x_n-x)(y_n-y)} 
$

\pause
\medskip

$\displaystyle
\qquad
=
\abs{x_n -x}\abs{y} + 
\abs{x}\abs{y_n -y} + 
\abs{x_n -x}\abs{y_n -y}
$

\pause
\medskip

$\displaystyle
\qquad
<
\frac{\epsilon}{3K} K + 
K \frac{\epsilon}{3K} + 
\frac{\epsilon}{3K}
\frac{\epsilon}{3K}$
\qquad \qquad (now notice that $\tfrac{\epsilon}{3K} \leq 1$
and $K \geq 1$)

\pause
\medskip

$\displaystyle
\qquad
\leq
\frac{\epsilon}{3} + \frac{\epsilon}{3} + \frac{\epsilon}{3}
 = \epsilon$

\end{frame}

\begin{frame}

\eqref{prop:contalg:iv} follows from 
\eqref{prop:contalg:iii} and the claim:

\medskip

Claim: \emph{If $\{ y_n \}_{n=1}^\infty$ is convergent,
$\displaystyle \lim_{n\to\infty} y_n \not= 0$, and $y_n \not= 0$ for all $n \in \N$, then
$\{ \nicefrac{1}{y_n} \}_{n=1}^\infty$ converges and}
$\displaystyle \lim_{n\to\infty} \frac{1}{y_n} = \frac{1}{\lim_{n\to\infty} y_n}$.

\pause
\medskip

Proof of claim:  Let $\epsilon > 0$ be given.

\pause
Let $\displaystyle y \coloneqq \lim_{n\to\infty} y_n$.

\pause
$\abs{y} \not= 0$ \wthus $\min \left\{ \abs{y}^2\frac{\epsilon}{2}, \, \frac{\abs{y}}{2} \right\} > 0$.

\pause
Find an $M$ such that for all $n \geq M$,
\quad $\abs{y_n - y} < \min \left\{ \abs{y}^2\frac{\epsilon}{2}, \, \frac{\abs{y}}{2}
\right\}$.

\pause
\medskip

For $n \geq M$, ~
$\abs{y - y_n} < \nicefrac{\abs{y}}{2}$, so
\pause
~~
$\abs{y} = 
\abs{y - y_n + y_n } \leq
\abs{y - y_n} + \abs{ y_n } < \frac{\abs{y}}{2} + \abs{y_n}$.

\pause
\medskip

Subtract $\nicefrac{\abs{y}}{2}$ from both sides to get
$\nicefrac{\abs{y}}{2} < \abs{y_n}$, \quad or \quad
$\frac{1}{\abs{y_n}} < \frac{2}{\abs{y}}$

\pause
\medskip
$\displaystyle
\abs{\frac{1}{y_n} - \frac{1}{y}}
\pause
=
\abs{\frac{y - y_n}{y y_n}} 
\pause
=
\frac{\abs{y - y_n}}{\abs{y} \abs{y_n}} 
\pause
\leq
\frac{\abs{y - y_n}}{\abs{y}} \, \frac{2}{\abs{y}} 
\pause
<
\frac{\abs{y}^2 \frac{\epsilon}{2}}{\abs{y}} \, \frac{2}{\abs{y}}
\pause
= \epsilon$.
\qed

\end{frame}

\begin{frame}
\textbf{Exercise:} (induction) $\lim\limits_{n\to\infty} x_n^k =
{\bigl(\lim\limits_{n\to\infty} x_n\bigr)}^k$ for all $k \in \N$.
\pause

\begin{proposition}
Let $\{ x_n \}_{n=1}^\infty$ be convergent and
$x_n \geq 0$ for all $n \in \N$.
Then
$\displaystyle
\lim_{n\to\infty} \sqrt{x_n} =
\sqrt{ \lim_{n\to\infty} x_n }$.
\end{proposition}

\pause
Notice $\displaystyle \lim_{n\to\infty} x_n \geq 0$ so the square root makes sense.

\pause
\textbf{Proof:}
Let $\displaystyle x \coloneqq \lim_{n\to\infty} x_n$,
\pause
\quad   $x \geq 0$.

\pause
\medskip

First suppose $x=0$.

\pause
Let $\epsilon > 0$ be given.

\pause
Find $M$ such that $\forall$ $n \geq M$, ~
$x_n = \abs{x_n} < \epsilon^2$, ~or ~ $\sqrt{x_n} < \epsilon$.

\pause
\thus \quad
$\abs{\sqrt{x_n} - \sqrt{x}} =
\sqrt{x_n} < \epsilon$.

\pause
\medskip

Now suppose $x > 0$ (and hence $\sqrt{x} > 0$).
\pause
\begin{equation*}
\abs{\sqrt{x_n}-\sqrt{x}}
\pause
= 
\abs{\frac{x_n-x}{\sqrt{x_n}+\sqrt{x}}} 
\pause
=
\frac{1}{\sqrt{x_n}+\sqrt{x}}
\abs{x_n-x}
\pause
 \leq
\frac{1}{\sqrt{x}}
\abs{x_n-x} .
\end{equation*}
\pause
The rest is an exercise.
\qed
\end{frame}

\begin{frame}

\begin{proposition}
If $\{ x_n \}_{n=1}^\infty$ is a convergent sequence, then $\{ \abs{x_n}
\}_{n=1}^\infty$
is convergent and
\begin{equation*}
\lim_{n\to\infty} \abs{x_n} = 
\abs{\lim_{n\to\infty} x_n} .
\end{equation*}
\end{proposition}
\pause

\textbf{Proof sketch:}
The reverse triangle inequality: \quad
$\big\lvert \abs{x_n} - \abs{x} \big\rvert \leq \abs{x_n-x}$.
\qed

\end{frame}

\begin{frame}

\textbf{Example:}

Note that
$\lim\limits_{n\to\infty} \nicefrac{1}{n} = 0$.

\pause
Then
\begin{equation*}
\lim_{n\to \infty}
\abs{\sqrt{1 + \nicefrac{1}{n}} - \nicefrac{100}{n^2}} =  
\abs{\sqrt{1 + \Bigl(\lim_{n\to\infty} \nicefrac{1}{n}\Bigr)} -
100 \Bigl(\lim_{n\to\infty} \nicefrac{1}{n}\Bigr)\Bigl(\lim_{n\to\infty} \nicefrac{1}{n}\Bigr)} = 1.
\end{equation*}

\pause
Read this from right to left.  The propositions ~\thus~ LHS exists.

\pause
\medskip

\textbf{Example:}
Be careful:

\medskip

$\displaystyle
\lim_{n\to \infty} \left( \frac{n^2}{n+1} - n \right)
= -1$,
\quad
\pause
but
\quad
$\displaystyle \left(\lim_{n\to\infty} \frac{n^2}{n+1}\right) -
\left(\lim_{n\to\infty} n\right)$ is nonsense.

\end{frame}


\begin{frame}
\textbf{Example:}
Define $\{ x_n \}_{n=1}^\infty$ by $x_1 \coloneqq 2$ and
$\displaystyle x_{n+1} \coloneqq x_n - \frac{x_n^2-2}{2x_n}$.

\medskip
\pause

We must prove:

\pause
1) the sequence is well-defined,

\pause
2) the sequence converges,

\pause
3) only then try to find the limit.

\medskip
\pause

First, $x_1 = 2 > 0$ (so $x_2$ exists: $x_2 = 2-\frac{2^2-2}{2\cdot 2} = 1.5
> 0$)

\medskip
\pause

Suppose for some $n$, $x_n$ exists and $x_n > 0$.

\medskip
\pause

$\displaystyle
x_{n+1} = x_n - \frac{x_n^2-2}{2x_n}
\pause
=
\frac{2x_n^2 - x_n^2+2}{2x_n}
\pause
=
\frac{x_n^2+2}{2x_n}$.

\medskip
\pause

$x_n^2+2 > 0$ and $x_n > 0$,
\pause
and so
$x_{n+1} = \frac{x_n^2+2}{2x_n} > 0$.

\medskip
\pause

By induction $\{ x_n \}_{n=1}^\infty$ exists and $x_n > 0$ for all $n$.

\medskip
\pause

We claim $\{ x_n \}_{n=1}^\infty$ is monotone decreasing.

\medskip
\pause

If we show that $x_n^2-2 \geq 0$ for all $n$, then $x_{n+1} \leq x_n$ for all $n$.

\end{frame}

\begin{frame}
(recall: $x_1 \coloneqq 2$ and $x_{n+1} \coloneqq x_n - \frac{x_n^2-2}{2x_n}$)

\medskip
\pause

$x_1^2-2 = 4-2 = 2 > 0$

\medskip
\pause

$\displaystyle
x_{n+1}^2-2
\pause
=
{\left( \frac{x_n^2+2}{2x_n} \right)}^2 - 2
\pause
=
\frac{x_n^4+4x_n^2+4 - 8x_n^2}{4x_n^2}
\pause
=
\frac{x_n^4-4x_n^2+4}{4x_n^2}
\pause
=
\frac{{\left( x_n^2-2 \right)}^2}{4x_n^2}
\pause
\geq 0
$.

\medskip
\pause

So $\{ x_n \}_{n=1}^\infty$ is monotone decreasing and bounded below, it
must converge.

\medskip
\pause

Write
\[
2x_nx_{n+1} = x_n^2+2.
\]
%\medskip
\pause
$\{ x_{n+1} \}_{n=1}^\infty$ is the 1-tail of $\{ x_n \}_{n=1}^\infty$, so
it converges to the same limit, say $x$.
\pause
So
\[
2x^2 = x^2+2
\qquad
\pause
\Rightarrow
\qquad
x^2 = 2 .
\]
\pause
As $x_n > 0$ for all $n$, \quad $x \geq 0$.
\pause
\wthus $x = \sqrt{2}$.

\pause
\bigskip

You may have noticed the above is \emph{Newton's method} for finding
$\sqrt{2}$,

a common and practical way to find roots of equations.
\end{frame}

\begin{frame}
\textbf{Example:}
Suppose $x_1 \coloneqq 1$ and $x_{n+1} \coloneqq x_n^2+x_n$.

\medskip
\pause

Suppose we blindly write $\displaystyle x \coloneq \lim_{n\to\infty} x_n$.
\pause
Then $x_{n+1} = x_n^2+x_n$ gives
\[
x = x^2+x
\pause
\qquad
\Rightarrow
\qquad
x = 0 .
\]
\pause
But the sequence does \textbf{not} converge (unbounded).
\pause
$\displaystyle \lim_{n\to\infty} x_n$ does not exist.

\medskip
\pause

\textbf{Moral:} Before you can compute what a limit is, you need to know the
sequence converges.

\medskip
\pause

\textbf{More general moral:}
Before you prove that a limit actually exists do not do any calculations
with
\[
\lim_{n\to\infty} x_n .
\]
\pause
Just don't write ``$\lim\limits_{n\to\infty}$'' anywhere before you prove the limit
exists.
\end{frame}

\begin{frame}

\begin{proposition}
Let $\{ x_n \}_{n=1}^\infty$ be a sequence. 
Suppose $\exists$ $x \in \R$
and a convergent $\{ a_n \}_{n=1}^\infty$
such that
\quad
$\displaystyle \lim_{n\to\infty} a_n = 0$
\quad
and 
\quad
$\displaystyle
\abs{x_n - x} \leq a_n$
for all $n \in \N$.

\pause
Then $\{ x_n \}_{n=1}^\infty$ converges and $\displaystyle \lim_{n\to\infty} x_n = x$.
\end{proposition}

\pause
\textbf{Proof:}
Let $\epsilon > 0$ be given.

\pause
\medskip

Note $a_n \geq 0$ for all $n$.

\pause
\medskip

Find $M \in \N$ such that $a_n = \abs{a_n - 0} < \epsilon$ for all $n \geq M$.

\pause
\medskip

So for all
$n \geq M$, \quad
$\abs{x_n - x} \leq a_n < \epsilon$.
\qed

\end{frame}

\begin{frame}

\begin{proposition}
Let $c > 0$.
\pause
\begin{enumerate}[(i)]
\item
If $c < 1$, then
$\displaystyle
\lim_{n\to\infty} c^n = 0$.
\item
\pause
If $c > 1$, then $\{ c^n \}_{n=1}^\infty$ is unbounded.
\end{enumerate}
\end{proposition}

\pause
\textbf{Proof:}
First consider $c < 1$.

\pause
As $c > 0$, then $c^n > 0$ for all $n \in \N$ by induction.

\pause
As $c < 1$, then $c^{n+1} < c^n$ for all $n$.

\pause
\thus~ $\{ c^n \}_{n=1}^\infty$ is decreasing bounded below ~\thus~ it
converges.
\pause
Let $\displaystyle x \coloneqq \lim_{n\to\infty} c^n$.

\pause
The 1-tail $\{ c^{n+1} \}_{n=1}^\infty$ converges to $x$.

\pause
Take the limit of both sides of $c^{n+1} = c \cdot c^n$ \wthus  $x = cx$,
~or~ $(1-c)x=0$.

\pause
\thus \quad $x=0$ as $c \not= 1$.

\pause
\medskip

Now consider $c > 1$.
\pause
\quad
Let $B > 0$ be arbitrary.

\pause
As $\nicefrac{1}{c} < 1$, then $\bigl\{ {(\nicefrac{1}{c})}^n
\bigr\}_{n=1}^\infty$ converges to $0$.

\pause
\thus~ for some large enough $n$, ~
$\frac{1}{c^n} =
{\left(\frac{1}{c}\right)}^n < \frac{1}{B}$.

\pause
\thus ~ $c^n > B$, and $B$ is not an upper bound for $\{ c^n \}_{n=1}^\infty$.

\pause
As $B$ was arbitrary, $\{ c^n \}_{n=1}^\infty$ is unbounded.
\qed

\end{frame}

\begin{frame}

\begin{lemma}[Ratio test for sequences]
Let $\{ x_n \}_{n=1}^\infty$ be such that $x_n \not= 0$ for all $n$ and
$\displaystyle L \coloneqq \lim_{n\to\infty} \frac{\abs{x_{n+1}}}{\abs{x_n}}$ exists.
\pause
\begin{enumerate}[(i)]
\item
If $L < 1$, then $\{ x_n \}_{n=1}^\infty$ converges and $\displaystyle \lim_{n\to\infty} x_n = 0$.
\item
\pause
If $L > 1$, then $\{ x_n \}_{n=1}^\infty$ is unbounded (hence diverges).
\end{enumerate}
\end{lemma}

\pause
If $L=1$, no conclusion. E.g.,

\pause
$\{ \nicefrac{1}{n} \}_{n=1}^\infty$ converges to zero, but $L=1$.
\pause
\quad\qquad~\,
$\{ 1 \}_{n=1}^\infty$ converges to 1, and $L=1$.

\pause
$\{ {(-1)}^n \}_{n=1}^\infty$ does not converge, and $L=1$.
\pause
\qquad
$\{  n \}_{n=1}^\infty$ is unbounded, and $L=1$.

\pause
\medskip

\textbf{Proof:}
Suppose $L < 1$.

\pause
\medskip

$\frac{\abs{x_{n+1}}}{\abs{x_n}} \geq 0$ for all $n$ \wthus $L \geq 0$.
\qquad
\pause
Pick $r$ such that $L < r < 1$.

\pause
\medskip

$\frac{\abs{x_{n+1}}}{\abs{x_n}}$ will eventually be less than $r$ and
so we compare the sequence to $\{ r^n \}_{n=1}^\infty$.

\pause
\begin{center}
\scalebox{0.9}{
\subimport*{../figures/}{figratseq.pdf_t}
}
\end{center}

\end{frame}

\begin{frame}
As $r-L > 0$, $\exists$ $M \in \N$ such that for
all $n \geq M$,
\begin{equation*}
\abs{\frac{\abs{x_{n+1}}}{\abs{x_n}} - L} < r-L .
\end{equation*}
\pause
\thus \quad for $n \geq M$,
\begin{equation*}
\frac{\abs{x_{n+1}}}{\abs{x_n}} - L < r-L 
\pause
\qquad \text{or} \qquad
\frac{\abs{x_{n+1}}}{\abs{x_n}} < r .
\end{equation*}
\pause
For $n > M$ ($n \geq M+1$)
\begin{equation*}
\abs{x_n} =
\abs{x_M}
\frac{\abs{x_{M+1}}}{\abs{x_{M}}}
\frac{\abs{x_{M+2}}}{\abs{x_{M+1}}}
\cdots
\frac{\abs{x_{n}}}{\abs{x_{n-1}}}
\pause
<
\abs{x_M}
r r \cdots r
\pause
= \abs{x_M} r^{n-M} = (\abs{x_M} r^{-M}) r^n .
\end{equation*}
\pause
$\{ r^n \}_{n=1}^\infty$ converges to zero
\pause
\wthus
$\{ \abs{x_M} r^{-M} r^n \}_{n=1}^\infty$ converges to zero.
  
\pause
\medskip

The $M$-tail of $\{x_n\}_{n=1}^\infty$ converges to zero \wthus
$\{x_n\}_{n=1}^\infty$ converges to zero.

\end{frame}

\begin{frame}
Now suppose $L > 1$.
\pause
\quad
Pick $r$ such that $1 < r < L$.
 
\pause
\medskip

As $L-r > 0$,
$\exists$ $M \in \N$ such that for
all $n \geq M$
\begin{equation*}
\abs{\frac{\abs{x_{n+1}}}{\abs{x_n}} - L} < L-r .
\end{equation*}
\pause
\thus
\begin{equation*}
\frac{\abs{x_{n+1}}}{\abs{x_n}} > r .
\end{equation*}
\pause
For $n > M$,
\begin{equation*}
\abs{x_n} =
\abs{x_M}
\frac{\abs{x_{M+1}}}{\abs{x_{M}}}
\frac{\abs{x_{M+2}}}{\abs{x_{M+1}}}
\cdots
\frac{\abs{x_{n}}}{\abs{x_{n-1}}}
\pause
>
\abs{x_M}
r r \cdots r
\pause
= \abs{x_M} r^{n-M} = (\abs{x_M} r^{-M}) r^n .
\end{equation*}
\pause
$\{ r^n \}_{n=1}^\infty$ is unbounded ($r > 1$)
\pause
\wthus
$\{x_n\}_{n=1}^\infty$ is not bounded

\pause
(if $\abs{x_n} \leq B$ $\forall n$ ~\thus~
$r^n < \frac{B}{\abs{x_M}} r^{\:M}$ for all $n > M$, \contradiction).

\pause
\medskip
\thus \quad $\{ x_n \}_{n=1}^\infty$ diverges.
\qed

\end{frame}

\begin{frame}

\textbf{Example:}
Claim:
$\displaystyle
\lim_{n\to\infty} \frac{2^n}{n!} = 0$.

\pause
\medskip

Proof:
\quad
$\displaystyle
\frac{2^{n+1} / (n+1)!}{2^n/n!}
\pause
=
\frac{2^{n+1}}{2^n}\frac{n!}{(n+1)!}
\pause
=
\frac{2}{n+1} \to 0$.
\pause
\quad Conclusion follows.

\pause
\medskip

\textbf{Example:}
Claim:
$\displaystyle
\lim_{n\to\infty} n^{1/n} = 1$.

\pause
Proof:
Let $\epsilon > 0$ be given. 
\pause
Consider
$\bigl\{ \frac{n}{{(1+\epsilon)}^n} \bigr\}_{n=1}^\infty$.
\pause
Compute

\quad
$\displaystyle
\frac{(n+1)/{(1+\epsilon)}^{n+1}}{n/{(1+\epsilon)}^{n}}
=
\frac{n+1}{n} \frac{1}{1+\epsilon} .
$

\pause
\medskip

limit of $\frac{n+1}{n} = 1+\frac{1}{n}$ is $1$
\pause
\wthus
$\displaystyle
\lim_{n\to \infty} \frac{(n+1)/{(1+\epsilon)}^{n+1}}{n/{(1+\epsilon)}^{n}}
=
\frac{1}{1+\epsilon}  \pause < 1$.

\pause
\medskip

\thus 
\quad
$\displaystyle \lim_{n\to\infty} \frac{n}{{(1+\epsilon)}^n} = 0$

\pause
\thus \quad
$\exists$ $M$, s.t. for $n \geq M$,
$\frac{n}{{(1+\epsilon)}^n} < 1$,
\pause
\quad or \quad
$n < {(1+\epsilon)}^n$,
\pause
\quad or \quad
$n^{1/n} < 1+\epsilon$.

\pause
\medskip

$n^{1/n} \geq 1$ (as $n \geq 1$)
\pause
\wthus
$0 \leq n^{1/n}-1 < \epsilon$.

\pause
\medskip

\thus \quad
$\displaystyle \lim_{n\to\infty} n^{1/n} = 1$.

\end{frame}

%\begin{frame}
%\textbf{Exercise:}
%Prove the following stronger version of ratio test.
%
%\medskip
%
%Suppose $\{ x_n \}_{n=1}^\infty$ is such that $x_n \not= 0$ for all
%$n$.
%\begin{enumerate}[a)]
%\item
%if $\exists$ $r < 1$ and $M \in \N$ such that
%for all $n \geq M$, ~
%$\displaystyle
%\frac{\abs{x_{n+1}}}{\abs{x_n}} \leq r$,
%
%then $\{ x_n \}_{n=1}^\infty$ converges to $0$.
%\item
%If $\exists$ $r > 1$ and $M \in \N$ such that
%for all $n \geq M$, ~
%$\displaystyle
%\frac{\abs{x_{n+1}}}{\abs{x_n}} \geq r$,
%
%then $\{ x_n \}_{n=1}^\infty$ is unbounded.
%\end{enumerate}
%\end{frame}

\end{document}
