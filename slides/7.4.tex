\documentclass[10pt,aspectratio=149]{beamer}

% All the boilerplate is in raslides.sty
% Note that this also pulls in a custom vogtwidebar.sty
\usepackage{raslides}

\author{Ji\v{r}\'i Lebl}

\institute[OSU]{%
Departemento pri Matematiko de Oklahoma {\^S}tata Universitato}

\title{BA: 7.4}

\date{}

\begin{document}

\begin{frame}
\titlepage
\end{frame}

\begin{frame}
\begin{definition}
A sequence $\{ x_n \}_{n=1}^\infty$ in a metric space $(X,d)$ is \emph{Cauchy}
\pause
if
for every $\epsilon > 0$, there exists an $M \in \N$ such that
for all $n \geq M$ and all $k \geq M$,
\quad $d(x_n, x_k) < \epsilon$.
\end{definition}

\pause
\begin{proposition}
A convergent sequence in a metric space is Cauchy.
\end{proposition}

\pause
\textbf{Proof:}
Suppose $\{ x_n \}_{n=1}^\infty$ converges to $p$.

\pause
\medskip

\pause
Given $\epsilon > 0$, $\exists$ $M$ such that $\forall$ $n \geq M$,
~~$d(p,x_n) < \nicefrac{\epsilon}{2}$.

\pause
\medskip

\thus \quad $\forall$ $n,k \geq M$, ~~
$d(x_n,x_k)
\pause
\leq d(x_n,p) + d(p,x_k)
\pause
< \nicefrac{\epsilon}{2} + \nicefrac{\epsilon}{2}
\pause
= \epsilon$.
\qed

\pause
\begin{definition}
A metric space $(X,d)$ is
\emph{complete} or \emph{Cauchy-complete}
if every Cauchy sequence $\{ x_n \}_{n=1}^\infty$ in $X$
converges to a $p \in X$.
\end{definition}

\end{frame}

\begin{frame}

\begin{proposition}
The space $\R^n$ with the standard metric is a complete metric space.
\end{proposition}

\pause
Already proved for $\R = \R^1$.

\pause
\medskip

\textbf{Proof:}
Let $\{ x_m \}_{m=1}^\infty$ in $\R^n$ be Cauchy,
\pause
\quad write 
$x_m = \bigl(x_{m,1},x_{m,2},\ldots,x_{m,n}\bigr)$.

\pause
\medskip

Given $\epsilon > 0$, $\exists$ $M$ such that $\forall$
$i,j \geq M$,
\quad 
$d(x_i,x_j) < \epsilon$.

\pause
\medskip

Fix $k=1,2,\ldots,n$.
\pause
For $i,j \geq M$,
\[
\bigl\lvert x_{i,k} - x_{j,k} \bigr\rvert
\pause
=
\sqrt{{\bigl(x_{i,k} - x_{j,k}\bigr)}^2}
\pause
\leq
\sqrt{\sum_{\ell=1}^n {\bigl(x_{i,\ell}-x_{j,\ell}\bigr)}^2}
\pause
= d(x_i,x_j)
\pause
< \epsilon .
\]
\pause
\thus \quad
$\{ x_{m,k} \}_{m=1}^\infty$ is Cauchy.

\pause
\medskip

$\R$ is complete
\pause
\wthus
$\{ x_{m,k} \}_{m=1}^\infty$ converges to some $y_k \in \R$.

\pause
\medskip

Write $y = (y_1,y_2,\ldots,y_n) \in \R^n$.

\pause
By a proposition, $\{ x_m \}_{m=1}^\infty$ converges
to $y$
\pause
\wthus $\R^n$ is complete.
\qed

\end{frame}

\begin{frame}

%In the language of metric spaces,
%the results on continuity of section {sec:liminter},
%say that the metric space
%$C\bigl([a,b],\R\bigr)$ of {example:msC01} is complete.
%The proof follows by ``unrolling the definitions,'' 
%and is left as {exercise:CabRcomplete}.
%
%\begin{proposition}
%The space of continuous functions $C\bigl([a,b],\R\bigr)$ with the uniform
%norm as metric is a complete metric space.
%\end{proposition}

\textbf{Example:}
A subset of a complete metric space need not be complete:

\pause
$(0,1]$ with the subspace metric is not complete.

\pause
$\{ \nicefrac{1}{n} \}_{n=1}^\infty$ is Cauchy
with no limit in $(0,1]$.

\pause
\begin{proposition}
Suppose $(X,d)$ is a complete metric space and $E \subset X$
is closed.
\pause
Then $E$ is a complete metric space with the subspace metric.
\end{proposition}

\pause
\textbf{Proof:} Exercise.

\end{frame}

\begin{frame}

\begin{definition}
Let $(X,d)$ be a metric space and $K \subset X$. 
\pause
The set $K$ is \emph{compact}
if for every collection
of open sets $\{ U_{\lambda} \}_{\lambda \in I}$ such that
\quad
$\displaystyle
K \subset \bigcup_{\lambda \in I} U_\lambda$,

\pause
there exists a finite subset
$\{ \lambda_1, \lambda_2,\ldots,\lambda_m \} \subset I$
such that
\quad
$\displaystyle
K \subset \bigcup_{j=1}^m U_{\lambda_j}$.
\end{definition}

\pause
Call $\{ U_{\lambda} \}_{\lambda \in I}$ an \emph{open cover} of $K$.


\pause
$K$ is compact if \emph{every open cover of $K$ has a finite subcover}.

\pause
\medskip

\textbf{Example:}
$\R$ is not compact.

\pause
Proof: Take $U_j \coloneqq (-j,j)$, ~~ $j \in \N$.
\pause
\quad
$\bigcup_j (-j,j) = \R$ by Archimedean property.

\pause
If $\R \subset U_{j_1} \cup U_{j_2} \cup \cdots \cup U_{j_m}$ ~
($j_1 < j_2 < \cdots < j_m$) \wthus $\R \subset U_{j_m}$
\pause
\quad
\contradiction

\pause
\medskip

\textbf{Example:}
$(0,1) \subset \R$ is not compact.

\pause
Proof:  Consider $U_{j} \coloneqq (\nicefrac{1}{j},1-\nicefrac{1}{j})$ for $j=3,4,5,\ldots$.
\pause
\quad
$(0,1) = \bigcup_{j=3}^\infty U_j$.

\pause
Again, if $\exists$ finite subcover then $(0,1) \subset U_j$ for some
$j$,
\pause
\qquad \contradiction

\pause
\medskip

\textbf{Example:}
$\{ 0 \} \subset \R$ is compact

\pause
Proof: For an open cover $\{ U_{\lambda} \}_{\lambda \in I}$,
\pause
~$\exists$ $\lambda_0$ s.t. $0 \in U_{\lambda_0}$.
\pause
~$\{ U_{\lambda_0} \}$ is a finite subcover.

\end{frame}

\begin{frame}

\begin{proposition}
Let $(X,d)$ be a metric space.  If $K \subset X$ is compact, then $K$ is closed and
bounded.
\end{proposition}

\pause
\textbf{Proof:}
Fix $p \in X$.

\pause
The open cover ~~
$\displaystyle K \subset \bigcup_{n=1}^\infty B(p,n) = X$

\pause
\medskip

has a finite subcover

\medskip

$\displaystyle K \subset \bigcup_{j=1}^m B(p,n_j)
\pause
= B(p,n_m)$.

\pause
\medskip

\thus \quad $K$ is bounded.

\pause
\vspace*{-1.5in}
\hspace*{2.6in}
\scalebox{0.9}{
\subimport*{../figures/}{compactbnd.pdf_t}
}

\end{frame}

\begin{frame}
Suppose $K$ is not closed: $\widebar{K} \not= K$

\pause
\thus \quad $\exists$ $x \in \widebar{K} \setminus K$.

\pause
\medskip

\pause
$x \notin K$
\pause
~\&~ if $y\not= x$
\thus
$y \in {C(x,\nicefrac{1}{n})}^c$ for some $n$

\pause
\medskip

\pause
\thus \quad
$\displaystyle
K \subset \bigcup_{n=1}^\infty {C(x,\nicefrac{1}{n})}^c$

\pause
\vspace*{-1.2in}
\hspace*{2.75in}
\scalebox{0.8}{
\subimport*{../figures/}{compactclosed.pdf_t}
}

\vspace*{-0.1in}

\pause
That's an open cover as 
${C(x,\nicefrac{1}{n})}^c$ are open.

%If $y \not= x$, then
%$y \notin C(x,\nicefrac{1}{n})$
%for $n \in \N$
%such that $\nicefrac{1}{n} < d(x,y)$.
%Furthermore, $x \notin K$, so
%\begin{equation*}
%\end{equation*}

\pause
\medskip

Take a finite collection
\quad
$\displaystyle
\bigcup_{j=1}^m {C(x,\nicefrac{1}{n_j})}^c 
\pause
=
{C(x,\nicefrac{1}{n_m})}^c 
$

\pause
\medskip

$x \in \overline{K}$
\pause
\wthus
$C(x,\nicefrac{1}{n_m}) \cap K \not= \emptyset$.

\pause
\medskip

\thus \quad so no finite subcover
\pause
\wthus 
$K$ is not compact.
\qed


\end{frame}

\begin{frame}

%We prove below that 
%in a finite-dimensional euclidean space,
%every closed bounded set is compact.
%So closed bounded sets
%of $\R^n$ are examples of compact sets.
%It is not true that in every metric space, closed and bounded is equivalent
%to compact.  A simple example is an incomplete metric space such as
%$(0,1)$ with the subspace metric from $\R$.
%There are many complete and very useful metric spaces
%where closed and bounded is not
%enough to give compactness: $C\bigl([a,b],\R\bigr)$ is a complete metric
%space, but the closed unit ball $C(0,1)$ is not compact, see
%{exercise:msclbounnotcompt}.  However, see also
%{exercise:mstotbound}.

%A useful property of compact sets in a metric space is that every
%sequence in the set has a convergent subsequence converging
%to a point in the set.
%Such sets are called
%\emph{sequentially compact}.
%We will prove that in the
%context of metric spaces, a set is compact if and only if it is sequentially
%compact.
%First we prove a lemma.

\begin{lemma}[Lebesgue covering lemma]
Let $(X,d)$ be a metric space and $K \subset X$.
\pause
Suppose 
every sequence in $K$ has a subsequence convergent in $K$.
\pause
Given
an open cover $\{ U_\lambda \}_{\lambda \in I}$ of $K$, there exists a
$\delta > 0$ such that for every $x \in K$, there is a $\lambda \in I$
with $B(x,\delta) \subset U_\lambda$.
\end{lemma}

\pause
\textbf{Proof:}
If conclusion not true, then $\exists$
open cover $\{ U_\lambda \}_{\lambda \in I}$ of $K$ such that:

\pause
\medskip

For every $n \in \N$, $\exists$ $x_n \in K$ s.t.
$B(x_n,\nicefrac{1}{n})$ is not a subset of any $U_\lambda$.

\pause
\medskip

Take $x \in K$ \wthus $\exists$
$\lambda \in I$ s.t. $x \in U_\lambda$.

\pause
$U_\lambda$ is open \wthus
$\exists$ $\epsilon > 0$ s.t.  $B(x,\epsilon) \subset U_\lambda$.

\pause
\medskip

Take $M$ s.t.  $\nicefrac{1}{M} < \nicefrac{\epsilon}{2}$.

\pause
\medskip

If $y \in B(x,\nicefrac{\epsilon}{2})$ and $n \geq M$, then 

\pause
\medskip

$
B(y,\nicefrac{1}{n})
\pause
\subset
B(y,\nicefrac{1}{M})
$

\pause
\qquad $
\subset
B(y,\nicefrac{\epsilon}{2})
\pause
\subset
B(x,\epsilon)
\pause
\subset
U_\lambda$,

\pause
\vspace*{-0.9in}
\hspace*{2.3in}
\scalebox{0.8}{
\subimport*{../figures/}{lebesguedelta.pdf_t}
}
\vspace*{-0.5in}

\pause
($B(y,\nicefrac{\epsilon}{2}) \subset B(x,\epsilon)$ follows by

triangle inequality.)

\pause
\medskip

\thus \quad $y \not= x_n$ \wthus $\{ x_n \}_{n=1}^\infty$
has no subsequence converging to $x$.
\qed

\end{frame}

\begin{frame}

\textbf{Remark:} If $K$ such that every sequence has a convergent subsequence in
$K$ is called \emph{sequentially compact}.

\pause
\medskip

The $\delta$ in the lemma depends on the cover, but not on the $x$.

\pause
\medskip

\textbf{Example:} $K \coloneqq [-10,10]$ and for $n \in \Z$,~
$U_n \coloneqq (n,n+2)$ give an open cover.

\pause
If $x \in K$, then $\exists$ $n \in \Z$, such that $n \leq x < n+1$.

\pause
If $n \leq x < n+\nicefrac{1}{2}$, then
$B\bigl(x,\nicefrac{1}{2}\bigr) \subset U_{n-1}$.

\pause
If $n+ \nicefrac{1}{2} \leq x < n+1$, then
$B\bigl(x,\nicefrac{1}{2}\bigr) \subset U_{n}$.

\pause
\thus \quad $\delta = \nicefrac{1}{2}$ works.

\pause
\medskip

For the cover given by
$U'_n \coloneqq \bigl(\frac{n}{2},\frac{n+2}{2} \bigr)$
the largest $\delta$ is
$\nicefrac{1}{4}$.

\pause
\medskip

\textbf{Example:}
$\N \subset \R$ is not sequentially compact.

\pause
\textbf{Exercise:} Find the cover with no $\delta > 0$.

\end{frame}

\begin{frame}

\begin{theorem}
Let $(X,d)$ be a metric space.  Then $K \subset X$ is compact if
and only if every sequence in $K$ has a subsequence converging to
a point in $K$.
\end{theorem}

\pause
\medskip

\textbf{Proof:}
Claim: \emph{Suppose $K \subset X$ and
$\{ x_n \}_{n=1}^\infty$ is a sequence in $K$.  Suppose $\forall$ $x \in K$,

$\exists$ $\alpha_x > 0$ s.t.
$x_n \in B(x,\alpha_x)$ for only finitely many $n \in \N$.
Then $K$ is not compact.}

\pause
Proof of the claim:
Notice
$
K \subset \bigcup_{x \in K} B(x,\alpha_x)$.

\pause
Any finite subcollection contains $x_n$ for only finitely many $n$.

\pause
\thus \quad there is no subcover
\pause
\wthus $K$ is not compact.

\pause
\medskip

Suppose $K$ is compact and $\{ x_n \}_{n=1}^\infty$ is a sequence in $K$.

\pause
By claim $\exists$ $x \in K$ s.t. $\forall$ $\delta > 0$,
$B(x,\delta)$ contains $x_n$ for infinitely many $n \in \N$.

\pause
\medskip

$B(x,1)$ contains some $x_k$ \wthus let $n_1 \coloneqq k$.

\pause
Suppose $n_{j-1}$ is defined.

\pause
$\exists$ $k > n_{j-1}$ such that $x_k \in B(x,\nicefrac{1}{j})$
\wthus let $n_j \coloneqq k$.

\pause
\medskip

$d(x,x_{n_j}) < \nicefrac{1}{j}$
\wthus
$\{ x_{n_j} \}_{j=1}^\infty$ converges to $x$.


\end{frame}

\begin{frame}

Now suppose $K$ is sequentially compact and
$\{ U_\lambda \}_{\lambda \in I}$ is an open cover of $K$.

\pause
By Lebesgue covering lemma, $\exists$ $\delta > 0$
s.t. $\forall$ $x \in K$, $\exists$ $\lambda \in I$ with
$B(x,\delta) \subset U_\lambda$.

\pause
\medskip

Pick $x_1 \in K$ and find $\lambda_1 \in I$ s.t. $B(x_1,\delta) \subset U_{\lambda_1}$.

\pause
\medskip

If $K \subset U_{\lambda_1}$, we are done.

\pause
Otherwise, $\exists$ $x_2 \in K \setminus U_{\lambda_1}$.
\pause
\qquad
Note $d(x_2,x_1) \geq \delta$.

\pause
$\exists$ $\lambda_2 \in I$ such that $B(x_2,\delta) \subset U_{\lambda_2}$.
\pause
\qquad
If $K \subset U_{\lambda_1} \cup U_{\lambda_2}$, then done.

\pause
\medskip

Suppose $\lambda_{n-1}$ is defined.

\pause
Either
$K \subset U_{\lambda_1} \cup
%U_{\lambda_2} \cup
\cdots \cup
U_{\lambda_{n-1}}$
(done),
\pause
or
$\exists$ $x_n \in
K \setminus \bigl( U_{\lambda_1} \cup
%U_{\lambda_2} \cup
\cdots \cup U_{\lambda_{n-1}}\bigr)$.

\pause
Note $d(x_n,x_j) \geq \delta$ for all $j = 1,2,\ldots,n-1$.

\pause
$\exists$
$\lambda_n \in I$ s.t. $B(x_n,\delta) \subset U_{\lambda_n}$.

\pause
\vspace*{-0.3in}
\hspace*{2in}
\scalebox{0.7}{
\subimport*{../figures/}{seqcompactiscompact.pdf_t}
}

%Covering $K$ by $U_{\lambda}$.  The points
%$x_1,x_2,x_3,x_4$, 
%the three sets 
%$U_{\lambda_1}$,
%$U_{\lambda_2}$,
%$U_{\lambda_2}$,
%and 
%the first three balls
%of radius $\delta$ are drawn.


\vspace*{-0.8in}

\pause
Either we find a subcover,

\pause
or we obtain a sequence
$\{ x_n \}_{n=1}^\infty$

in $K$ such that 
$d(x_n,x_k) \geq \delta$

whenever $k \not= n$.

\pause
\medskip

No subsequence of $\{ x_n \}_{n=1}^\infty$ can be
Cauchy.
\pause
\quad \contradiction
\qed

\end{frame}

\begin{frame}

\textbf{Example:}
Bolzano--Weierstrass theorem for sequences of real numbers:

Every bounded sequence in $\R$ has a convergent subsequence.

\pause
\thus \quad Every sequence in $[a,b] \subset \R$ has 
a convergent subsequence.

\pause
\thus \quad The limit is also in $[a,b]$.

\pause
\thus \quad A closed bounded interval $[a,b] \subset \R$ is (sequentially) compact.

\pause
\begin{proposition}
Let $(X,d)$ be a metric space and let $K \subset X$ be compact.  If
$E \subset K$ is a closed set, then $E$ is compact.
\end{proposition}

\pause
As $K$ is closed, $E$ is closed in $K$ \wiffif
$E$ is closed in $X$.

\pause
\medskip

\textbf{Proof:}
Let $\{ x_n \}_{n=1}^\infty$ be a sequence in $E$.

\pause
$\{ x_n \}_{n=1}^\infty$ is a sequence in $K$.

\pause
\thus \quad $\exists$ a convergent $\{ x_{n_j} \}_{j=1}^\infty$ converging
to some $p \in K$.

\pause
$E$ is closed \wthus $p \in E$.

\pause
\thus \quad $E$ is (sequentially) compact.
\qed

\end{frame}

\begin{frame}

\begin{theorem}[Heine--Borel]
A closed bounded subset $K \subset \R^n$ is compact.
\end{theorem}

\pause
\textbf{Warning:}
Heine--Borel only holds for $\R^n$ and not for metric spaces in general.

\pause
The theorem does not hold even for subspaces of $\R^n$, just in $\R^n$
itself.

\pause
\medskip

\textbf{Example:} In $X = (0,\infty)$ the subset $(0,1]$ is closed and
bounded and \textbf{not} compact.

\pause
\medskip

In general, compact \thus closed and bounded, \textbf{but not vice versa!}

\pause
\medskip

As there's always someone who wants to use this:

\pause
\medskip

{\Huge
\textbf{Closed and bounded does not imply compact in general!}
}

\end{frame}

\begin{frame}

\textbf{Proof:}
Suppose $n=1$.

\pause
$K \subset [a,b]$.
\pause
\qquad
$[a,b]$ is compact.
\pause
\qquad
$K$ is a closed subset of a compact set.

\pause
\thus \quad $K$ is compact.

\pause
\medskip

Now consider $n=2$.
\pause
\quad
Arbitrary $n$ is an exercise.

\pause
If $K \subset \R^2$ is bounded, $\exists$
$B = [a,b]\times[c,d] \subset \R^2$ such that $K \subset B$.

\pause
If $B$ is compact, then as $K$ is closed it is also compact.

\pause
\medskip

Let $\bigl\{ (x_k,y_k) \bigr\}_{k=1}^\infty$ be a sequence in $B$.

\pause
That is, $a \leq x_k \leq b$ and $c \leq y_k \leq d$ for all $k$.

\pause
By BW, $\exists$ convergent subsequence
$\{ x_{k_j} \}_{j=1}^\infty$.

\pause
$\{ y_{k_j} \}_{j=1}^\infty$ is also a bounded, so 
$\exists$ convergent
subsequence
$\{ y_{k_{j_i}} \}_{i=1}^\infty$.

\pause
$\{ x_{k_{j_i}} \}_{i=1}^\infty$ is also convergent.

\pause
Let
~~
$x \coloneqq \lim\limits_{i\to\infty} x_{k_{j_i}}$
~~
and
~~
$y \coloneqq \lim\limits_{i\to\infty} y_{k_{j_i}}$.

\pause
\medskip

So 
$\bigl\{ (x_{k_{j_i}},y_{k_{j_i}}) \bigr\}_{i=1}^\infty$ converges to $(x,y)$.

\pause
$a \leq x_k \leq b$ and
$c \leq y_k \leq d$ for all $k$ \wthus $(x,y) \in B$.

\pause
\medskip

$B$ is compact \wthus $K$ is compact.
\qed

\end{frame}

\begin{frame}

\textbf{Example:}
Let $(X,d)$ be a metric space with the discrete metric

($d(x,y) = 1$ if $x \not= y$).

\pause
Suppose $X$ is an infinite set.  Then
\begin{enumerate}[(i)]
\item
\pause
$(X,d)$ is a complete metric space.
\item
\pause
Any $K \subset X$ is closed and bounded.
\item
\pause
$K \subset X$ is compact \wiffif $K$ is a finite set.
\item
\pause
The conclusion of the Lebesgue covering lemma is always satisfied,
e.g. with $\delta = \nicefrac{1}{2}$,
even for noncompact $K \subset X$.
\end{enumerate}

\pause
\textbf{Proof:} Exercises.

\pause
\medskip

\textbf{Remark:}
Compactness only depends on topology (on the set of open sets).

\pause
\medskip

Completeness depends on which sequences are Cauchy, so
it depends on the actual metric.

\end{frame}

\begin{frame}

\textbf{Exercise:}
Finite sets are compact.

\pause
\medskip

\textbf{Exercise:}
The union of finitely many compact sets is compact.

\pause
\medskip

\textbf{Exercise:}
A compact set is a complete metric space.

\pause
\medskip

\textbf{Exercise:}
Suppose $(X,d)$ is complete and
$E_1 \supset E_2 \supset E_3 \supset \cdots$ are compact and nonempty.
Prove $\bigcap_{j=1}^\infty E_j \not= \emptyset$.

%\begin{exercise}[Challenging]
%Let $(X,d)$ be a complete metric space.
%Show that $K \subset X$ is compact if and only if $K$ is closed
%and such that for every $\epsilon > 0$
%there exists a finite set of points $x_1,x_2,\ldots,x_n$ with
%$K \subset \bigcup_{j=1}^n B(x_j,\epsilon)$.
%Note: Such a set $K$ is said to be \emph{totally bounded},
%so in a complete metric space a set is compact if and only
%if it is closed and totally bounded.
%\end{exercise}

\pause
\medskip

\textbf{Exercise:}
Prove the general Bolzano--Weierstrass theorem:
Any bounded sequence $\{ x_k \}_{k=1}^\infty$ in $\R^n$
has a convergent subsequence.

\pause
\medskip

\textbf{Exercise:}
Let $(X,d)$ be a metric space and $K \subset X$.
Prove that $K$ is compact as a subset of $(X,d)$ if and only if $K$ is
compact as a subset of itself with the subspace metric.

\end{frame}

\end{document}
